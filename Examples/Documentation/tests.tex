\chapter{Regression tests}\label{chap:test}
\textcolor{red}{Ich habe die Ueberschrift geandert}
The DOpE test suite consists of regression tests. They are run to compare 
the output to previous outputs. This is useful (necessary) after 
changing programming code anywhere in the library. If a test
succeeds, everything is fine in the library. If not, you should not
check in your code into DOpE. Please make sure what is going wrong and WHY!

Every command is computed via a Makefile. In the basic example 
directory.

\textcolor{red}{
\begin{remark}
Please keep in mind that there are two optimization examples that 
need external packages to be able to run successfully. These tests will
fail if these libraries are not installed!
\end{remark}
}

%%%%%%%%%%%%%%%%%%%%%%%%%%%%%%%%%%%%%%%%%%%%%%%%%%
\section{Where can I find the tests}
In each example directory you find a sub directory `Test'. Herein, you find 
the parameter files for meshes (*.inp) and a param file (test.prm).
Moreover, the executable is denoted by `test.sh'. Please make sure, that 
the 
\begin{verbatim}
set never_write_list
\end{verbatim}
contains every possible output
\begin{verbatim}
Gradient;Hessian;Tangent;Residual;Update;Control;State
\end{verbatim}
That means, no solution files are written to the output.
Recall, that we are just interested in terminal output that 
is of course sufficient to verify the things.

Hence, the results directory should be empty
\begin{verbatim}
set results_dir   = ./
\end{verbatim}
The rest in the param file must be identically the same as 
in the dope.prm file in the parent directory. 

%%%%%%%%%%%%%%%%%%%%%%%%%%%%%%%%%%%%%%%%%%%%%%%%%%
\section{How to start testing?}
You start testing by typing 
\begin{verbatim}
> ./test.sh Store
\end{verbatim}
in the terminal.

After the run, you have to call 
\begin{verbatim}
> ./test.sh Test
\end{verbatim}
to compare your stored output. Of course, there should be no differences.

The useful point is now the following. After implementation of 
new pieces of code in the DOpE library or in the examples, you can
run 
\begin{verbatim}
> ./test.sh Test
\end{verbatim}
Hereby, you compare your `new' output with the previous stored output.

Attention: After changes you should NOT run again
\begin{verbatim}
> ./test.sh Store
\end{verbatim}
In that case, you overwrite your previous output.
