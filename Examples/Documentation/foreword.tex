\chapter{Foreword}
In this report, we describe the DOpE
\textit{Deal Optimization Environment} project. 
Originally, the project was initiated in the year 2009 at the 
University of Heidelberg (Germany) in the numerical analysis
group of Rolf Rannacher. 

%%%%%%%%%%%%%%%%%%%%%%%%%%%%%%%%%%%%%%%%%%%%%%%%%%
The \textit{Deal Optimization Environment} (DOpE) project is mainly 
based on the
\textit{deal.II} finite element library which has been developed initially
by W. Bangerth, R. Hartmann, and G. Kanschat. 
In addition, some algorithmic aspects to compute optimization problems
in which the state is given in terms of a partial
differential equations are similar to algorithms of the
Gascoigne/RoDoBo 
project, which was initiated by 
Roland Becker, Dominik Meidner,  and Boris Vexler. 
Both projects have also been 
initiated at the University of Heidelberg. 

%%%%%%%%%%%%%%%%%%%%%%%%%%%%%%%%%%%%%%%%%%%%%%%%%%
The aim of DOpE is to provide a software toolkit to solve forward PDE
problems as well as optimal control problems constrained by PDE. The
solution of a broad variety of PDE is possible in deal.II as well, but
DOpE concentrates on a unified approach for both linear and nonlinear
problems by interpreting every PDE problem as nonlinear and applying a
Newton method to solve it. While deal.II leaves much of the work and many
decisions to the user, DOpE intends to be user-optimized by delivering
prefabricated tools which require from the user only adjustments connected
to his specific problem. The solution of optimal control problems with PDE
constraints is an innovation in the DOpE framework.
The focus is on the numerical solution of both stationary and nonstationary
problems which come from different application fields, like elasticity and
plasticity, fluid dynamics, and fluid-structure interactions.

%%%%%%%%%%%%%%%%%%%%%%%%%%%%%%%%%%%%%%%%%%%%%%%%%%
At the present stage, we are able to provide the following features and 
problems:
\begin{itemize}
\item Computation of forward problems in 1d, 2d, and 3d.
\item Solution of stationary and nonstationary forward problems
\item Various timestepping schemes (based on finite differences), 
  such as forward Euler, backward Euler,
  Crank-Nicolson, shifted Crank-Nicolson, and Fractional-Step-$\Theta$ scheme.
\item Various finite elements taken from deal.ii.
\item Computation of stationary optimal control problems.
\item Solving of standard Laplace equation, Navier-Stokes, and 
fluid-structure interaction. 
\end{itemize}
 
%%%%%%%%%%%%%%%%%%%%%%%%%%%%%%%%%%%%%%%%%%%%%%%%%%
In the present manual we give a short introduction to all example programs
which can be found currently in DOpE. The organization of the examples is
as follows: we distinguish between the mere solution of a PDE and the
solution of an optimal control problem, and in each case we differentiate
further between stationary and nonstationary problems. This leads to the
following structure (we list problems treated in the examples in
brackets):
\begin{itemize}
\item
solution of PDE problem
 \begin{itemize}
   \item
     stationary PDE (Laplace equation, Stokes equations, FSI problems,
elasticity and plasticity problems)
   \item
     nonstationary PDE (Navier-Stokes equations, FSI problem, Black-Scholes
equation)
 \end{itemize}
\item
solution of PDE constrained optimal control problem
 \begin{itemize}
   \item
     stationary constraining PDE
   \item
     nonstationary constraining PDE
 \end{itemize}
\end{itemize}
The seventh stationary PDE example \ref{PDE_Stat_Laplace_2D} 
(solution of the Laplace equation) is
implemented analogously to the corresponding tutorial step in deal.II. By
comparing the respective program codes, deal.II users should be able to
catch the differences between the two software packages.
In later examples we also provide algorithms for 'goal-oriented' automatic
mesh adaption and error analysis.

{\bf Comment on deal.II}

The deal.II library is open
source and can be downloaded from \texttt{http://www.dealii.org/}. On this
homepage, one also finds lots of further information on deal.II as well as
an extensive tutorial where many features of deal.II are discussed in a
well-documented example framework. In order to use DOpE, it is highly
recommendable to be roughly acquainted with deal.II.
