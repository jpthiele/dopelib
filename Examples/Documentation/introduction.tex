\chapter{Introduction}\label{chap:intro}
\section{Developers}
The library is currently maintained by 
\begin{itemize}
  \item Christian Goll (University of Heidelberg)
  \item Thomas Wick (University of Heidelberg)
  \item Winnifried Wollner (University of Hamburg)
\end{itemize}

However, there are more highly appreciated contributions
made by (in alphabetical order)
\begin{itemize}
  \item Michael Geiger (Examples for Plasticity, and Documentation of several PDE-Examples)
\end{itemize}

\section{Software requirements}
The library DOpE has been tested to work on both Linux and 
MAC OSX. It requires a recent \texttt{ g++} (at least 4.4.3) due to 
some new \texttt{ C++} features implemented in \texttt{ C++0x} and \texttt{ C++11}. 

\subsection{deal.II}
This library is based upon \texttt{deal.II} hence in order to run 
\texttt{ DOpElib} you need a running copy of \texttt{deal.II}.

The \texttt{deal.II} library is open source and is freely available for noncommercial project.
It can be downloaded from \url{http://www.dealii.org/}. On this
homepage, one also finds lots of further information on deal.II as well as
an extensive tutorial where many features of deal.II are discussed in a
well-documented example framework. In order to use DOpE, it is highly
recommendable to be roughly acquainted with deal.II.

\subsection{SNOPT}
If you would like to use the features offered in our SNOPT wrapper. You will 
need to obtain a license for \texttt{ SNOPT} 
\url{http://www.sbsi-sol-optimize.com/asp/sol_product_snopt.htm}.
Unfortunately this is at present not available for free, but you should 
check if 
there is a department license already available.

\section{How this library is structured}
This library is designed to allow easy implementation and numerical solutions 
of problems involving partial differential equations (PDEs). The easiest case 
is that of a PDE in weak form to find some $u$
\[
a(u)(\phi) = 0 \quad \forall \phi \in V
\]
with some appropriate space $V$.
More complex cases involve optimization problems given in the form (OPT)
\begin{align*}
\min\;&J(q,u) \\
  &\text{s.t.}\; a(q,u)(\phi) = 0 \quad \forall \phi\\
  &a \le q \le b\\
  &g(q,u) \le 0  
\end{align*}
where $u$ is a FE-function and $q$ can either be a FE-function or some 
fixed number of parameters.

\subsection{Problem description}
In order to allow our algorithms the automatic assembly of all required 
data we need to have some container which contains the complete problem 
description in a common data format. For this we have the following 
classes in \texttt{ DOpEsrc/container}
\begin{itemize}
  \item \texttt{ pdeproblemcontainer.h} Is used to describe  stationary PDE problems.
  \item \texttt{ instatpdeproblemcontainer.h} {\bf TODO is still missing but will be used for nonstationary problems.} This will be implemented once we have nonstationary optimization problems running to avoid error duplication in the coding process.
  \item \texttt{ optproblem.h} Is used to describe  OPT problems governed by 
    stationary PDEs. {\bf TODO should be renamed to optproblemcontainer.h including the class name!}
  \item \texttt{ instatoptproblemcontainer.h} Is used to describe  OPT problems governed by nonstationary PDEs. The only difference to the stationary case is that we need to specify a time-stepping method.  
\end{itemize}
In order to fill these containers there are two things to be done,
first we need to actually write some data like $a$, $J$, $\ldots$
which describe the problem. Then we have to select some numerical 
algorithm components like finite elements, linear solvers $\ldots$.
The latter ones should be written such that when exchanging these components
none of the problem descriptions should require changes. 
Note that it still may be necessary to write some additional descriptions, 
e.g., if you solve the PDE with a fix point iteration you don't need derivatives
but if you want to use Newton's method you then have to provide some.

We will start by discussing the problem description components implemented so far

\subsubsection{Problem specific classes}
The following classes are used to describe the problem and will usually require 
some implementation.

\begin{itemize}
  \item \texttt{ basic/constraints.h} This is used by the spacetimehandlers to 
    compute the number of constraints from the control and state vectors. 
    It must not be reimplemented by the user, but needs to be properly 
    initialized if OPT is used with box control constraints or $g(q,u) \le 0$.
  \item \texttt{ interfaces/functionalinterface.h} This gives an interface 
    for the functional $J$ and any other functional you may want to evaluate.
    In general this can be used as a base class to write your own functionals 
    in examples. We note that we only need to write the integrands on 
    elements or faces the loop over elements will be taken care of else where.
  \item \texttt{ interfaces/constraintinterface.h} This gives an interface for both 
    the control box constraints as well as the general constraint $g \le 0$. This 
    needs to be specified if constraints are to be used. If they are not needed 
    a default class \texttt{ problemdata/noconstraints.h} can be used. We note that we only 
    need to write the integrands on 
    elements or faces the loop over elements will be taken care of else where.
  \item \texttt{ interfaces/pdeinterface.h} This defines an interface for the 
    partial differential equation $a(q,u)(\phi) = 0$. This needs to be written
    by the user. We note that we only need to write the integrands on 
    elements or faces the loop over elements will be taken care of else where.
  \item \texttt{ interfaces/dirichletdatainterface.h} This gives an interface to the 
    Dirichlet data for a problem. If the Dirichlet data are simply a function 
    (and do not depend on the control $q$ one can use the default class
    \texttt{ problemdata/simpledirichletdata.h}.
\end{itemize}

\subsubsection{Numeric components}
These are the components from which a user needs to select some in order to actually 
solve the given problem. They will not require any rewriting, but sometimes it is 
advisable to write other than the default parameter into the param file for the 
solution.

First we need to select a method how to handle all dofs in space and time.
\begin{itemize}
\item \texttt{ basic/spacetimehandler\underline{ }base.h} This class is used to define 
  an interface to the dimension independent functionality of all space time dof handlers. 
\item \texttt{ basic/statespacetimehandler.h} Another intermediate interface class which adds 
  the dimension dependent functionality if only the variable $u$ is considered, i.e., a 
  PDE problem.
\item \texttt{ basic/spacetimehandler.h } Same as above but with both $q$ and $u$, i.e., for
  OPT problems.
\item \texttt{ basic/mol\underline{ }statespacetimehandler.h} Implementation of a method of 
  line space time dof handler for PDE problems. It has only one spatial 
  dofhandler which is used for all time intervals.
\item \texttt{ basic/mol\underline{ }spacetimehandler.h} Same as above for OPT problems.
  A separate spatial dof handler for each of the variables $q$ and $u$ is maintained 
  but only one triangulation.
\item \texttt{ basic/mol\underline{ }multimesh\underline{ }spacetimehandler.h}
  Same as above, but now in addition the triangulations for $q$ and $u$ can be refined
  separately from one common initial coarse triangulation. Note that this will
  in addition require the use of the multimesh version for integrator and 
  face- as well as celldatacontainer.
\end{itemize}
Note that we use these for stationary problems as well, but then you don't have to specify
any time information.

Second you will need to specify some container classes to be used to 
pass data between objects. At present you don't have much choice, but you may wish 
to reimplement some of these if you need data that is not currently included in 
the containers.
\begin{itemize}
\item \texttt{ container/celldatacontainer.h} This object is used to pass data 
  given on the current element (cell) of the mesh to the functions in PDE, functional, 
  $\ldots$. 
\item \texttt{ container/facedatacontainer.h} This object is used to pass data 
  given on the current face of the mesh to the functions in PDE, functional, 
  $\ldots$. 
\item \texttt{ container/multimesh\underline{ }celldatacontainer.h} This is the same as the 
  celldatacontainer, but it
  is capable to handle data defined on an alternative triangulation.
\item \texttt{ container/multimesh\underline{ }facedatacontainer.h} This is the same as the
  facedatacontainer, but it
  is capable to handle data defined on an alternative triangulation.
\item \texttt{ container/integratordatacontainer.h} This contains some data that 
  should be passed to the integrator like quadrature formulas and the above cell and 
  face data container.
\end{itemize}

Third, at least for nonstationary PDEs we need to select a time stepping scheme
the file names of which are mostly self explanatory:
\begin{itemize}
\item \texttt{ include/forward\underline{ }euler\underline{ }problem.h}
\item \texttt{ include/shifted\underline{ }crank\underline{ }nicolson\underline{ }problem.h}
\item \texttt{ include/backward\underline{ }euler\underline{ }problem.h}
\item \texttt{ include/fractional\underline{ }step\underline{ }theta\underline{ }problem.h} Note that the use of this scheme requires a special Newton solver!
\item \texttt{ include/crank\underline{ }nicolson\underline{ }problem.h}
\end{itemize}

Finally, we need to select a way how to integrate and solve linear and nonlinear equations
\begin{itemize}
\item \texttt{ templates/integrator.h} This class computes integrals over a given 
  triangulation (including its faces).
\item \texttt{ templates/integrator\underline{ }multimesh.h} The same as above but it is 
  possible that some of the FE functions are defined on an other triangulation 
  as long as the have a common coarse triangulation.
\item \texttt{ templates/integratormixeddims.h} This is used to compute integral which 
  are given in another (larger) dimension than the current variable. (This is exclusively
  used if the control variable is given by some parameters. Which means \texttt{ dopedim == 0}). 
\item \texttt{ templates/cglinearsolver.h} This is a wrapper for the cg solver implemented in 
  \texttt{deal.II}. The solver will build and store the stiffness matrix for the PDE.
\item \texttt{ templates/gmreslinearsolver.h} This is a wrapper for the gmres solver 
  implemented in \texttt{deal.II}. The solver will build and store the stiffness matrix 
  for the PDE.
\item \texttt{ templates/directlinearsolver.h} This is a wrapper for the direct solver 
  implemented in \texttt{deal.II} using \texttt{ UMFPACK}. 
  The solver will build and store the stiffness matrix for the PDE.
\item \texttt{ templates/voidlinearsolver.h} This is a wrapper for certain cases when we 
  know that the matrix to be inverted is the identity. It simply copies the rhs to the
  lhs. This is only needed for compatibility reasons some other components.
\item \texttt{ templates/newtonsolver.h} This solves some nonlinear equation using a 
  line-search Newton method.
\item \texttt{ templates/newtonsolvermixeddims.h} The same but in the case when there is 
  another variable in a (larger) dimension is involved. See 
  \texttt{ integratormixeddims.h}.
\item \texttt{ templates/instat\underline{ }step\underline{ }newtonsolver.h} This is a 
  Newton method as above to invert the next time-step. It differs from the plain vanilla
  version in that it computes certain data from the previous time step only once 
  and not in every Newton iteration.
\item \texttt{ templates/fractional\underline{ }step\underline{ }theta\underline{ }step\underline{ }newtonsolver.h} This is the Newton solver for the time step in a 
  fractional-step-theta scheme. It combines the computation of all three sub steps.
\end{itemize}

\subsection{Reduced problems (Solve the PDE)}
At times it is nice to remove the PDE constraint. This is handled by so called reduced 
problems. This means that the reduced problem implicitly solves the PDE whenever required
and eliminates the variable $u$ from the problem.
\begin{itemize}
\item \texttt{ reducedproblems/statpdeproblem.h} This is used to remove the variable $u$ in 
  a stationary PDE problem. This means that call the method \\
  \texttt{ StatPDEProblem::ComputeReducedFunctionals} will evaluate the functionals 
  defined in the problem description, i.e., in \texttt{ PDEProblemContainer}, in the 
  solution of the given PDE.
\item \texttt{ reducedproblems/statreducedproblem.h} This eliminates $u$ from the OPT
  problem with a stationary PDE.
\item \texttt{ reducedproblems/instatreducedproblem.h} The same as above but for a
  nonstationary PDE. {\bf FIXME there is something wrong in this file see FIXME 
    comment in the source.}
\item \texttt{ reducedproblems/voidreducedproblem.h} A wrapper file that eliminates $u$ 
  if it is not present anyways. This is used so that we can use the same routines to 
  solve problems that have no PDE constraint.
\end{itemize}

\subsection{Optimization algorithms}
Now, in order to solve optimization algorithms we need to define some algorithms.
At present we offer a selection of algorithms that solve the reduced optimization 
problem where the PDE constraint has been eliminated as explained in the previous section
\begin{itemize}
\item \texttt{ opt\underline{ }algorithms/reducedalgorithm.h} An interface for all 
  optimization problems in the reduced formulation. It offers some test functionality
  to assert that the derivatives of the problem are computed correctly.
\item \texttt{ opt\underline{ }algorithms/reducednewtonalgorithm.h}
  A line-search Newton algorithm using a cg method to invert the reduced hessian. 
  Implementation ignores any additional constraints.
\item \texttt{ opt\underline{ }algorithms/reducedtrustregionnewton.h}
  A trust region Newton algorithm using a cg method to invert the reduced hessian.
  Implementation ignores any additional constraints.
\item \texttt{ opt\underline{ }algorithms/reduced\underline{ }snopt\underline{ }algorithm.h}
  An algorithm to solve reduced optimization problems with additional control constraints.
  ((reduced) state constraints are not yet implemented.)
\item \texttt{ opt\underline{ }algorithms/reducednewtonalgorithmwithinverse.h}
  Line-search Newton algorithm that assumes there exists a method in the reduced problem
  that can invert the reduced hessian. (This usually makes sense only if there is no 
  PDE constraint.)
\item \texttt{ opt\underline{ }algorithms/generalized\underline{ }mma\underline{ }algorithm.h}
  An implementation of the MMA-Algorithm for structural optimization using an augmented
  Lagrangian formulation for the subproblems. The subproblem is implemented using the 
  special purpose file
  \texttt{ include/augmentedlagrangianproblem.h}.
\end{itemize} 

\subsection{Other Components}
Beyond these clearly structured groups before there are some classes remaining that
do not fit the above but are important for the user to know:
\begin{itemize}
\item \texttt{ include/statevector.h} This stores all dofs in space and time for the state 
  variable $u$. It is possible to select whether all this should be kept in memory or 
  or unused parts can be written to the hard disk.
\item \texttt{ include/controlvector.h} This stores all dofs in space and time for the 
  control variable $q$. At present no time dependence is implemented.
\item \texttt{ include/constraintvector.h} This stores all dofs in space and time for the 
  non PDE constraints (and corresponding multipliers). 
  At present no time dependence is implemented.
\item \texttt{ include/parameterreader.h} This file is used to define a parameter reader
  that is used to read run time parameters from a given file.
\item \texttt{ include/dopeexception.h} Defines some Exceptions that are thrown by the program
  should it encounter any unexpected errors.
\item \texttt{ include/dopeexceptionhandler.h} This class is used to write information 
  contained in the exceptions to the output in a uniform manner.
\item \texttt{ include/outputhandler.h} This file defines an outputhandler object which 
  can be used to decide whether some information should be written to screen or file.
  In addition it can format output according to some run time parameters given by a 
  parameter file.
\item \texttt{include/constraintsmaker.h} This class sets the constraints on the DOFs of the state FE solution. In the standard implementation, only the hanging-node-constraints are built, but the user can reimplement this class and then include other constraints as well (for example periodic BC).
\item \texttt{include/sparsitymaker.h} This class sets the sparsity pattern for the state FE solution. The standard implementiation is just a wrapper for \texttt{dealii::DoFTools::} \texttt{make\_sparsity\_pattern}, but the user can reimplement this class to allow for more sophisticated sparsity patterns.
\item \texttt{interfaces/active\underline{ }fe\underline{ }index\underline{ }setter\underline{ }interface.h} In the case of hp finite elements, one has to specify for each cell which finite element to use. This is done via this interface.
\end{itemize}

\marginpar{Write the rest! WW}
\subsection{Internal structures}
interfaces:
transposeddirichletdatainterface.h
reducedprobleminterface.h
            pdeprobleminterface.h  

problemdata:
noconstraints.h        stateproblem.h          transposedgradientdirichletdata.h
primaldirichletdata.h  tangentdirichletdata.h  transposedhessiandirichletdata.h
simpledirichletdata.h
\begin{itemize}
\item \texttt{include/timedofhandler.h} DoFHandler responsible for the management of the timedofs (this is a part of the \texttt{SpaceTimeDoFHandler}-classes). Basically a wrapper for a $1d$ \texttt{deal.II}-DoFHandler.
\item \texttt{include/timeiterator.h} This class works as an iterator on the \texttt{TimeDoFHandler}.                 
\end{itemize}
\subsection{Wrapper classes}

wrapper:
dofhandler\underline{ }wrapper.h  finiteelement\underline{ }wrapper.h  preconditioner\underline{ }wrapper.h
fevalues\underline{ }wrapper.h    function\underline{ }wrapper.h       snopt\underline{ }wrapper.h

{\bf TODO! The following files need some appropriate description here 
This should be done by someone who knows what they are used for. Please move the description to the right place above.}
\marginpar{TODO}

basic:
      sth\underline{ }internals.h                     
      DOpEtypes.h 
include:   
solutionextractor.h                 
            helloworld.h                               
            helper.h                                          

reducedproblems:
 forwardtimestepreducedproblem.h
  


