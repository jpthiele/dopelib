\chapter{Terminal Output  and Parameter Files when Running the Examples}
\label{sec_terminal_output}
We have implemented various options for terminal output and 
data output into files that can be controlled with the help of 
the respective parameter files
\begin{verbatim}
dope.prm
\end{verbatim}
in each Examples folder.

In the following we briefly show such output and briefly 
explain where to control it.
\section{Example of a terminal output: \texttt{PDE/InstatPDE/Example8}}

\begin{verbatim}
Computing State Solution:
Computing Initial Values:
                 Newton step: 0  Residual (abs.):   9.7656e-04
                 Newton step: 0  Residual (rel.):     1.0000e+00
                 Newton step: 1  Residual (rel.): < 1.0000e-11   LineSearch {0} 
Writing [Results/Mesh0/State_InstatPDEProblemContainer.00000.vtk]
                 Precalculating functional values 
        StressX: 0
         Timestep: 1 (0 -> 0.0001) using Crank-Nicolson
                 Newton step: 0  Residual (abs.):   4.0385e-04
                 Newton step: 0  Residual (rel.):     1.0000e+00
                 Newton step: 1  Residual (rel.):   1.8656e-03   LineSearch {0} 
                 Newton step: 2  Residual (rel.):   3.4959e-06   LineSearch {0} 
                 Newton step: 3  Residual (rel.):   1.3608e-10   LineSearch {0} 
Writing [Results/Mesh0/State_InstatPDEProblemContainer.00001.vtk]
                 Precalculating functional values 
        StressX: 6.42408
\end{verbatim}

We observe the respective Newtons step, its current (relative) residual. 
Then, we see at which place the result is written and the respective 
name of that file. Afterwards, we calculate the functional values, here the 
\texttt{StressX} value (see the detailed description of that example before).
Then, we increment the time step number and proceed further.


\section{Controlling the output in the parameter file}
We have implemented various options to control the output via the 
parameter file. Specifically in the part \texttt{subsection output
  parameters}.
Taking the parameter file from the previous example it looks like:
\begin{verbatim}
subsection output parameters
  # File format for the output of solution variables
  set file_format       = .vtk

  # Iteration Counters that should not reflect in the outputname, seperated by
  # `;`
  set ignore_iterations = PDENewton;Cg
  
  # Name of the logfile
  set logfile           = dope.log

  # Do not write files whose name contains a substring given here by a list of
  # `;` separated words
  set never_write_list  = Gradient;Residual;Hessian;Tangent;Adjoint;Update
  #set never_write_list  = Gradient;Hessian;Tangent;Adjoint
      
  # Defines what strings should be printed, the higher the number the more
  # output
  set printlevel        = 6
  
  # Set the precision of the newton output
  set number_precision	 = 4

  # Set manually the machine tolarance for the output
  set eps_machine_set_by_user	 = 1.0e-11


  # Directory where the output goes to
  set results_dir       = Results/
end
\end{verbatim}
Most settings are self-explaining with the help of their respective comments
provided here. In the following, we discuss some of them in more detail though.



\newpage
\subsection{Printlevel}
Let us first look to 
\begin{verbatim}
 set printlevel        = 6
\end{verbatim}
Taking $5$ rather than $6$ yields the terminal output:
\begin{verbatim}
Computing State Solution:
Computing Initial Values:
Writing [Results/Mesh0/State_InstatPDEProblemContainer.00000.vtk]
         Timestep: 1 (0 -> 0.0001) using Crank-Nicolson
Writing [Results/Mesh0/State_InstatPDEProblemContainer.00001.vtk]
\end{verbatim}
Reducing further the number yields less and less output on the screen. 
For debugging purposes in the developement stage, it might be helpful to 
work with print level $6$. 

\begin{remark}
In the \texttt{test.prm} file for running the regression tests, the print
level 
should be sufficiently high in order to compare `something'. If nothing 
is printed, nothing can be compared whether \dope{} still works correctly.
\end{remark}

\begin{remark}
Also
\begin{verbatim}
 set printlevel        = -1
\end{verbatim}
has been implemented, which just prints all possible output on the terminal
thus deactivating all output filters.
\end{remark}


\subsection{Set never write list}
With respect to graphical file output (here *.vtk), specifically in
optimization many steps in the algorithm may be useful when checking 
new developements. For instance, in the previous list, we never write 
\begin{verbatim}
 set never_write_list  = Gradient;Residual;Hessian;Tangent;Adjoint;Update
\end{verbatim}
On the other hand, if \texttt{set never write list} is empty, 
we would write for each vector the respective *.vtk file
which requires for fine discretizations a lot of memory on the hard disk.

Let us give an example: we set 
\begin{verbatim}
 set never_write_list  = Gradient;Residual;Hessian;Tangent;Adjoint;Update
\end{verbatim}
Then, only the physical solution is written as *.vtk:
\begin{verbatim}
Writing [Results/Mesh0/State_InstatPDEProblemContainer.00000.vtk]
\end{verbatim}
If we remove \texttt{Residual} from that list, we obtain:
\begin{verbatim}
 set never_write_list  = Gradient;Hessian;Tangent;Adjoint;Update
\end{verbatim}
and we write the residual of the solution after each Newton step (using
\texttt{printlevel 6} here):
\begin{verbatim}
         Timestep: 1 (0 -> 0.0001) using Crank-Nicolson
Writing [Results/Mesh0/Residualstate.00001.vtk]
                 Newton step: 0  Residual (abs.):   4.0385e-04
                 Newton step: 0  Residual (rel.):     1.0000e+00
Writing [Results/Mesh0/Residualstate.00001.vtk]
                 Newton step: 1  Residual (rel.):   1.8656e-03   LineSearch {0} 
Writing [Results/Mesh0/Residualstate.00001.vtk]
                 Newton step: 2  Residual (rel.):   3.4959e-06   LineSearch {0} 
Writing [Results/Mesh0/Residualstate.00001.vtk]
                 Newton step: 3  Residual (rel.):   1.3608e-10   LineSearch {0} 
Writing [Results/Mesh0/State_InstatPDEProblemContainer.00001.vtk]
                 Precalculating functional values 
        StressX: 6.42408
\end{verbatim}

\begin{remark}
By removing \texttt{Update}, we would also write each Newton update into a
vtk-file.
These options allow detailed debugging of the code.
\end{remark}


\subsection{dope.log}
The \texttt{dope.log} file contains exactly the output of the terminal 
during and after running an example. This allows to double-check the output 
when necessary.


\subsection{Functional output into \texttt{gnuplot}}
At the end of a simulation (in particular a time-dependent problem), we 
are often interested in the temporal evolution of the functionals of interest.
Here, in Example 8 it is \texttt{StressX}. If too much output would appear,
\dope{} passes it directly into a file:
\begin{verbatim}
Computing Functionals:
StressX too large. Writing to file instead: 
Writing [Results/Mesh0/StressX_InstatPDEProblemContainer.00140.gpl]
\end{verbatim}
This file can then be plotted using gnuplot.
The behavior for time-dependent problems is indeed usually to write 
all functional values into a file at the end of a computation.
For stationary problems, the functional value will be printed in the terminal.
