\chapter{Example Handling, Creating new Examples}
\label{chap:howtoex}
\section{Getting started}
Beside the fact that \dope{} is still under development,
it offers already various different (linear and nonlinear) 
examples for a lots 
of different applications in two and three dimensions; 
we refer the reader to the 
next two Chapters \ref{PDE} and \ref{OPT}. 

To implement new examples or to use existing examples 
from the library for own research, the user 
can simply copy an existing example. In this 
new example, own code and changes can be compiled. Here is some 
advice to get started:
\begin{itemize}
\item If you are a first time user of \dope{} 
with some numerics background, you 
might be familiar with the Poisson (or more general Laplace) equation.
\dope{} has it too. Check-out Example \ref{PDE_Stat_Laplace_2D},
to see how \dope{} implements this well-known equation in two
dimensions or \ref{PDE_Stat_Laplace_3D} for its three-dimensional version.
\item Before you implement a new example, please check which 
existing example might be similar to your goals and get familiar 
to it. Then, proceed as described in Section \ref{getting_started}. 
\end{itemize}
%\begin{remark}[Why two examples for the Laplace equation?]
%\dope{} offers dimensional-independent programming! So, we could have 
%implemented the 2d and 3d case in one file. The reason why we
%present
%to separate programs \ref{PDE_Stat_Laplace_2D} and \ref{PDE_Stat_Laplace_3D}
%is, that in the 3d case, we show additional 
%capabilities of the library itself; specifically how to use different
%iterative linear solvers.
%\end{remark}

%%%%%%%%%%%%%%%%%%%%%%%%%%%%%%%%%%%%%%%%%%%%%%%%%%%%%%%%%%%%%%%%%%%%%%%%%%%%%%%
\section{How to run existing examples}
\subsection{The global way}
The easiest way is to first build all examples. Go into 
\begin{lstlisting}
dopelib/Examples/
\end{lstlisting}
Herein, type 
\begin{lstlisting}
make c-all
\end{lstlisting}
By typing only `make' you will see all options 
(we also refer to Section \ref{sec_installation}).
This procedure will take some minutes. Afterwards,
go into the examples folder of your choice. For instance:
\begin{lstlisting}
dopelib/Examples/PDE/InstatPDE/Example1/autobuild
\end{lstlisting}
Herein you create the executable via
\begin{lstlisting}
make 
\end{lstlisting}
and then go back via ../ and 
\begin{lstlisting}
./DOpE-PDE-InstatPDE-Example1 
\end{lstlisting}
Or both commands together:
\begin{lstlisting}
autobuild> make && cd .. &&  ./DOpE-PDE-InstatPDE-Example1 
\end{lstlisting}
You will find the results on the screen in the terminal as well
as some graphical output in the `Results' folder.

\subsection{Building, making and running in the local folder}
In case we have run first globally as previsously described, 
in each examples folder we find an autobuild subfolder. 
If we now want to work locally (which is the usual way) then 
we have to type `make' in the autobuild folder:
\begin{lstlisting}
autobuild> make
\end{lstlisting}
In case we need to modify the Makefile, we need first to do:
\begin{lstlisting}
autobuild> cmake ..
\end{lstlisting}
For instance one reason to modify the Makefile could be 
to change to debug mode as described in Section \ref{sec_release_debug}.
Finally, go one folder back and run the object file.

In case you work and test out new things (modifiying the 
main.cc and localpde.h files etc.), one command for instance 
in Example 9 is:
\begin{lstlisting}
Example9> cd autobuild/ && make && cd .. &&  ./DOpE-PDE-StatPDE-Example9
\end{lstlisting}





%%%%%%%%%%%%%%%%%%%%%%%%%%%%%%%%%%%%%%%%%%%%%%%%%%%%%%%%%%%%%%%%%%%%%%%%%%%%%%%
\section{Changing from Release mode to Debug mode}
\label{sec_release_debug}
In a specific example go into the autobuild folder 
and change therein:
\begin{lstlisting}
cmake CMAKE_BUILD_TYPE=Release ..
\end{lstlisting}
to some other behavior:
\begin{lstlisting}
Debug Release RelWithDebInfo MinSizeRel
\end{lstlisting}
Then type `make' to build the executable file and go back 
into the parent folder and execute the object file.



%%%%%%%%%%%%%%%%%%%%%%%%%%%%%%%%%%%%%%%%%%%%%%%%%%%%%%%%%%%%%%%%%%%%%%%%%%%%%%%
\section{Creating new examples}
\label{getting_started}
Before being able to change and compile the new code, the user must 
follow some easy steps in order to modify the information related to the old code. In 
this section we explain how to modify such information using as model 
\texttt{PDE/StatPDE/Example1}.

\begin{enumerate}
\item \textbf{New: in git from March 2017} In a first step, we copy \texttt{Example1} and renamed it, e.g., 
\texttt{MyWonderfulFirstExample}. 
After having reached the folder of the example in question in the terminal, 
\texttt{PDE/StatPDE} in our case, we perform these operation writing the following:
\begin{verbatim}
cp -r Example1 MyWonderfulFirstExample
cd MyWonderfulFirstExample
\end{verbatim}


 \item \textbf{Old: in the svn up to version 8.3} In a first step, we copy \texttt{Example1} and renamed it, e.g., 
\texttt{MyWonderfulFirstExample}. At the same time it is important to remove the repository 
information that it is stored in the directory \texttt{.svn/}. \\
After having reached the folder of the example in question in the terminal, 
\texttt{PDE/StatPDE} in our case, we perform these operation writing the following:
\begin{verbatim}
cp -r Example1 MyWonderfulFirstExample
cd MyWonderfulFirstExample
rm -rf .svn
cd Test
rm -rf .svn
\end{verbatim}
Please note that removing the \texttt{.svn} sub-directories is important,
as otherwise your files may be replaced or changed during your next
update. Also, if you can submit information to the subversion repository 
you might accidentally overwrite the original example, here \texttt{Example1}.

\item You will find a file \texttt{CMakeLists.txt} in the folder 
\texttt{MyWonderfulFirstExample}. Open this file, in it you can 
find the line 
\begin{verbatim}
  SET(TARGET "DOpE-PDE-StatPDE-Example1")
\end{verbatim}
the string \texttt{DOpE-PDE-StatPDE-Example1} will be the name of the
executable of your program. Change it to something suitable, e.g.,
\begin{verbatim}
  SET(TARGET "MyWonderfulFirstExample")
\end{verbatim}
Moreover, this file contains the lines 
\begin{verbatim}
SET(dope_dimension 2)
SET(deal_dimension 2)
\end{verbatim}
which define the dimension of the domain for the control-variable
(\texttt{dope\_dimension})
and the PDE solution (\texttt{deal\_dimension}). If for your example 
one of these differs from $2$ adjust the number accordingly.
This file will not need any further modifications.

\item  Now, you are prepared to change any of the problem
  dependent data in information in the files 
\begin{verbatim}
main.cc, localpde.h, functionals.h, localfunctional.h, etc
\end{verbatim} 
If above you have changed the dimensions, make sure to adjust all
files accordingly!

\item The cmake system can - in principle be used
  ''in-source'' to create the executable. 
  However, this may break some of the automated tests (later),
  so you are encouraged to proceed differently. To do this end
  proceed as follows in the directory \texttt{MyWonderfulFirstExample}:
  \begin{verbatim}
MyWonderfulFirstExample$ mkdir build
MyWonderfulFirstExample$ cd build
MyWonderfulFirstExample/build$ cmake -DCMAKE_BUILD_TYPE=Release ../
\end{verbatim}
which will configure the build (if you want to debug it is useful to
replace the string \texttt{Release} with \texttt{Debug}). 
Once this is done, you can compile and run the code:
\begin{verbatim}
MyWonderfulFirstExample/build$ make 
MyWonderfulFirstExample/build$ cd ..
MyWonderfulFirstExample$ ./MyWonderfulFirstExample
\end{verbatim}
(Assuming that no errors occurred during the make call)

\item The \texttt{Makefile} in the directory is present only to
  preserve backward compatibility. If you wish to use the automated
  build/test routines in DOpE lib, you need to make sure that it its
  configured correctly. If you just want to use the cmake capabilities 
  you can safely ignore this passage.

  Furthermore, if you have followed the instructions above, then no
  changes will be needed.

  If you have moved the directory to some other place, i.e., it is not
  in the same folder as Example1, then open the \texttt{Makefile} 
  in the directory \texttt{MyWonderfulFirstExample}
  you will find the line 
  \begin{verbatim}
  DOpE = ../../../../
  \end{verbatim}
  which points to the root directory of your DOpE installation.
  Adjust the path to match accordingly.

  When working with cmake, the easiest option is to set 
  the DOpElib environment variable, so that it will be found 
  be cmake (similar to Section \ref{sec_installation}):
  \begin{verbatim}
  export DOPE_DIR=path_to_DOpElib
  \end{verbatim}
  or alternatively, passing directly a flag to cmake:
  \begin{verbatim}
  -DDOPE_DIR=path_to_DOpElib
  \end{verbatim}

\item If you want to run automated tests on you program so that you can 
  verify whether your code is running as expected after updating the 
  library you may want to update the sub-directory \texttt{Test} 
  as well, see also Chapter~\ref{chap:test}. Otherwise you may skip this 
  step.

  Change to the \texttt{Test} sub-directory. And then modify the
  test-script to contain the new name of the executable.
  Assuming you want to use Emacs, open the file \texttt{test.sh}
\begin{verbatim}
PDE/StatPDE/Example1/Test> emacs test.sh
\end{verbatim}
  where, in our example you find the line
\begin{verbatim}
PROGRAM=../DOpE-PDE-StatPDE-Example1-2d-2d
\end{verbatim} 
  if you made a copy of an other example the part \texttt{DOpE-PDE-StatPDE-Example1-2d-2d}
  may differ. These lines need to be replaced with the new name of the 
  executable, i.e., for our given example
\begin{verbatim}
PROGRAM=../MyWonderfulFirstExample
\end{verbatim} 
    
\item Once you have finished and are sure that your example is running correctly
  and you want to use the automated test scripts --see 3) above-- You need 
  to store new test information to account for your changes. 
  
  To do so, change to the \texttt{Test} sub-directory and run the test:
\begin{verbatim}
./test.sh Test
\end{verbatim}
  Note that this should fail, otherwise you have not changed anything in the program, 
  or forgot part 3) of this description.
  
  If it failed have a look into the file \texttt{dope.log} and see whether you like the 
  output. If you do not like it you may wish to update the file \texttt{test.prm} that 
  takes care of the parameters for the test run.
  
  Once your satisfied with what you see in the log-file \texttt{dope.log} you need to store 
  that information using
\begin{verbatim}
./test.sh Store
\end{verbatim}
\end{enumerate}


