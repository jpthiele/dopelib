\documentclass[a4paper,cleardoubleempty]{scrreprt}
\usepackage{amsmath,amssymb}
\usepackage[pdftex,bookmarks=false,pdfpagelabels=true,plainpages=false]{hyperref}
\usepackage{booktabs}
\usepackage{makeidx}
%\usepackage{sfmath}
\newtheorem{Problem}{Problem}[section]

\newcommand{\todo}[1]{\textbf{\textsc{\textcolor{black}{TODO: #1}}}}


\makeindex

%%%%%%%%%%%%%%%%%%%%%%%%%%%%%%%%%%%%%%%%%%%%%%%%%%
\begin{document}
\title{DOpE - Deal Optimization Environment}
\author{
  Michael Geiger\thanks{University of Heidelberg, 
    {michael.geiger@iwr.uni-heidelberg.de}}, 
  Christian Goll\thanks{University of Heidelberg, 
    {christian.goll@iwr.uni-heidelberg.de}}, 
  Thomas Wick\thanks{University of Heidelberg, 
    {thomas.wick@iwr.uni-heidelberg.de}}, 
  Winnifried Wollner\thanks{University of Hamburg, 
    {winnifried.wollner@math.uni-hamburg.de}}
}



\date{\today}

%%%%%%%%%%%%%%%%%%%%%%%%%%%%%%%%%%%%%%%%%%%%%%%%%%
\maketitle
\cleardoublepage\pagestyle{headings}
\tableofcontents
\cleardoublepage

%%%%%%%%%%%%%%%%%%%%%%%%%%%%%%%%%%%%%%%%%%%%%%%%%%
\chapter{Introduction}
In this report, we describe the DOpE
\textit{Deal Optimization Environment} project. 
Originally, the project was initiated in the year 2009 at the 
University of Heidelberg (Germany) in the numerical analysis
group of Rolf Rannacher. 

%%%%%%%%%%%%%%%%%%%%%%%%%%%%%%%%%%%%%%%%%%%%%%%%%%
The \textit{Deal Optimization Environment} (DOpE) project is mainly 
based on the
\textit{deal.II} finite element library which has been developed initially
by W. Bangerth, R. Hartmann, and G. Kanschat. 
In addition, some algorithmic aspects to compute optimization problems
in which the state is given in terms of a partial
differential equations are similar to algorithms of the
Gascoigne/RoDoBo 
project, which was initiated by 
Roland Becker, Dominik Meidner,  and Boris Vexler. 
Both projects have also been 
initiated at the University of Heidelberg. 

%%%%%%%%%%%%%%%%%%%%%%%%%%%%%%%%%%%%%%%%%%%%%%%%%%
The aim of DOpE is to provide a software toolkit to solve forward PDE
problems as well as optimal control problems constrained by PDE. The
solution of a broad variety of PDE is possible in deal.II as well, but
DOpE concentrates on a unified approach for both linear and nonlinear
problems by interpreting every PDE problem as nonlinear and applying a
Newton method to solve it. While deal.II leaves much of the work and many
decisions to the user, DOpE intends to be user-optimized by delivering
prefabricated tools which require from the user only adjustments connected
to his specific problem. The solution of optimal control problems with PDE
constraints is an innovation in the DOpE framework.
The focus is on the numerical solution of both stationary and nonstationary
problems which come from different application fields, like elasticity and
plasticity, fluid dynamics, and fluid-structure interactions.

%%%%%%%%%%%%%%%%%%%%%%%%%%%%%%%%%%%%%%%%%%%%%%%%%%
At the present stage, we are able to provide the following features and 
problems:
\begin{itemize}
\item Computation of forward problems in 1d, 2d, and 3d.
\item Solution of stationary and nonstationary forward problems
\item Various timestepping schemes (based on finite differences), 
  such as forward Euler, backward Euler,
  Crank-Nicolson, shifted Crank-Nicolson, and Fractional-Step-$\Theta$ scheme.
\item Various finite elements taken from deal.ii.
\item Computation of stationary optimal control problems.
\item Solving of standard Laplace equation, Navier-Stokes, and 
fluid-structure interaction. 
\end{itemize}
 
%%%%%%%%%%%%%%%%%%%%%%%%%%%%%%%%%%%%%%%%%%%%%%%%%%
In the present manual we give a short introduction to all example programs
which can be found currently in DOpE. The organization of the examples is
as follows: we distinguish between the mere solution of a PDE and the
solution of an optimal control problem, and in each case we differentiate
further between stationary and nonstationary problems. This leads to the
following structure (we list problems treated in the examples in
brackets):
\begin{itemize}
\item
solution of PDE problem
 \begin{itemize}
   \item
     stationary PDE (Laplace equation, Stokes equations, FSI problems,
elasticity and plasticity problems)
   \item
     nonstationary PDE (Navier-Stokes equations, FSI problem, Black-Scholes
equation)
 \end{itemize}
\item
solution of PDE constrained optimal control problem
 \begin{itemize}
   \item
     stationary constraining PDE
   \item
     nonstationary constraining PDE
 \end{itemize}
\end{itemize}
The seventh stationary PDE example \ref{PDE_Stat_Laplace_2D} 
(solution of the Laplace equation) is
implemented analogously to the corresponding tutorial step in deal.II. By
comparing the respective program codes, deal.II users should be able to
catch the differences between the two software packages.
In later examples we also provide algorithms for 'goal-oriented' automatic
mesh adaption and error analysis.

{\bf Comment on deal.II}

The deal.II library is open
source and can be downloaded from \texttt{http://www.dealii.org/}. On this
homepage, one also finds lots of further information on deal.II as well as
an extensive tutorial where many features of deal.II are discussed in a
well-documented example framework. In order to use DOpE, it is highly
recommendable to be roughly acquainted with deal.II.



\cleardoublepage

%%%%%%%%%%%%%%%%%%%%%%%%%%%%%%%%%%%%%%%%%%%%%%%%%%
\chapter{Example Handling, Creating new Examples}
To implement new examples or to use existing examples 
from the library for own research, the user 
can simply copy an existing example. In this 
new example, own code and changes can be compiled.

Adding a new regression test to the repository 
needs some concentration of the user. In the 
following, we explain how to do this and what 
\textit{must} be taken into account:
\begin{verbatim}
cp -r Example1 ExampleNew
cd ExampleNew
rm -r CVS
cd Test
rm -r CVS
\end{verbatim}
It is important to remove the repository information,
which is stored in the directory CVS.

Then, we have to consider the following step. In order to 
run the regression tests, we start the executable file
of the present example. This executable comes from the 
Makefile. For example in the Makefile of 
Example \ref{PDE_Stat_Stokes}, 
\begin{verbatim}
target   = $(BINDIR)/DOpE-PDE-StatPDE-Example1-$
           (dope_dimension)d-$(deal_II_dimension)d
\end{verbatim}
and concretely:
\begin{verbatim}
DOpE-PDE-StatPDE-Example1-2d-2d
\end{verbatim}
The aforementioned expression is the name of 
the executable that is required to run the 
regression test. Hence, in 
\begin{verbatim}
PDE/StatPDE/Example1/Test> emacs test.sh
\end{verbatim}
you find two times the line
\begin{verbatim}
echo "Running Program 
     ../../../../../bin/DOpE-PDE-StatPDE-Example1-2d-2d test.prm"
     (../../../../../bin/DOpE-PDE-StatPDE-Example1-2d-2d test.prm 2>&1) 
     > /dev/null
\end{verbatim} 
It is really important when copying an existing example,
to change this information. 
First, we have to change the target in the Makefile
\begin{verbatim}
target   = $(BINDIR)/DOpE-PDE-StatPDE-ExampleNew-$
           (dope_dimension)d-$(deal_II_dimension)d
\end{verbatim}
Second, we have to replace 
Example1 by ExampleNew such that 
\begin{verbatim}
echo "Running Program 
     ../../../../../bin/DOpE-PDE-StatPDE-ExampleNew-2d-2d test.prm"
     (../../../../../bin/DOpE-PDE-StatPDE-ExampleNew-2d-2d test.prm 2>&1) 
     > /dev/null
\end{verbatim} 
Please, also change the dimension parameters 
\begin{verbatim}
-2d-2d
\end{verbatim} 
Now, the user is prepared to change any information in the 
\begin{verbatim}
main.cc, localpde.h, functionals.h, localfunctional.h, etc
\end{verbatim} 



%%%%%%%%%%%%%%%%%%%%%%%%%%%%%%%%%%%%%%%%%%%%%%%%%%
\chapter{Examples for PDE Solution}
\label{PDE}
%%%%%%%%%%%%%%%%%%%%%%%%%%%%%%%%%%%%%%%%%%%%%%%%%%
\section{Stationary PDEs} 
\label{PDE_Stat}
%%%%%%%%%%%%%%%%%%%%%%%%%%%%%%%%%%%%%%%%%%%%%%%%%%
\subsection{Stationary Stokes Equations} 
\label{PDE_Stat_Stokes}
\subsubsection{General problem description}

In this example we consider the following benchmark problem from elasticity theory: \index{stationary PDE} \index{stationary PDE!elasticity equations}
\begin{equation}
   (\sigma(u),\varepsilon(\varphi)) = (g,\varphi)_{\Gamma_N}.
\end{equation}
Here $\tilde{\Omega}$ is a quadratic domain with side length $200$ mm, where a circular hole with radius $10$ mm around the center is cut out. Using symmetries of the domain, we restrict our actual computational domain $\Omega$ to the upper left quarter of $\tilde{\Omega}$.\\
In the above equation, $\varepsilon(v) := \frac{1}{2}(\nabla v + \nabla v^T)$ is the symmetric strain tensor, and \index{strain tensor}
\begin{equation*}
   \sigma(v) := 2\mu \varepsilon(v)^D + \kappa \cdot tr (\varepsilon(v)) I
\end{equation*}
denotes the symmetric stress tensor. Here $\tau^D$ is the deviatoric part of a tensor $\tau$, in two dimensions defined as \index{deviator}
\begin{equation*}
   \tau^D := \tau - \frac{1}{2} tr(\tau) I,
\end{equation*}
and the parameters $\mu$ and $\kappa$ are chosen as $\mu = 80193.800283$ resp. $\kappa = 190937.589172$.\\
The corner points of our computational domain are in anticlockwise order: $(0,0), (90,0)$, $(100,10), (100,100)$ and $(0,100)$. We prescribe homogeneous Dirichlet boundary conditions in $y$-direction between $(0,0)$ and $(90,0)$ (lower boundary part), homogeneous Dirichlet boundary conditions in $x$-direction between $(100,10)$ and $(100,100)$ (right boundary part), and we interpret the right hand side of equation $(1)$ with $g = 450$ as a boundary condition between $(0,100)$ and $(100,100)$ (upper boundary part).\\
The goal of our computations is to match the following functional reference values taken from \textit{E. Stein (editor), Error-controlled Adaptive Finite Elements in Solid Mechanics, Wiley (2003), pp. 386 - 387}:
\begin{table}[h]
\centering
\begin{tabular}{lccc}    
\hline
 Functional & $u_1$ at $(90,0)$ & $\sigma_{22}$ at $(90,0)$ & $u_2$ at $(100,100)$\\
\hline 
 Reference value & 0.021290 & 1388.732343 & 0.20951\\
\hline
\end{tabular}
\end{table}
\begin{table}[h]
\centering
\begin{tabular}{lcc}    
\hline
 Functional & $u_1$ at $(0,100)$ & $\int_{(100,100)}^{(0,100)} u_2$\\
\hline 
 Reference value & 0.076758 & 20.40344\\
\hline
\end{tabular}
\end{table}

\subsubsection{Program description}

From the previous examples we know how to read a grid from an \textit{.inp} file. The grid of our current example comes from the above mentioned benchmark problem. The parameter file (\textit{dope.prm}) contains the same variables as in the first two examples.\\
Apart from different point values of derivatives of the solution, we want to evaluate an integral over part of the boundary. \index{functional} \index{functional!boundary integral} This is newly implemented in \textit{functionals.h}.\\
The (linear) elasticity equation is much simpler than the FSI problems we treated before, which can be seen in \textit{localpde.h} where the \texttt{ElementEquation} and \texttt{ElementMatrix} are implemented. In principle, everything is clear from the preceding examples.\\
The \textit{main.cc} file does not contain any innovation, either. Here, we refine the grid globally instead of using an error estimator for local refinement.\\
The output of the program reflects again the linearity of the problem (only one Newton step is needed for solution).\\

\clearpage
%%%%%%%%%%%%%%%%%%%%%%%%%%%%%%%%%%%%%%%%%%%%%%%%%%
\subsection{Stationary FSI with INH Material} 
\label{PDE_Stat_FSI_INH}
\subsubsection{General problem description}

In this example we consider the following benchmark problem from elasticity theory: \index{stationary PDE} \index{stationary PDE!elasticity equations}
\begin{equation}
   (\sigma(u),\varepsilon(\varphi)) = (g,\varphi)_{\Gamma_N}.
\end{equation}
Here $\tilde{\Omega}$ is a quadratic domain with side length $200$ mm, where a circular hole with radius $10$ mm around the center is cut out. Using symmetries of the domain, we restrict our actual computational domain $\Omega$ to the upper left quarter of $\tilde{\Omega}$.\\
In the above equation, $\varepsilon(v) := \frac{1}{2}(\nabla v + \nabla v^T)$ is the symmetric strain tensor, and \index{strain tensor}
\begin{equation*}
   \sigma(v) := 2\mu \varepsilon(v)^D + \kappa \cdot tr (\varepsilon(v)) I
\end{equation*}
denotes the symmetric stress tensor. Here $\tau^D$ is the deviatoric part of a tensor $\tau$, in two dimensions defined as \index{deviator}
\begin{equation*}
   \tau^D := \tau - \frac{1}{2} tr(\tau) I,
\end{equation*}
and the parameters $\mu$ and $\kappa$ are chosen as $\mu = 80193.800283$ resp. $\kappa = 190937.589172$.\\
The corner points of our computational domain are in anticlockwise order: $(0,0), (90,0)$, $(100,10), (100,100)$ and $(0,100)$. We prescribe homogeneous Dirichlet boundary conditions in $y$-direction between $(0,0)$ and $(90,0)$ (lower boundary part), homogeneous Dirichlet boundary conditions in $x$-direction between $(100,10)$ and $(100,100)$ (right boundary part), and we interpret the right hand side of equation $(1)$ with $g = 450$ as a boundary condition between $(0,100)$ and $(100,100)$ (upper boundary part).\\
The goal of our computations is to match the following functional reference values taken from \textit{E. Stein (editor), Error-controlled Adaptive Finite Elements in Solid Mechanics, Wiley (2003), pp. 386 - 387}:
\begin{table}[h]
\centering
\begin{tabular}{lccc}    
\hline
 Functional & $u_1$ at $(90,0)$ & $\sigma_{22}$ at $(90,0)$ & $u_2$ at $(100,100)$\\
\hline 
 Reference value & 0.021290 & 1388.732343 & 0.20951\\
\hline
\end{tabular}
\end{table}
\begin{table}[h]
\centering
\begin{tabular}{lcc}    
\hline
 Functional & $u_1$ at $(0,100)$ & $\int_{(100,100)}^{(0,100)} u_2$\\
\hline 
 Reference value & 0.076758 & 20.40344\\
\hline
\end{tabular}
\end{table}

\subsubsection{Program description}

From the previous examples we know how to read a grid from an \textit{.inp} file. The grid of our current example comes from the above mentioned benchmark problem. The parameter file (\textit{dope.prm}) contains the same variables as in the first two examples.\\
Apart from different point values of derivatives of the solution, we want to evaluate an integral over part of the boundary. \index{functional} \index{functional!boundary integral} This is newly implemented in \textit{functionals.h}.\\
The (linear) elasticity equation is much simpler than the FSI problems we treated before, which can be seen in \textit{localpde.h} where the \texttt{ElementEquation} and \texttt{ElementMatrix} are implemented. In principle, everything is clear from the preceding examples.\\
The \textit{main.cc} file does not contain any innovation, either. Here, we refine the grid globally instead of using an error estimator for local refinement.\\
The output of the program reflects again the linearity of the problem (only one Newton step is needed for solution).\\

\clearpage
%%%%%%%%%%%%%%%%%%%%%%%%%%%%%%%%%%%%%%%%%%%%%%%%%%
\subsection{Stationary FSI with STVK Material} 
\label{PDE_Stat_FSI_STVK}
\subsubsection{General problem description}

In this example we consider the following benchmark problem from elasticity theory: \index{stationary PDE} \index{stationary PDE!elasticity equations}
\begin{equation}
   (\sigma(u),\varepsilon(\varphi)) = (g,\varphi)_{\Gamma_N}.
\end{equation}
Here $\tilde{\Omega}$ is a quadratic domain with side length $200$ mm, where a circular hole with radius $10$ mm around the center is cut out. Using symmetries of the domain, we restrict our actual computational domain $\Omega$ to the upper left quarter of $\tilde{\Omega}$.\\
In the above equation, $\varepsilon(v) := \frac{1}{2}(\nabla v + \nabla v^T)$ is the symmetric strain tensor, and \index{strain tensor}
\begin{equation*}
   \sigma(v) := 2\mu \varepsilon(v)^D + \kappa \cdot tr (\varepsilon(v)) I
\end{equation*}
denotes the symmetric stress tensor. Here $\tau^D$ is the deviatoric part of a tensor $\tau$, in two dimensions defined as \index{deviator}
\begin{equation*}
   \tau^D := \tau - \frac{1}{2} tr(\tau) I,
\end{equation*}
and the parameters $\mu$ and $\kappa$ are chosen as $\mu = 80193.800283$ resp. $\kappa = 190937.589172$.\\
The corner points of our computational domain are in anticlockwise order: $(0,0), (90,0)$, $(100,10), (100,100)$ and $(0,100)$. We prescribe homogeneous Dirichlet boundary conditions in $y$-direction between $(0,0)$ and $(90,0)$ (lower boundary part), homogeneous Dirichlet boundary conditions in $x$-direction between $(100,10)$ and $(100,100)$ (right boundary part), and we interpret the right hand side of equation $(1)$ with $g = 450$ as a boundary condition between $(0,100)$ and $(100,100)$ (upper boundary part).\\
The goal of our computations is to match the following functional reference values taken from \textit{E. Stein (editor), Error-controlled Adaptive Finite Elements in Solid Mechanics, Wiley (2003), pp. 386 - 387}:
\begin{table}[h]
\centering
\begin{tabular}{lccc}    
\hline
 Functional & $u_1$ at $(90,0)$ & $\sigma_{22}$ at $(90,0)$ & $u_2$ at $(100,100)$\\
\hline 
 Reference value & 0.021290 & 1388.732343 & 0.20951\\
\hline
\end{tabular}
\end{table}
\begin{table}[h]
\centering
\begin{tabular}{lcc}    
\hline
 Functional & $u_1$ at $(0,100)$ & $\int_{(100,100)}^{(0,100)} u_2$\\
\hline 
 Reference value & 0.076758 & 20.40344\\
\hline
\end{tabular}
\end{table}

\subsubsection{Program description}

From the previous examples we know how to read a grid from an \textit{.inp} file. The grid of our current example comes from the above mentioned benchmark problem. The parameter file (\textit{dope.prm}) contains the same variables as in the first two examples.\\
Apart from different point values of derivatives of the solution, we want to evaluate an integral over part of the boundary. \index{functional} \index{functional!boundary integral} This is newly implemented in \textit{functionals.h}.\\
The (linear) elasticity equation is much simpler than the FSI problems we treated before, which can be seen in \textit{localpde.h} where the \texttt{ElementEquation} and \texttt{ElementMatrix} are implemented. In principle, everything is clear from the preceding examples.\\
The \textit{main.cc} file does not contain any innovation, either. Here, we refine the grid globally instead of using an error estimator for local refinement.\\
The output of the program reflects again the linearity of the problem (only one Newton step is needed for solution).\\

\clearpage
%%%%%%%%%%%%%%%%%%%%%%%%%%%%%%%%%%%%%%%%%%%%%%%%%%
\subsection{Stationary Elasticity Benchmark} 
\label{PDE_Stat_Elasticity}
\subsubsection{General problem description}

In this example we consider the following benchmark problem from elasticity theory: \index{stationary PDE} \index{stationary PDE!elasticity equations}
\begin{equation}
   (\sigma(u),\varepsilon(\varphi)) = (g,\varphi)_{\Gamma_N}.
\end{equation}
Here $\tilde{\Omega}$ is a quadratic domain with side length $200$ mm, where a circular hole with radius $10$ mm around the center is cut out. Using symmetries of the domain, we restrict our actual computational domain $\Omega$ to the upper left quarter of $\tilde{\Omega}$.\\
In the above equation, $\varepsilon(v) := \frac{1}{2}(\nabla v + \nabla v^T)$ is the symmetric strain tensor, and \index{strain tensor}
\begin{equation*}
   \sigma(v) := 2\mu \varepsilon(v)^D + \kappa \cdot tr (\varepsilon(v)) I
\end{equation*}
denotes the symmetric stress tensor. Here $\tau^D$ is the deviatoric part of a tensor $\tau$, in two dimensions defined as \index{deviator}
\begin{equation*}
   \tau^D := \tau - \frac{1}{2} tr(\tau) I,
\end{equation*}
and the parameters $\mu$ and $\kappa$ are chosen as $\mu = 80193.800283$ resp. $\kappa = 190937.589172$.\\
The corner points of our computational domain are in anticlockwise order: $(0,0), (90,0)$, $(100,10), (100,100)$ and $(0,100)$. We prescribe homogeneous Dirichlet boundary conditions in $y$-direction between $(0,0)$ and $(90,0)$ (lower boundary part), homogeneous Dirichlet boundary conditions in $x$-direction between $(100,10)$ and $(100,100)$ (right boundary part), and we interpret the right hand side of equation $(1)$ with $g = 450$ as a boundary condition between $(0,100)$ and $(100,100)$ (upper boundary part).\\
The goal of our computations is to match the following functional reference values taken from \textit{E. Stein (editor), Error-controlled Adaptive Finite Elements in Solid Mechanics, Wiley (2003), pp. 386 - 387}:
\begin{table}[h]
\centering
\begin{tabular}{lccc}    
\hline
 Functional & $u_1$ at $(90,0)$ & $\sigma_{22}$ at $(90,0)$ & $u_2$ at $(100,100)$\\
\hline 
 Reference value & 0.021290 & 1388.732343 & 0.20951\\
\hline
\end{tabular}
\end{table}
\begin{table}[h]
\centering
\begin{tabular}{lcc}    
\hline
 Functional & $u_1$ at $(0,100)$ & $\int_{(100,100)}^{(0,100)} u_2$\\
\hline 
 Reference value & 0.076758 & 20.40344\\
\hline
\end{tabular}
\end{table}

\subsubsection{Program description}

From the previous examples we know how to read a grid from an \textit{.inp} file. The grid of our current example comes from the above mentioned benchmark problem. The parameter file (\textit{dope.prm}) contains the same variables as in the first two examples.\\
Apart from different point values of derivatives of the solution, we want to evaluate an integral over part of the boundary. \index{functional} \index{functional!boundary integral} This is newly implemented in \textit{functionals.h}.\\
The (linear) elasticity equation is much simpler than the FSI problems we treated before, which can be seen in \textit{localpde.h} where the \texttt{ElementEquation} and \texttt{ElementMatrix} are implemented. In principle, everything is clear from the preceding examples.\\
The \textit{main.cc} file does not contain any innovation, either. Here, we refine the grid globally instead of using an error estimator for local refinement.\\
The output of the program reflects again the linearity of the problem (only one Newton step is needed for solution).\\

\clearpage
%%%%%%%%%%%%%%%%%%%%%%%%%%%%%%%%%%%%%%%%%%%%%%%%%%
\subsection{Stationary Plasticity Benchmark} 
\label{PDE_Stat_Plasticity}
\subsubsection{General problem description}

In this example we consider the following benchmark problem from elasticity theory: \index{stationary PDE} \index{stationary PDE!elasticity equations}
\begin{equation}
   (\sigma(u),\varepsilon(\varphi)) = (g,\varphi)_{\Gamma_N}.
\end{equation}
Here $\tilde{\Omega}$ is a quadratic domain with side length $200$ mm, where a circular hole with radius $10$ mm around the center is cut out. Using symmetries of the domain, we restrict our actual computational domain $\Omega$ to the upper left quarter of $\tilde{\Omega}$.\\
In the above equation, $\varepsilon(v) := \frac{1}{2}(\nabla v + \nabla v^T)$ is the symmetric strain tensor, and \index{strain tensor}
\begin{equation*}
   \sigma(v) := 2\mu \varepsilon(v)^D + \kappa \cdot tr (\varepsilon(v)) I
\end{equation*}
denotes the symmetric stress tensor. Here $\tau^D$ is the deviatoric part of a tensor $\tau$, in two dimensions defined as \index{deviator}
\begin{equation*}
   \tau^D := \tau - \frac{1}{2} tr(\tau) I,
\end{equation*}
and the parameters $\mu$ and $\kappa$ are chosen as $\mu = 80193.800283$ resp. $\kappa = 190937.589172$.\\
The corner points of our computational domain are in anticlockwise order: $(0,0), (90,0)$, $(100,10), (100,100)$ and $(0,100)$. We prescribe homogeneous Dirichlet boundary conditions in $y$-direction between $(0,0)$ and $(90,0)$ (lower boundary part), homogeneous Dirichlet boundary conditions in $x$-direction between $(100,10)$ and $(100,100)$ (right boundary part), and we interpret the right hand side of equation $(1)$ with $g = 450$ as a boundary condition between $(0,100)$ and $(100,100)$ (upper boundary part).\\
The goal of our computations is to match the following functional reference values taken from \textit{E. Stein (editor), Error-controlled Adaptive Finite Elements in Solid Mechanics, Wiley (2003), pp. 386 - 387}:
\begin{table}[h]
\centering
\begin{tabular}{lccc}    
\hline
 Functional & $u_1$ at $(90,0)$ & $\sigma_{22}$ at $(90,0)$ & $u_2$ at $(100,100)$\\
\hline 
 Reference value & 0.021290 & 1388.732343 & 0.20951\\
\hline
\end{tabular}
\end{table}
\begin{table}[h]
\centering
\begin{tabular}{lcc}    
\hline
 Functional & $u_1$ at $(0,100)$ & $\int_{(100,100)}^{(0,100)} u_2$\\
\hline 
 Reference value & 0.076758 & 20.40344\\
\hline
\end{tabular}
\end{table}

\subsubsection{Program description}

From the previous examples we know how to read a grid from an \textit{.inp} file. The grid of our current example comes from the above mentioned benchmark problem. The parameter file (\textit{dope.prm}) contains the same variables as in the first two examples.\\
Apart from different point values of derivatives of the solution, we want to evaluate an integral over part of the boundary. \index{functional} \index{functional!boundary integral} This is newly implemented in \textit{functionals.h}.\\
The (linear) elasticity equation is much simpler than the FSI problems we treated before, which can be seen in \textit{localpde.h} where the \texttt{ElementEquation} and \texttt{ElementMatrix} are implemented. In principle, everything is clear from the preceding examples.\\
The \textit{main.cc} file does not contain any innovation, either. Here, we refine the grid globally instead of using an error estimator for local refinement.\\
The output of the program reflects again the linearity of the problem (only one Newton step is needed for solution).\\

\clearpage
%%%%%%%%%%%%%%%%%%%%%%%%%%%%%%%%%%%%%%%%%%%%%%%%%%
\subsection{Stationary Stokes Equations with periodic BC} 
\label{PDE_Stat_Periodic_Stokes}
\subsubsection{General problem description}

In this example we consider the following benchmark problem from elasticity theory: \index{stationary PDE} \index{stationary PDE!elasticity equations}
\begin{equation}
   (\sigma(u),\varepsilon(\varphi)) = (g,\varphi)_{\Gamma_N}.
\end{equation}
Here $\tilde{\Omega}$ is a quadratic domain with side length $200$ mm, where a circular hole with radius $10$ mm around the center is cut out. Using symmetries of the domain, we restrict our actual computational domain $\Omega$ to the upper left quarter of $\tilde{\Omega}$.\\
In the above equation, $\varepsilon(v) := \frac{1}{2}(\nabla v + \nabla v^T)$ is the symmetric strain tensor, and \index{strain tensor}
\begin{equation*}
   \sigma(v) := 2\mu \varepsilon(v)^D + \kappa \cdot tr (\varepsilon(v)) I
\end{equation*}
denotes the symmetric stress tensor. Here $\tau^D$ is the deviatoric part of a tensor $\tau$, in two dimensions defined as \index{deviator}
\begin{equation*}
   \tau^D := \tau - \frac{1}{2} tr(\tau) I,
\end{equation*}
and the parameters $\mu$ and $\kappa$ are chosen as $\mu = 80193.800283$ resp. $\kappa = 190937.589172$.\\
The corner points of our computational domain are in anticlockwise order: $(0,0), (90,0)$, $(100,10), (100,100)$ and $(0,100)$. We prescribe homogeneous Dirichlet boundary conditions in $y$-direction between $(0,0)$ and $(90,0)$ (lower boundary part), homogeneous Dirichlet boundary conditions in $x$-direction between $(100,10)$ and $(100,100)$ (right boundary part), and we interpret the right hand side of equation $(1)$ with $g = 450$ as a boundary condition between $(0,100)$ and $(100,100)$ (upper boundary part).\\
The goal of our computations is to match the following functional reference values taken from \textit{E. Stein (editor), Error-controlled Adaptive Finite Elements in Solid Mechanics, Wiley (2003), pp. 386 - 387}:
\begin{table}[h]
\centering
\begin{tabular}{lccc}    
\hline
 Functional & $u_1$ at $(90,0)$ & $\sigma_{22}$ at $(90,0)$ & $u_2$ at $(100,100)$\\
\hline 
 Reference value & 0.021290 & 1388.732343 & 0.20951\\
\hline
\end{tabular}
\end{table}
\begin{table}[h]
\centering
\begin{tabular}{lcc}    
\hline
 Functional & $u_1$ at $(0,100)$ & $\int_{(100,100)}^{(0,100)} u_2$\\
\hline 
 Reference value & 0.076758 & 20.40344\\
\hline
\end{tabular}
\end{table}

\subsubsection{Program description}

From the previous examples we know how to read a grid from an \textit{.inp} file. The grid of our current example comes from the above mentioned benchmark problem. The parameter file (\textit{dope.prm}) contains the same variables as in the first two examples.\\
Apart from different point values of derivatives of the solution, we want to evaluate an integral over part of the boundary. \index{functional} \index{functional!boundary integral} This is newly implemented in \textit{functionals.h}.\\
The (linear) elasticity equation is much simpler than the FSI problems we treated before, which can be seen in \textit{localpde.h} where the \texttt{ElementEquation} and \texttt{ElementMatrix} are implemented. In principle, everything is clear from the preceding examples.\\
The \textit{main.cc} file does not contain any innovation, either. Here, we refine the grid globally instead of using an error estimator for local refinement.\\
The output of the program reflects again the linearity of the problem (only one Newton step is needed for solution).\\

\clearpage
%%%%%%%%%%%%%%%%%%%%%%%%%%%%%%%%%%%%%%%%%%%%%%%%%%
\subsection{Laplace Equation in 2D} 
\label{PDE_Stat_Laplace_2D}
\subsubsection{General problem description}

In this example we consider the following benchmark problem from elasticity theory: \index{stationary PDE} \index{stationary PDE!elasticity equations}
\begin{equation}
   (\sigma(u),\varepsilon(\varphi)) = (g,\varphi)_{\Gamma_N}.
\end{equation}
Here $\tilde{\Omega}$ is a quadratic domain with side length $200$ mm, where a circular hole with radius $10$ mm around the center is cut out. Using symmetries of the domain, we restrict our actual computational domain $\Omega$ to the upper left quarter of $\tilde{\Omega}$.\\
In the above equation, $\varepsilon(v) := \frac{1}{2}(\nabla v + \nabla v^T)$ is the symmetric strain tensor, and \index{strain tensor}
\begin{equation*}
   \sigma(v) := 2\mu \varepsilon(v)^D + \kappa \cdot tr (\varepsilon(v)) I
\end{equation*}
denotes the symmetric stress tensor. Here $\tau^D$ is the deviatoric part of a tensor $\tau$, in two dimensions defined as \index{deviator}
\begin{equation*}
   \tau^D := \tau - \frac{1}{2} tr(\tau) I,
\end{equation*}
and the parameters $\mu$ and $\kappa$ are chosen as $\mu = 80193.800283$ resp. $\kappa = 190937.589172$.\\
The corner points of our computational domain are in anticlockwise order: $(0,0), (90,0)$, $(100,10), (100,100)$ and $(0,100)$. We prescribe homogeneous Dirichlet boundary conditions in $y$-direction between $(0,0)$ and $(90,0)$ (lower boundary part), homogeneous Dirichlet boundary conditions in $x$-direction between $(100,10)$ and $(100,100)$ (right boundary part), and we interpret the right hand side of equation $(1)$ with $g = 450$ as a boundary condition between $(0,100)$ and $(100,100)$ (upper boundary part).\\
The goal of our computations is to match the following functional reference values taken from \textit{E. Stein (editor), Error-controlled Adaptive Finite Elements in Solid Mechanics, Wiley (2003), pp. 386 - 387}:
\begin{table}[h]
\centering
\begin{tabular}{lccc}    
\hline
 Functional & $u_1$ at $(90,0)$ & $\sigma_{22}$ at $(90,0)$ & $u_2$ at $(100,100)$\\
\hline 
 Reference value & 0.021290 & 1388.732343 & 0.20951\\
\hline
\end{tabular}
\end{table}
\begin{table}[h]
\centering
\begin{tabular}{lcc}    
\hline
 Functional & $u_1$ at $(0,100)$ & $\int_{(100,100)}^{(0,100)} u_2$\\
\hline 
 Reference value & 0.076758 & 20.40344\\
\hline
\end{tabular}
\end{table}

\subsubsection{Program description}

From the previous examples we know how to read a grid from an \textit{.inp} file. The grid of our current example comes from the above mentioned benchmark problem. The parameter file (\textit{dope.prm}) contains the same variables as in the first two examples.\\
Apart from different point values of derivatives of the solution, we want to evaluate an integral over part of the boundary. \index{functional} \index{functional!boundary integral} This is newly implemented in \textit{functionals.h}.\\
The (linear) elasticity equation is much simpler than the FSI problems we treated before, which can be seen in \textit{localpde.h} where the \texttt{ElementEquation} and \texttt{ElementMatrix} are implemented. In principle, everything is clear from the preceding examples.\\
The \textit{main.cc} file does not contain any innovation, either. Here, we refine the grid globally instead of using an error estimator for local refinement.\\
The output of the program reflects again the linearity of the problem (only one Newton step is needed for solution).\\

\clearpage
%%%%%%%%%%%%%%%%%%%%%%%%%%%%%%%%%%%%%%%%%%%%%%%%%%
\subsection{Laplace Equation in 3D}
\label{PDE_Stat_Laplace_3D}
\subsubsection{General problem description}

In this example we consider the following benchmark problem from elasticity theory: \index{stationary PDE} \index{stationary PDE!elasticity equations}
\begin{equation}
   (\sigma(u),\varepsilon(\varphi)) = (g,\varphi)_{\Gamma_N}.
\end{equation}
Here $\tilde{\Omega}$ is a quadratic domain with side length $200$ mm, where a circular hole with radius $10$ mm around the center is cut out. Using symmetries of the domain, we restrict our actual computational domain $\Omega$ to the upper left quarter of $\tilde{\Omega}$.\\
In the above equation, $\varepsilon(v) := \frac{1}{2}(\nabla v + \nabla v^T)$ is the symmetric strain tensor, and \index{strain tensor}
\begin{equation*}
   \sigma(v) := 2\mu \varepsilon(v)^D + \kappa \cdot tr (\varepsilon(v)) I
\end{equation*}
denotes the symmetric stress tensor. Here $\tau^D$ is the deviatoric part of a tensor $\tau$, in two dimensions defined as \index{deviator}
\begin{equation*}
   \tau^D := \tau - \frac{1}{2} tr(\tau) I,
\end{equation*}
and the parameters $\mu$ and $\kappa$ are chosen as $\mu = 80193.800283$ resp. $\kappa = 190937.589172$.\\
The corner points of our computational domain are in anticlockwise order: $(0,0), (90,0)$, $(100,10), (100,100)$ and $(0,100)$. We prescribe homogeneous Dirichlet boundary conditions in $y$-direction between $(0,0)$ and $(90,0)$ (lower boundary part), homogeneous Dirichlet boundary conditions in $x$-direction between $(100,10)$ and $(100,100)$ (right boundary part), and we interpret the right hand side of equation $(1)$ with $g = 450$ as a boundary condition between $(0,100)$ and $(100,100)$ (upper boundary part).\\
The goal of our computations is to match the following functional reference values taken from \textit{E. Stein (editor), Error-controlled Adaptive Finite Elements in Solid Mechanics, Wiley (2003), pp. 386 - 387}:
\begin{table}[h]
\centering
\begin{tabular}{lccc}    
\hline
 Functional & $u_1$ at $(90,0)$ & $\sigma_{22}$ at $(90,0)$ & $u_2$ at $(100,100)$\\
\hline 
 Reference value & 0.021290 & 1388.732343 & 0.20951\\
\hline
\end{tabular}
\end{table}
\begin{table}[h]
\centering
\begin{tabular}{lcc}    
\hline
 Functional & $u_1$ at $(0,100)$ & $\int_{(100,100)}^{(0,100)} u_2$\\
\hline 
 Reference value & 0.076758 & 20.40344\\
\hline
\end{tabular}
\end{table}

\subsubsection{Program description}

From the previous examples we know how to read a grid from an \textit{.inp} file. The grid of our current example comes from the above mentioned benchmark problem. The parameter file (\textit{dope.prm}) contains the same variables as in the first two examples.\\
Apart from different point values of derivatives of the solution, we want to evaluate an integral over part of the boundary. \index{functional} \index{functional!boundary integral} This is newly implemented in \textit{functionals.h}.\\
The (linear) elasticity equation is much simpler than the FSI problems we treated before, which can be seen in \textit{localpde.h} where the \texttt{ElementEquation} and \texttt{ElementMatrix} are implemented. In principle, everything is clear from the preceding examples.\\
The \textit{main.cc} file does not contain any innovation, either. Here, we refine the grid globally instead of using an error estimator for local refinement.\\
The output of the program reflects again the linearity of the problem (only one Newton step is needed for solution).\\

\clearpage
%%%%%%%%%%%%%%%%%%%%%%%%%%%%%%%%%%%%%%%%%%%%%%%%%%
\section{Nonstationary PDEs}
\label{PDE_Instat}
\subsubsection{General problem description}

In this example we consider the following benchmark problem from elasticity theory: \index{stationary PDE} \index{stationary PDE!elasticity equations}
\begin{equation}
   (\sigma(u),\varepsilon(\varphi)) = (g,\varphi)_{\Gamma_N}.
\end{equation}
Here $\tilde{\Omega}$ is a quadratic domain with side length $200$ mm, where a circular hole with radius $10$ mm around the center is cut out. Using symmetries of the domain, we restrict our actual computational domain $\Omega$ to the upper left quarter of $\tilde{\Omega}$.\\
In the above equation, $\varepsilon(v) := \frac{1}{2}(\nabla v + \nabla v^T)$ is the symmetric strain tensor, and \index{strain tensor}
\begin{equation*}
   \sigma(v) := 2\mu \varepsilon(v)^D + \kappa \cdot tr (\varepsilon(v)) I
\end{equation*}
denotes the symmetric stress tensor. Here $\tau^D$ is the deviatoric part of a tensor $\tau$, in two dimensions defined as \index{deviator}
\begin{equation*}
   \tau^D := \tau - \frac{1}{2} tr(\tau) I,
\end{equation*}
and the parameters $\mu$ and $\kappa$ are chosen as $\mu = 80193.800283$ resp. $\kappa = 190937.589172$.\\
The corner points of our computational domain are in anticlockwise order: $(0,0), (90,0)$, $(100,10), (100,100)$ and $(0,100)$. We prescribe homogeneous Dirichlet boundary conditions in $y$-direction between $(0,0)$ and $(90,0)$ (lower boundary part), homogeneous Dirichlet boundary conditions in $x$-direction between $(100,10)$ and $(100,100)$ (right boundary part), and we interpret the right hand side of equation $(1)$ with $g = 450$ as a boundary condition between $(0,100)$ and $(100,100)$ (upper boundary part).\\
The goal of our computations is to match the following functional reference values taken from \textit{E. Stein (editor), Error-controlled Adaptive Finite Elements in Solid Mechanics, Wiley (2003), pp. 386 - 387}:
\begin{table}[h]
\centering
\begin{tabular}{lccc}    
\hline
 Functional & $u_1$ at $(90,0)$ & $\sigma_{22}$ at $(90,0)$ & $u_2$ at $(100,100)$\\
\hline 
 Reference value & 0.021290 & 1388.732343 & 0.20951\\
\hline
\end{tabular}
\end{table}
\begin{table}[h]
\centering
\begin{tabular}{lcc}    
\hline
 Functional & $u_1$ at $(0,100)$ & $\int_{(100,100)}^{(0,100)} u_2$\\
\hline 
 Reference value & 0.076758 & 20.40344\\
\hline
\end{tabular}
\end{table}

\subsubsection{Program description}

From the previous examples we know how to read a grid from an \textit{.inp} file. The grid of our current example comes from the above mentioned benchmark problem. The parameter file (\textit{dope.prm}) contains the same variables as in the first two examples.\\
Apart from different point values of derivatives of the solution, we want to evaluate an integral over part of the boundary. \index{functional} \index{functional!boundary integral} This is newly implemented in \textit{functionals.h}.\\
The (linear) elasticity equation is much simpler than the FSI problems we treated before, which can be seen in \textit{localpde.h} where the \texttt{ElementEquation} and \texttt{ElementMatrix} are implemented. In principle, everything is clear from the preceding examples.\\
The \textit{main.cc} file does not contain any innovation, either. Here, we refine the grid globally instead of using an error estimator for local refinement.\\
The output of the program reflects again the linearity of the problem (only one Newton step is needed for solution).\\

%%%%%%%%%%%%%%%%%%%%%%%%%%%%%%%%%%%%%%%%%%%%%%%%%%
\subsection{Nonstationary Stokes Equations}
\label{PDE_Instat_Stokes}
\subsubsection{General problem description}

In this example we consider the following benchmark problem from elasticity theory: \index{stationary PDE} \index{stationary PDE!elasticity equations}
\begin{equation}
   (\sigma(u),\varepsilon(\varphi)) = (g,\varphi)_{\Gamma_N}.
\end{equation}
Here $\tilde{\Omega}$ is a quadratic domain with side length $200$ mm, where a circular hole with radius $10$ mm around the center is cut out. Using symmetries of the domain, we restrict our actual computational domain $\Omega$ to the upper left quarter of $\tilde{\Omega}$.\\
In the above equation, $\varepsilon(v) := \frac{1}{2}(\nabla v + \nabla v^T)$ is the symmetric strain tensor, and \index{strain tensor}
\begin{equation*}
   \sigma(v) := 2\mu \varepsilon(v)^D + \kappa \cdot tr (\varepsilon(v)) I
\end{equation*}
denotes the symmetric stress tensor. Here $\tau^D$ is the deviatoric part of a tensor $\tau$, in two dimensions defined as \index{deviator}
\begin{equation*}
   \tau^D := \tau - \frac{1}{2} tr(\tau) I,
\end{equation*}
and the parameters $\mu$ and $\kappa$ are chosen as $\mu = 80193.800283$ resp. $\kappa = 190937.589172$.\\
The corner points of our computational domain are in anticlockwise order: $(0,0), (90,0)$, $(100,10), (100,100)$ and $(0,100)$. We prescribe homogeneous Dirichlet boundary conditions in $y$-direction between $(0,0)$ and $(90,0)$ (lower boundary part), homogeneous Dirichlet boundary conditions in $x$-direction between $(100,10)$ and $(100,100)$ (right boundary part), and we interpret the right hand side of equation $(1)$ with $g = 450$ as a boundary condition between $(0,100)$ and $(100,100)$ (upper boundary part).\\
The goal of our computations is to match the following functional reference values taken from \textit{E. Stein (editor), Error-controlled Adaptive Finite Elements in Solid Mechanics, Wiley (2003), pp. 386 - 387}:
\begin{table}[h]
\centering
\begin{tabular}{lccc}    
\hline
 Functional & $u_1$ at $(90,0)$ & $\sigma_{22}$ at $(90,0)$ & $u_2$ at $(100,100)$\\
\hline 
 Reference value & 0.021290 & 1388.732343 & 0.20951\\
\hline
\end{tabular}
\end{table}
\begin{table}[h]
\centering
\begin{tabular}{lcc}    
\hline
 Functional & $u_1$ at $(0,100)$ & $\int_{(100,100)}^{(0,100)} u_2$\\
\hline 
 Reference value & 0.076758 & 20.40344\\
\hline
\end{tabular}
\end{table}

\subsubsection{Program description}

From the previous examples we know how to read a grid from an \textit{.inp} file. The grid of our current example comes from the above mentioned benchmark problem. The parameter file (\textit{dope.prm}) contains the same variables as in the first two examples.\\
Apart from different point values of derivatives of the solution, we want to evaluate an integral over part of the boundary. \index{functional} \index{functional!boundary integral} This is newly implemented in \textit{functionals.h}.\\
The (linear) elasticity equation is much simpler than the FSI problems we treated before, which can be seen in \textit{localpde.h} where the \texttt{ElementEquation} and \texttt{ElementMatrix} are implemented. In principle, everything is clear from the preceding examples.\\
The \textit{main.cc} file does not contain any innovation, either. Here, we refine the grid globally instead of using an error estimator for local refinement.\\
The output of the program reflects again the linearity of the problem (only one Newton step is needed for solution).\\

\clearpage
%%%%%%%%%%%%%%%%%%%%%%%%%%%%%%%%%%%%%%%%%%%%%%%%%%
\subsection{Nonstationary FSI Problem}
\label{PDE_Instat_FSI}
\subsubsection{General problem description}

In this example we consider the following benchmark problem from elasticity theory: \index{stationary PDE} \index{stationary PDE!elasticity equations}
\begin{equation}
   (\sigma(u),\varepsilon(\varphi)) = (g,\varphi)_{\Gamma_N}.
\end{equation}
Here $\tilde{\Omega}$ is a quadratic domain with side length $200$ mm, where a circular hole with radius $10$ mm around the center is cut out. Using symmetries of the domain, we restrict our actual computational domain $\Omega$ to the upper left quarter of $\tilde{\Omega}$.\\
In the above equation, $\varepsilon(v) := \frac{1}{2}(\nabla v + \nabla v^T)$ is the symmetric strain tensor, and \index{strain tensor}
\begin{equation*}
   \sigma(v) := 2\mu \varepsilon(v)^D + \kappa \cdot tr (\varepsilon(v)) I
\end{equation*}
denotes the symmetric stress tensor. Here $\tau^D$ is the deviatoric part of a tensor $\tau$, in two dimensions defined as \index{deviator}
\begin{equation*}
   \tau^D := \tau - \frac{1}{2} tr(\tau) I,
\end{equation*}
and the parameters $\mu$ and $\kappa$ are chosen as $\mu = 80193.800283$ resp. $\kappa = 190937.589172$.\\
The corner points of our computational domain are in anticlockwise order: $(0,0), (90,0)$, $(100,10), (100,100)$ and $(0,100)$. We prescribe homogeneous Dirichlet boundary conditions in $y$-direction between $(0,0)$ and $(90,0)$ (lower boundary part), homogeneous Dirichlet boundary conditions in $x$-direction between $(100,10)$ and $(100,100)$ (right boundary part), and we interpret the right hand side of equation $(1)$ with $g = 450$ as a boundary condition between $(0,100)$ and $(100,100)$ (upper boundary part).\\
The goal of our computations is to match the following functional reference values taken from \textit{E. Stein (editor), Error-controlled Adaptive Finite Elements in Solid Mechanics, Wiley (2003), pp. 386 - 387}:
\begin{table}[h]
\centering
\begin{tabular}{lccc}    
\hline
 Functional & $u_1$ at $(90,0)$ & $\sigma_{22}$ at $(90,0)$ & $u_2$ at $(100,100)$\\
\hline 
 Reference value & 0.021290 & 1388.732343 & 0.20951\\
\hline
\end{tabular}
\end{table}
\begin{table}[h]
\centering
\begin{tabular}{lcc}    
\hline
 Functional & $u_1$ at $(0,100)$ & $\int_{(100,100)}^{(0,100)} u_2$\\
\hline 
 Reference value & 0.076758 & 20.40344\\
\hline
\end{tabular}
\end{table}

\subsubsection{Program description}

From the previous examples we know how to read a grid from an \textit{.inp} file. The grid of our current example comes from the above mentioned benchmark problem. The parameter file (\textit{dope.prm}) contains the same variables as in the first two examples.\\
Apart from different point values of derivatives of the solution, we want to evaluate an integral over part of the boundary. \index{functional} \index{functional!boundary integral} This is newly implemented in \textit{functionals.h}.\\
The (linear) elasticity equation is much simpler than the FSI problems we treated before, which can be seen in \textit{localpde.h} where the \texttt{ElementEquation} and \texttt{ElementMatrix} are implemented. In principle, everything is clear from the preceding examples.\\
The \textit{main.cc} file does not contain any innovation, either. Here, we refine the grid globally instead of using an error estimator for local refinement.\\
The output of the program reflects again the linearity of the problem (only one Newton step is needed for solution).\\

\clearpage
%%%%%%%%%%%%%%%%%%%%%%%%%%%%%%%%%%%%%%%%%%%%%%%%%%
\subsection{Black-Scholes Equation}
\label{PDE_Instat_Black_Scholes}
\subsubsection{General problem description}

In this example we consider the following benchmark problem from elasticity theory: \index{stationary PDE} \index{stationary PDE!elasticity equations}
\begin{equation}
   (\sigma(u),\varepsilon(\varphi)) = (g,\varphi)_{\Gamma_N}.
\end{equation}
Here $\tilde{\Omega}$ is a quadratic domain with side length $200$ mm, where a circular hole with radius $10$ mm around the center is cut out. Using symmetries of the domain, we restrict our actual computational domain $\Omega$ to the upper left quarter of $\tilde{\Omega}$.\\
In the above equation, $\varepsilon(v) := \frac{1}{2}(\nabla v + \nabla v^T)$ is the symmetric strain tensor, and \index{strain tensor}
\begin{equation*}
   \sigma(v) := 2\mu \varepsilon(v)^D + \kappa \cdot tr (\varepsilon(v)) I
\end{equation*}
denotes the symmetric stress tensor. Here $\tau^D$ is the deviatoric part of a tensor $\tau$, in two dimensions defined as \index{deviator}
\begin{equation*}
   \tau^D := \tau - \frac{1}{2} tr(\tau) I,
\end{equation*}
and the parameters $\mu$ and $\kappa$ are chosen as $\mu = 80193.800283$ resp. $\kappa = 190937.589172$.\\
The corner points of our computational domain are in anticlockwise order: $(0,0), (90,0)$, $(100,10), (100,100)$ and $(0,100)$. We prescribe homogeneous Dirichlet boundary conditions in $y$-direction between $(0,0)$ and $(90,0)$ (lower boundary part), homogeneous Dirichlet boundary conditions in $x$-direction between $(100,10)$ and $(100,100)$ (right boundary part), and we interpret the right hand side of equation $(1)$ with $g = 450$ as a boundary condition between $(0,100)$ and $(100,100)$ (upper boundary part).\\
The goal of our computations is to match the following functional reference values taken from \textit{E. Stein (editor), Error-controlled Adaptive Finite Elements in Solid Mechanics, Wiley (2003), pp. 386 - 387}:
\begin{table}[h]
\centering
\begin{tabular}{lccc}    
\hline
 Functional & $u_1$ at $(90,0)$ & $\sigma_{22}$ at $(90,0)$ & $u_2$ at $(100,100)$\\
\hline 
 Reference value & 0.021290 & 1388.732343 & 0.20951\\
\hline
\end{tabular}
\end{table}
\begin{table}[h]
\centering
\begin{tabular}{lcc}    
\hline
 Functional & $u_1$ at $(0,100)$ & $\int_{(100,100)}^{(0,100)} u_2$\\
\hline 
 Reference value & 0.076758 & 20.40344\\
\hline
\end{tabular}
\end{table}

\subsubsection{Program description}

From the previous examples we know how to read a grid from an \textit{.inp} file. The grid of our current example comes from the above mentioned benchmark problem. The parameter file (\textit{dope.prm}) contains the same variables as in the first two examples.\\
Apart from different point values of derivatives of the solution, we want to evaluate an integral over part of the boundary. \index{functional} \index{functional!boundary integral} This is newly implemented in \textit{functionals.h}.\\
The (linear) elasticity equation is much simpler than the FSI problems we treated before, which can be seen in \textit{localpde.h} where the \texttt{ElementEquation} and \texttt{ElementMatrix} are implemented. In principle, everything is clear from the preceding examples.\\
The \textit{main.cc} file does not contain any innovation, either. Here, we refine the grid globally instead of using an error estimator for local refinement.\\
The output of the program reflects again the linearity of the problem (only one Newton step is needed for solution).\\

\clearpage
%%%%%%%%%%%%%%%%%%%%%%%%%%%%%%%%%%%%%%%%%%%%%%%%%%
\subsection{Heat Equation in 1D}
\label{PDE_Instat_Heat_1D}
\subsubsection{General problem description}

In this example we consider the following benchmark problem from elasticity theory: \index{stationary PDE} \index{stationary PDE!elasticity equations}
\begin{equation}
   (\sigma(u),\varepsilon(\varphi)) = (g,\varphi)_{\Gamma_N}.
\end{equation}
Here $\tilde{\Omega}$ is a quadratic domain with side length $200$ mm, where a circular hole with radius $10$ mm around the center is cut out. Using symmetries of the domain, we restrict our actual computational domain $\Omega$ to the upper left quarter of $\tilde{\Omega}$.\\
In the above equation, $\varepsilon(v) := \frac{1}{2}(\nabla v + \nabla v^T)$ is the symmetric strain tensor, and \index{strain tensor}
\begin{equation*}
   \sigma(v) := 2\mu \varepsilon(v)^D + \kappa \cdot tr (\varepsilon(v)) I
\end{equation*}
denotes the symmetric stress tensor. Here $\tau^D$ is the deviatoric part of a tensor $\tau$, in two dimensions defined as \index{deviator}
\begin{equation*}
   \tau^D := \tau - \frac{1}{2} tr(\tau) I,
\end{equation*}
and the parameters $\mu$ and $\kappa$ are chosen as $\mu = 80193.800283$ resp. $\kappa = 190937.589172$.\\
The corner points of our computational domain are in anticlockwise order: $(0,0), (90,0)$, $(100,10), (100,100)$ and $(0,100)$. We prescribe homogeneous Dirichlet boundary conditions in $y$-direction between $(0,0)$ and $(90,0)$ (lower boundary part), homogeneous Dirichlet boundary conditions in $x$-direction between $(100,10)$ and $(100,100)$ (right boundary part), and we interpret the right hand side of equation $(1)$ with $g = 450$ as a boundary condition between $(0,100)$ and $(100,100)$ (upper boundary part).\\
The goal of our computations is to match the following functional reference values taken from \textit{E. Stein (editor), Error-controlled Adaptive Finite Elements in Solid Mechanics, Wiley (2003), pp. 386 - 387}:
\begin{table}[h]
\centering
\begin{tabular}{lccc}    
\hline
 Functional & $u_1$ at $(90,0)$ & $\sigma_{22}$ at $(90,0)$ & $u_2$ at $(100,100)$\\
\hline 
 Reference value & 0.021290 & 1388.732343 & 0.20951\\
\hline
\end{tabular}
\end{table}
\begin{table}[h]
\centering
\begin{tabular}{lcc}    
\hline
 Functional & $u_1$ at $(0,100)$ & $\int_{(100,100)}^{(0,100)} u_2$\\
\hline 
 Reference value & 0.076758 & 20.40344\\
\hline
\end{tabular}
\end{table}

\subsubsection{Program description}

From the previous examples we know how to read a grid from an \textit{.inp} file. The grid of our current example comes from the above mentioned benchmark problem. The parameter file (\textit{dope.prm}) contains the same variables as in the first two examples.\\
Apart from different point values of derivatives of the solution, we want to evaluate an integral over part of the boundary. \index{functional} \index{functional!boundary integral} This is newly implemented in \textit{functionals.h}.\\
The (linear) elasticity equation is much simpler than the FSI problems we treated before, which can be seen in \textit{localpde.h} where the \texttt{ElementEquation} and \texttt{ElementMatrix} are implemented. In principle, everything is clear from the preceding examples.\\
The \textit{main.cc} file does not contain any innovation, either. Here, we refine the grid globally instead of using an error estimator for local refinement.\\
The output of the program reflects again the linearity of the problem (only one Newton step is needed for solution).\\

\clearpage
%%%%%%%%%%%%%%%%%%%%%%%%%%%%%%%%%%%%%%%%%%%%%%%%%%
\subsection{Heat Equation in 2D with nonlinearity}
\label{PDE_Instat_Heat_2D}
\subsubsection{General problem description}

In this example we consider the following benchmark problem from elasticity theory: \index{stationary PDE} \index{stationary PDE!elasticity equations}
\begin{equation}
   (\sigma(u),\varepsilon(\varphi)) = (g,\varphi)_{\Gamma_N}.
\end{equation}
Here $\tilde{\Omega}$ is a quadratic domain with side length $200$ mm, where a circular hole with radius $10$ mm around the center is cut out. Using symmetries of the domain, we restrict our actual computational domain $\Omega$ to the upper left quarter of $\tilde{\Omega}$.\\
In the above equation, $\varepsilon(v) := \frac{1}{2}(\nabla v + \nabla v^T)$ is the symmetric strain tensor, and \index{strain tensor}
\begin{equation*}
   \sigma(v) := 2\mu \varepsilon(v)^D + \kappa \cdot tr (\varepsilon(v)) I
\end{equation*}
denotes the symmetric stress tensor. Here $\tau^D$ is the deviatoric part of a tensor $\tau$, in two dimensions defined as \index{deviator}
\begin{equation*}
   \tau^D := \tau - \frac{1}{2} tr(\tau) I,
\end{equation*}
and the parameters $\mu$ and $\kappa$ are chosen as $\mu = 80193.800283$ resp. $\kappa = 190937.589172$.\\
The corner points of our computational domain are in anticlockwise order: $(0,0), (90,0)$, $(100,10), (100,100)$ and $(0,100)$. We prescribe homogeneous Dirichlet boundary conditions in $y$-direction between $(0,0)$ and $(90,0)$ (lower boundary part), homogeneous Dirichlet boundary conditions in $x$-direction between $(100,10)$ and $(100,100)$ (right boundary part), and we interpret the right hand side of equation $(1)$ with $g = 450$ as a boundary condition between $(0,100)$ and $(100,100)$ (upper boundary part).\\
The goal of our computations is to match the following functional reference values taken from \textit{E. Stein (editor), Error-controlled Adaptive Finite Elements in Solid Mechanics, Wiley (2003), pp. 386 - 387}:
\begin{table}[h]
\centering
\begin{tabular}{lccc}    
\hline
 Functional & $u_1$ at $(90,0)$ & $\sigma_{22}$ at $(90,0)$ & $u_2$ at $(100,100)$\\
\hline 
 Reference value & 0.021290 & 1388.732343 & 0.20951\\
\hline
\end{tabular}
\end{table}
\begin{table}[h]
\centering
\begin{tabular}{lcc}    
\hline
 Functional & $u_1$ at $(0,100)$ & $\int_{(100,100)}^{(0,100)} u_2$\\
\hline 
 Reference value & 0.076758 & 20.40344\\
\hline
\end{tabular}
\end{table}

\subsubsection{Program description}

From the previous examples we know how to read a grid from an \textit{.inp} file. The grid of our current example comes from the above mentioned benchmark problem. The parameter file (\textit{dope.prm}) contains the same variables as in the first two examples.\\
Apart from different point values of derivatives of the solution, we want to evaluate an integral over part of the boundary. \index{functional} \index{functional!boundary integral} This is newly implemented in \textit{functionals.h}.\\
The (linear) elasticity equation is much simpler than the FSI problems we treated before, which can be seen in \textit{localpde.h} where the \texttt{ElementEquation} and \texttt{ElementMatrix} are implemented. In principle, everything is clear from the preceding examples.\\
The \textit{main.cc} file does not contain any innovation, either. Here, we refine the grid globally instead of using an error estimator for local refinement.\\
The output of the program reflects again the linearity of the problem (only one Newton step is needed for solution).\\

\cleardoublepage
%%%%%%%%%%%%%%%%%%%%%%%%%%%%%%%%%%%%%%%%%%%%%%%%%%
\chapter{Examples with Optimization}
\label{OPT}
%%%%%%%%%%%%%%%%%%%%%%%%%%%%%%%%%%%%%%%%%%%%%%%%%%
\section{Subject to a Stationary PDE}
\label{OPT_Stat}
%%%%%%%%%%%%%%%%%%%%%%%%%%%%%%%%%%%%%%%%%%%%%%%%%%
\subsection{Distributed control with a linear elliptic PDE}
\label{OPT_Stat_Distrib_Lin_Ellipt}
\subsubsection{General problem description}

In this example we consider the following benchmark problem from elasticity theory: \index{stationary PDE} \index{stationary PDE!elasticity equations}
\begin{equation}
   (\sigma(u),\varepsilon(\varphi)) = (g,\varphi)_{\Gamma_N}.
\end{equation}
Here $\tilde{\Omega}$ is a quadratic domain with side length $200$ mm, where a circular hole with radius $10$ mm around the center is cut out. Using symmetries of the domain, we restrict our actual computational domain $\Omega$ to the upper left quarter of $\tilde{\Omega}$.\\
In the above equation, $\varepsilon(v) := \frac{1}{2}(\nabla v + \nabla v^T)$ is the symmetric strain tensor, and \index{strain tensor}
\begin{equation*}
   \sigma(v) := 2\mu \varepsilon(v)^D + \kappa \cdot tr (\varepsilon(v)) I
\end{equation*}
denotes the symmetric stress tensor. Here $\tau^D$ is the deviatoric part of a tensor $\tau$, in two dimensions defined as \index{deviator}
\begin{equation*}
   \tau^D := \tau - \frac{1}{2} tr(\tau) I,
\end{equation*}
and the parameters $\mu$ and $\kappa$ are chosen as $\mu = 80193.800283$ resp. $\kappa = 190937.589172$.\\
The corner points of our computational domain are in anticlockwise order: $(0,0), (90,0)$, $(100,10), (100,100)$ and $(0,100)$. We prescribe homogeneous Dirichlet boundary conditions in $y$-direction between $(0,0)$ and $(90,0)$ (lower boundary part), homogeneous Dirichlet boundary conditions in $x$-direction between $(100,10)$ and $(100,100)$ (right boundary part), and we interpret the right hand side of equation $(1)$ with $g = 450$ as a boundary condition between $(0,100)$ and $(100,100)$ (upper boundary part).\\
The goal of our computations is to match the following functional reference values taken from \textit{E. Stein (editor), Error-controlled Adaptive Finite Elements in Solid Mechanics, Wiley (2003), pp. 386 - 387}:
\begin{table}[h]
\centering
\begin{tabular}{lccc}    
\hline
 Functional & $u_1$ at $(90,0)$ & $\sigma_{22}$ at $(90,0)$ & $u_2$ at $(100,100)$\\
\hline 
 Reference value & 0.021290 & 1388.732343 & 0.20951\\
\hline
\end{tabular}
\end{table}
\begin{table}[h]
\centering
\begin{tabular}{lcc}    
\hline
 Functional & $u_1$ at $(0,100)$ & $\int_{(100,100)}^{(0,100)} u_2$\\
\hline 
 Reference value & 0.076758 & 20.40344\\
\hline
\end{tabular}
\end{table}

\subsubsection{Program description}

From the previous examples we know how to read a grid from an \textit{.inp} file. The grid of our current example comes from the above mentioned benchmark problem. The parameter file (\textit{dope.prm}) contains the same variables as in the first two examples.\\
Apart from different point values of derivatives of the solution, we want to evaluate an integral over part of the boundary. \index{functional} \index{functional!boundary integral} This is newly implemented in \textit{functionals.h}.\\
The (linear) elasticity equation is much simpler than the FSI problems we treated before, which can be seen in \textit{localpde.h} where the \texttt{ElementEquation} and \texttt{ElementMatrix} are implemented. In principle, everything is clear from the preceding examples.\\
The \textit{main.cc} file does not contain any innovation, either. Here, we refine the grid globally instead of using an error estimator for local refinement.\\
The output of the program reflects again the linearity of the problem (only one Newton step is needed for solution).\\

\clearpage
%%%%%%%%%%%%%%%%%%%%%%%%%%%%%%%%%%%%%%%%%%%%%%%%%%
\subsection{Parameter control with a linear elliptic PDE}
\label{OPT_Stat_Param_Lin_Ellipt}
\subsubsection{General problem description}

In this example we consider the following benchmark problem from elasticity theory: \index{stationary PDE} \index{stationary PDE!elasticity equations}
\begin{equation}
   (\sigma(u),\varepsilon(\varphi)) = (g,\varphi)_{\Gamma_N}.
\end{equation}
Here $\tilde{\Omega}$ is a quadratic domain with side length $200$ mm, where a circular hole with radius $10$ mm around the center is cut out. Using symmetries of the domain, we restrict our actual computational domain $\Omega$ to the upper left quarter of $\tilde{\Omega}$.\\
In the above equation, $\varepsilon(v) := \frac{1}{2}(\nabla v + \nabla v^T)$ is the symmetric strain tensor, and \index{strain tensor}
\begin{equation*}
   \sigma(v) := 2\mu \varepsilon(v)^D + \kappa \cdot tr (\varepsilon(v)) I
\end{equation*}
denotes the symmetric stress tensor. Here $\tau^D$ is the deviatoric part of a tensor $\tau$, in two dimensions defined as \index{deviator}
\begin{equation*}
   \tau^D := \tau - \frac{1}{2} tr(\tau) I,
\end{equation*}
and the parameters $\mu$ and $\kappa$ are chosen as $\mu = 80193.800283$ resp. $\kappa = 190937.589172$.\\
The corner points of our computational domain are in anticlockwise order: $(0,0), (90,0)$, $(100,10), (100,100)$ and $(0,100)$. We prescribe homogeneous Dirichlet boundary conditions in $y$-direction between $(0,0)$ and $(90,0)$ (lower boundary part), homogeneous Dirichlet boundary conditions in $x$-direction between $(100,10)$ and $(100,100)$ (right boundary part), and we interpret the right hand side of equation $(1)$ with $g = 450$ as a boundary condition between $(0,100)$ and $(100,100)$ (upper boundary part).\\
The goal of our computations is to match the following functional reference values taken from \textit{E. Stein (editor), Error-controlled Adaptive Finite Elements in Solid Mechanics, Wiley (2003), pp. 386 - 387}:
\begin{table}[h]
\centering
\begin{tabular}{lccc}    
\hline
 Functional & $u_1$ at $(90,0)$ & $\sigma_{22}$ at $(90,0)$ & $u_2$ at $(100,100)$\\
\hline 
 Reference value & 0.021290 & 1388.732343 & 0.20951\\
\hline
\end{tabular}
\end{table}
\begin{table}[h]
\centering
\begin{tabular}{lcc}    
\hline
 Functional & $u_1$ at $(0,100)$ & $\int_{(100,100)}^{(0,100)} u_2$\\
\hline 
 Reference value & 0.076758 & 20.40344\\
\hline
\end{tabular}
\end{table}

\subsubsection{Program description}

From the previous examples we know how to read a grid from an \textit{.inp} file. The grid of our current example comes from the above mentioned benchmark problem. The parameter file (\textit{dope.prm}) contains the same variables as in the first two examples.\\
Apart from different point values of derivatives of the solution, we want to evaluate an integral over part of the boundary. \index{functional} \index{functional!boundary integral} This is newly implemented in \textit{functionals.h}.\\
The (linear) elasticity equation is much simpler than the FSI problems we treated before, which can be seen in \textit{localpde.h} where the \texttt{ElementEquation} and \texttt{ElementMatrix} are implemented. In principle, everything is clear from the preceding examples.\\
The \textit{main.cc} file does not contain any innovation, either. Here, we refine the grid globally instead of using an error estimator for local refinement.\\
The output of the program reflects again the linearity of the problem (only one Newton step is needed for solution).\\

\clearpage
%%%%%%%%%%%%%%%%%%%%%%%%%%%%%%%%%%%%%%%%%%%%%%%%%%
\subsection{Parameter control with a nonlinear PDE from fluid dynamcis}
\label{OPT_Stat_Param_Nonlin_Fluid}
\subsubsection{General problem description}

In this example we consider the following benchmark problem from elasticity theory: \index{stationary PDE} \index{stationary PDE!elasticity equations}
\begin{equation}
   (\sigma(u),\varepsilon(\varphi)) = (g,\varphi)_{\Gamma_N}.
\end{equation}
Here $\tilde{\Omega}$ is a quadratic domain with side length $200$ mm, where a circular hole with radius $10$ mm around the center is cut out. Using symmetries of the domain, we restrict our actual computational domain $\Omega$ to the upper left quarter of $\tilde{\Omega}$.\\
In the above equation, $\varepsilon(v) := \frac{1}{2}(\nabla v + \nabla v^T)$ is the symmetric strain tensor, and \index{strain tensor}
\begin{equation*}
   \sigma(v) := 2\mu \varepsilon(v)^D + \kappa \cdot tr (\varepsilon(v)) I
\end{equation*}
denotes the symmetric stress tensor. Here $\tau^D$ is the deviatoric part of a tensor $\tau$, in two dimensions defined as \index{deviator}
\begin{equation*}
   \tau^D := \tau - \frac{1}{2} tr(\tau) I,
\end{equation*}
and the parameters $\mu$ and $\kappa$ are chosen as $\mu = 80193.800283$ resp. $\kappa = 190937.589172$.\\
The corner points of our computational domain are in anticlockwise order: $(0,0), (90,0)$, $(100,10), (100,100)$ and $(0,100)$. We prescribe homogeneous Dirichlet boundary conditions in $y$-direction between $(0,0)$ and $(90,0)$ (lower boundary part), homogeneous Dirichlet boundary conditions in $x$-direction between $(100,10)$ and $(100,100)$ (right boundary part), and we interpret the right hand side of equation $(1)$ with $g = 450$ as a boundary condition between $(0,100)$ and $(100,100)$ (upper boundary part).\\
The goal of our computations is to match the following functional reference values taken from \textit{E. Stein (editor), Error-controlled Adaptive Finite Elements in Solid Mechanics, Wiley (2003), pp. 386 - 387}:
\begin{table}[h]
\centering
\begin{tabular}{lccc}    
\hline
 Functional & $u_1$ at $(90,0)$ & $\sigma_{22}$ at $(90,0)$ & $u_2$ at $(100,100)$\\
\hline 
 Reference value & 0.021290 & 1388.732343 & 0.20951\\
\hline
\end{tabular}
\end{table}
\begin{table}[h]
\centering
\begin{tabular}{lcc}    
\hline
 Functional & $u_1$ at $(0,100)$ & $\int_{(100,100)}^{(0,100)} u_2$\\
\hline 
 Reference value & 0.076758 & 20.40344\\
\hline
\end{tabular}
\end{table}

\subsubsection{Program description}

From the previous examples we know how to read a grid from an \textit{.inp} file. The grid of our current example comes from the above mentioned benchmark problem. The parameter file (\textit{dope.prm}) contains the same variables as in the first two examples.\\
Apart from different point values of derivatives of the solution, we want to evaluate an integral over part of the boundary. \index{functional} \index{functional!boundary integral} This is newly implemented in \textit{functionals.h}.\\
The (linear) elasticity equation is much simpler than the FSI problems we treated before, which can be seen in \textit{localpde.h} where the \texttt{ElementEquation} and \texttt{ElementMatrix} are implemented. In principle, everything is clear from the preceding examples.\\
The \textit{main.cc} file does not contain any innovation, either. Here, we refine the grid globally instead of using an error estimator for local refinement.\\
The output of the program reflects again the linearity of the problem (only one Newton step is needed for solution).\\

\clearpage
%%%%%%%%%%%%%%%%%%%%%%%%%%%%%%%%%%%%%%%%%%%%%%%%%%
\subsection{Control in the dirichlet boundary values}
\label{OPT_Stat_Dirichlet_Boundary}
\subsubsection{General problem description}

In this example we consider the following benchmark problem from elasticity theory: \index{stationary PDE} \index{stationary PDE!elasticity equations}
\begin{equation}
   (\sigma(u),\varepsilon(\varphi)) = (g,\varphi)_{\Gamma_N}.
\end{equation}
Here $\tilde{\Omega}$ is a quadratic domain with side length $200$ mm, where a circular hole with radius $10$ mm around the center is cut out. Using symmetries of the domain, we restrict our actual computational domain $\Omega$ to the upper left quarter of $\tilde{\Omega}$.\\
In the above equation, $\varepsilon(v) := \frac{1}{2}(\nabla v + \nabla v^T)$ is the symmetric strain tensor, and \index{strain tensor}
\begin{equation*}
   \sigma(v) := 2\mu \varepsilon(v)^D + \kappa \cdot tr (\varepsilon(v)) I
\end{equation*}
denotes the symmetric stress tensor. Here $\tau^D$ is the deviatoric part of a tensor $\tau$, in two dimensions defined as \index{deviator}
\begin{equation*}
   \tau^D := \tau - \frac{1}{2} tr(\tau) I,
\end{equation*}
and the parameters $\mu$ and $\kappa$ are chosen as $\mu = 80193.800283$ resp. $\kappa = 190937.589172$.\\
The corner points of our computational domain are in anticlockwise order: $(0,0), (90,0)$, $(100,10), (100,100)$ and $(0,100)$. We prescribe homogeneous Dirichlet boundary conditions in $y$-direction between $(0,0)$ and $(90,0)$ (lower boundary part), homogeneous Dirichlet boundary conditions in $x$-direction between $(100,10)$ and $(100,100)$ (right boundary part), and we interpret the right hand side of equation $(1)$ with $g = 450$ as a boundary condition between $(0,100)$ and $(100,100)$ (upper boundary part).\\
The goal of our computations is to match the following functional reference values taken from \textit{E. Stein (editor), Error-controlled Adaptive Finite Elements in Solid Mechanics, Wiley (2003), pp. 386 - 387}:
\begin{table}[h]
\centering
\begin{tabular}{lccc}    
\hline
 Functional & $u_1$ at $(90,0)$ & $\sigma_{22}$ at $(90,0)$ & $u_2$ at $(100,100)$\\
\hline 
 Reference value & 0.021290 & 1388.732343 & 0.20951\\
\hline
\end{tabular}
\end{table}
\begin{table}[h]
\centering
\begin{tabular}{lcc}    
\hline
 Functional & $u_1$ at $(0,100)$ & $\int_{(100,100)}^{(0,100)} u_2$\\
\hline 
 Reference value & 0.076758 & 20.40344\\
\hline
\end{tabular}
\end{table}

\subsubsection{Program description}

From the previous examples we know how to read a grid from an \textit{.inp} file. The grid of our current example comes from the above mentioned benchmark problem. The parameter file (\textit{dope.prm}) contains the same variables as in the first two examples.\\
Apart from different point values of derivatives of the solution, we want to evaluate an integral over part of the boundary. \index{functional} \index{functional!boundary integral} This is newly implemented in \textit{functionals.h}.\\
The (linear) elasticity equation is much simpler than the FSI problems we treated before, which can be seen in \textit{localpde.h} where the \texttt{ElementEquation} and \texttt{ElementMatrix} are implemented. In principle, everything is clear from the preceding examples.\\
The \textit{main.cc} file does not contain any innovation, either. Here, we refine the grid globally instead of using an error estimator for local refinement.\\
The output of the program reflects again the linearity of the problem (only one Newton step is needed for solution).\\

\clearpage
%%%%%%%%%%%%%%%%%%%%%%%%%%%%%%%%%%%%%%%%%%%%%%%%%%
\subsection{Distributed Control with Different Meshes for Control and State}
\label{OPT_Stat_Distributed_MultiMesh}
\subsubsection{General problem description}

In this example we consider the following benchmark problem from elasticity theory: \index{stationary PDE} \index{stationary PDE!elasticity equations}
\begin{equation}
   (\sigma(u),\varepsilon(\varphi)) = (g,\varphi)_{\Gamma_N}.
\end{equation}
Here $\tilde{\Omega}$ is a quadratic domain with side length $200$ mm, where a circular hole with radius $10$ mm around the center is cut out. Using symmetries of the domain, we restrict our actual computational domain $\Omega$ to the upper left quarter of $\tilde{\Omega}$.\\
In the above equation, $\varepsilon(v) := \frac{1}{2}(\nabla v + \nabla v^T)$ is the symmetric strain tensor, and \index{strain tensor}
\begin{equation*}
   \sigma(v) := 2\mu \varepsilon(v)^D + \kappa \cdot tr (\varepsilon(v)) I
\end{equation*}
denotes the symmetric stress tensor. Here $\tau^D$ is the deviatoric part of a tensor $\tau$, in two dimensions defined as \index{deviator}
\begin{equation*}
   \tau^D := \tau - \frac{1}{2} tr(\tau) I,
\end{equation*}
and the parameters $\mu$ and $\kappa$ are chosen as $\mu = 80193.800283$ resp. $\kappa = 190937.589172$.\\
The corner points of our computational domain are in anticlockwise order: $(0,0), (90,0)$, $(100,10), (100,100)$ and $(0,100)$. We prescribe homogeneous Dirichlet boundary conditions in $y$-direction between $(0,0)$ and $(90,0)$ (lower boundary part), homogeneous Dirichlet boundary conditions in $x$-direction between $(100,10)$ and $(100,100)$ (right boundary part), and we interpret the right hand side of equation $(1)$ with $g = 450$ as a boundary condition between $(0,100)$ and $(100,100)$ (upper boundary part).\\
The goal of our computations is to match the following functional reference values taken from \textit{E. Stein (editor), Error-controlled Adaptive Finite Elements in Solid Mechanics, Wiley (2003), pp. 386 - 387}:
\begin{table}[h]
\centering
\begin{tabular}{lccc}    
\hline
 Functional & $u_1$ at $(90,0)$ & $\sigma_{22}$ at $(90,0)$ & $u_2$ at $(100,100)$\\
\hline 
 Reference value & 0.021290 & 1388.732343 & 0.20951\\
\hline
\end{tabular}
\end{table}
\begin{table}[h]
\centering
\begin{tabular}{lcc}    
\hline
 Functional & $u_1$ at $(0,100)$ & $\int_{(100,100)}^{(0,100)} u_2$\\
\hline 
 Reference value & 0.076758 & 20.40344\\
\hline
\end{tabular}
\end{table}

\subsubsection{Program description}

From the previous examples we know how to read a grid from an \textit{.inp} file. The grid of our current example comes from the above mentioned benchmark problem. The parameter file (\textit{dope.prm}) contains the same variables as in the first two examples.\\
Apart from different point values of derivatives of the solution, we want to evaluate an integral over part of the boundary. \index{functional} \index{functional!boundary integral} This is newly implemented in \textit{functionals.h}.\\
The (linear) elasticity equation is much simpler than the FSI problems we treated before, which can be seen in \textit{localpde.h} where the \texttt{ElementEquation} and \texttt{ElementMatrix} are implemented. In principle, everything is clear from the preceding examples.\\
The \textit{main.cc} file does not contain any innovation, either. Here, we refine the grid globally instead of using an error estimator for local refinement.\\
The output of the program reflects again the linearity of the problem (only one Newton step is needed for solution).\\

\clearpage
%%%%%%%%%%%%%%%%%%%%%%%%%%%%%%%%%%%%%%%%%%%%%%%%%%
\subsection{Compliance Minimization of a variable Thickness MBB-Beam}
\label{OPT_Stat_MBB-Beam}
\subsubsection{General problem description}

In this example we consider the following benchmark problem from elasticity theory: \index{stationary PDE} \index{stationary PDE!elasticity equations}
\begin{equation}
   (\sigma(u),\varepsilon(\varphi)) = (g,\varphi)_{\Gamma_N}.
\end{equation}
Here $\tilde{\Omega}$ is a quadratic domain with side length $200$ mm, where a circular hole with radius $10$ mm around the center is cut out. Using symmetries of the domain, we restrict our actual computational domain $\Omega$ to the upper left quarter of $\tilde{\Omega}$.\\
In the above equation, $\varepsilon(v) := \frac{1}{2}(\nabla v + \nabla v^T)$ is the symmetric strain tensor, and \index{strain tensor}
\begin{equation*}
   \sigma(v) := 2\mu \varepsilon(v)^D + \kappa \cdot tr (\varepsilon(v)) I
\end{equation*}
denotes the symmetric stress tensor. Here $\tau^D$ is the deviatoric part of a tensor $\tau$, in two dimensions defined as \index{deviator}
\begin{equation*}
   \tau^D := \tau - \frac{1}{2} tr(\tau) I,
\end{equation*}
and the parameters $\mu$ and $\kappa$ are chosen as $\mu = 80193.800283$ resp. $\kappa = 190937.589172$.\\
The corner points of our computational domain are in anticlockwise order: $(0,0), (90,0)$, $(100,10), (100,100)$ and $(0,100)$. We prescribe homogeneous Dirichlet boundary conditions in $y$-direction between $(0,0)$ and $(90,0)$ (lower boundary part), homogeneous Dirichlet boundary conditions in $x$-direction between $(100,10)$ and $(100,100)$ (right boundary part), and we interpret the right hand side of equation $(1)$ with $g = 450$ as a boundary condition between $(0,100)$ and $(100,100)$ (upper boundary part).\\
The goal of our computations is to match the following functional reference values taken from \textit{E. Stein (editor), Error-controlled Adaptive Finite Elements in Solid Mechanics, Wiley (2003), pp. 386 - 387}:
\begin{table}[h]
\centering
\begin{tabular}{lccc}    
\hline
 Functional & $u_1$ at $(90,0)$ & $\sigma_{22}$ at $(90,0)$ & $u_2$ at $(100,100)$\\
\hline 
 Reference value & 0.021290 & 1388.732343 & 0.20951\\
\hline
\end{tabular}
\end{table}
\begin{table}[h]
\centering
\begin{tabular}{lcc}    
\hline
 Functional & $u_1$ at $(0,100)$ & $\int_{(100,100)}^{(0,100)} u_2$\\
\hline 
 Reference value & 0.076758 & 20.40344\\
\hline
\end{tabular}
\end{table}

\subsubsection{Program description}

From the previous examples we know how to read a grid from an \textit{.inp} file. The grid of our current example comes from the above mentioned benchmark problem. The parameter file (\textit{dope.prm}) contains the same variables as in the first two examples.\\
Apart from different point values of derivatives of the solution, we want to evaluate an integral over part of the boundary. \index{functional} \index{functional!boundary integral} This is newly implemented in \textit{functionals.h}.\\
The (linear) elasticity equation is much simpler than the FSI problems we treated before, which can be seen in \textit{localpde.h} where the \texttt{ElementEquation} and \texttt{ElementMatrix} are implemented. In principle, everything is clear from the preceding examples.\\
The \textit{main.cc} file does not contain any innovation, either. Here, we refine the grid globally instead of using an error estimator for local refinement.\\
The output of the program reflects again the linearity of the problem (only one Newton step is needed for solution).\\

\clearpage
%%%%%%%%%%%%%%%%%%%%%%%%%%%%%%%%%%%%%%%%%%%%%%%%%%
\subsection{Distributed control with a linear elliptic PDE using SNOPT}
\label{OPT_Stat_Box_controlconstraints_SNOPT}
\subsubsection{General problem description}

In this example we consider the following benchmark problem from elasticity theory: \index{stationary PDE} \index{stationary PDE!elasticity equations}
\begin{equation}
   (\sigma(u),\varepsilon(\varphi)) = (g,\varphi)_{\Gamma_N}.
\end{equation}
Here $\tilde{\Omega}$ is a quadratic domain with side length $200$ mm, where a circular hole with radius $10$ mm around the center is cut out. Using symmetries of the domain, we restrict our actual computational domain $\Omega$ to the upper left quarter of $\tilde{\Omega}$.\\
In the above equation, $\varepsilon(v) := \frac{1}{2}(\nabla v + \nabla v^T)$ is the symmetric strain tensor, and \index{strain tensor}
\begin{equation*}
   \sigma(v) := 2\mu \varepsilon(v)^D + \kappa \cdot tr (\varepsilon(v)) I
\end{equation*}
denotes the symmetric stress tensor. Here $\tau^D$ is the deviatoric part of a tensor $\tau$, in two dimensions defined as \index{deviator}
\begin{equation*}
   \tau^D := \tau - \frac{1}{2} tr(\tau) I,
\end{equation*}
and the parameters $\mu$ and $\kappa$ are chosen as $\mu = 80193.800283$ resp. $\kappa = 190937.589172$.\\
The corner points of our computational domain are in anticlockwise order: $(0,0), (90,0)$, $(100,10), (100,100)$ and $(0,100)$. We prescribe homogeneous Dirichlet boundary conditions in $y$-direction between $(0,0)$ and $(90,0)$ (lower boundary part), homogeneous Dirichlet boundary conditions in $x$-direction between $(100,10)$ and $(100,100)$ (right boundary part), and we interpret the right hand side of equation $(1)$ with $g = 450$ as a boundary condition between $(0,100)$ and $(100,100)$ (upper boundary part).\\
The goal of our computations is to match the following functional reference values taken from \textit{E. Stein (editor), Error-controlled Adaptive Finite Elements in Solid Mechanics, Wiley (2003), pp. 386 - 387}:
\begin{table}[h]
\centering
\begin{tabular}{lccc}    
\hline
 Functional & $u_1$ at $(90,0)$ & $\sigma_{22}$ at $(90,0)$ & $u_2$ at $(100,100)$\\
\hline 
 Reference value & 0.021290 & 1388.732343 & 0.20951\\
\hline
\end{tabular}
\end{table}
\begin{table}[h]
\centering
\begin{tabular}{lcc}    
\hline
 Functional & $u_1$ at $(0,100)$ & $\int_{(100,100)}^{(0,100)} u_2$\\
\hline 
 Reference value & 0.076758 & 20.40344\\
\hline
\end{tabular}
\end{table}

\subsubsection{Program description}

From the previous examples we know how to read a grid from an \textit{.inp} file. The grid of our current example comes from the above mentioned benchmark problem. The parameter file (\textit{dope.prm}) contains the same variables as in the first two examples.\\
Apart from different point values of derivatives of the solution, we want to evaluate an integral over part of the boundary. \index{functional} \index{functional!boundary integral} This is newly implemented in \textit{functionals.h}.\\
The (linear) elasticity equation is much simpler than the FSI problems we treated before, which can be seen in \textit{localpde.h} where the \texttt{ElementEquation} and \texttt{ElementMatrix} are implemented. In principle, everything is clear from the preceding examples.\\
The \textit{main.cc} file does not contain any innovation, either. Here, we refine the grid globally instead of using an error estimator for local refinement.\\
The output of the program reflects again the linearity of the problem (only one Newton step is needed for solution).\\

\clearpage
%%%%%%%%%%%%%%%%%%%%%%%%%%%%%%%%%%%%%%%%%%%%%%%%%%
\subsection{Topology optimization of an MBB-Beam using SNOPT}
\label{OPT_Stat_TopOpt_MBB_SNOPT}
\subsubsection{General problem description}

In this example we consider the following benchmark problem from elasticity theory: \index{stationary PDE} \index{stationary PDE!elasticity equations}
\begin{equation}
   (\sigma(u),\varepsilon(\varphi)) = (g,\varphi)_{\Gamma_N}.
\end{equation}
Here $\tilde{\Omega}$ is a quadratic domain with side length $200$ mm, where a circular hole with radius $10$ mm around the center is cut out. Using symmetries of the domain, we restrict our actual computational domain $\Omega$ to the upper left quarter of $\tilde{\Omega}$.\\
In the above equation, $\varepsilon(v) := \frac{1}{2}(\nabla v + \nabla v^T)$ is the symmetric strain tensor, and \index{strain tensor}
\begin{equation*}
   \sigma(v) := 2\mu \varepsilon(v)^D + \kappa \cdot tr (\varepsilon(v)) I
\end{equation*}
denotes the symmetric stress tensor. Here $\tau^D$ is the deviatoric part of a tensor $\tau$, in two dimensions defined as \index{deviator}
\begin{equation*}
   \tau^D := \tau - \frac{1}{2} tr(\tau) I,
\end{equation*}
and the parameters $\mu$ and $\kappa$ are chosen as $\mu = 80193.800283$ resp. $\kappa = 190937.589172$.\\
The corner points of our computational domain are in anticlockwise order: $(0,0), (90,0)$, $(100,10), (100,100)$ and $(0,100)$. We prescribe homogeneous Dirichlet boundary conditions in $y$-direction between $(0,0)$ and $(90,0)$ (lower boundary part), homogeneous Dirichlet boundary conditions in $x$-direction between $(100,10)$ and $(100,100)$ (right boundary part), and we interpret the right hand side of equation $(1)$ with $g = 450$ as a boundary condition between $(0,100)$ and $(100,100)$ (upper boundary part).\\
The goal of our computations is to match the following functional reference values taken from \textit{E. Stein (editor), Error-controlled Adaptive Finite Elements in Solid Mechanics, Wiley (2003), pp. 386 - 387}:
\begin{table}[h]
\centering
\begin{tabular}{lccc}    
\hline
 Functional & $u_1$ at $(90,0)$ & $\sigma_{22}$ at $(90,0)$ & $u_2$ at $(100,100)$\\
\hline 
 Reference value & 0.021290 & 1388.732343 & 0.20951\\
\hline
\end{tabular}
\end{table}
\begin{table}[h]
\centering
\begin{tabular}{lcc}    
\hline
 Functional & $u_1$ at $(0,100)$ & $\int_{(100,100)}^{(0,100)} u_2$\\
\hline 
 Reference value & 0.076758 & 20.40344\\
\hline
\end{tabular}
\end{table}

\subsubsection{Program description}

From the previous examples we know how to read a grid from an \textit{.inp} file. The grid of our current example comes from the above mentioned benchmark problem. The parameter file (\textit{dope.prm}) contains the same variables as in the first two examples.\\
Apart from different point values of derivatives of the solution, we want to evaluate an integral over part of the boundary. \index{functional} \index{functional!boundary integral} This is newly implemented in \textit{functionals.h}.\\
The (linear) elasticity equation is much simpler than the FSI problems we treated before, which can be seen in \textit{localpde.h} where the \texttt{ElementEquation} and \texttt{ElementMatrix} are implemented. In principle, everything is clear from the preceding examples.\\
The \textit{main.cc} file does not contain any innovation, either. Here, we refine the grid globally instead of using an error estimator for local refinement.\\
The output of the program reflects again the linearity of the problem (only one Newton step is needed for solution).\\

\clearpage
%%%%%%%%%%%%%%%%%%%%%%%%%%%%%%%%%%%%%%%%%%%%%%%%%%
\subsection{Parameter control with a non-linear PDE from FSI dynamics}
\label{OPT_Stat_Param_Nonlin_FSI}
%\subsubsection{General problem description}

In this example we consider the following benchmark problem from elasticity theory: \index{stationary PDE} \index{stationary PDE!elasticity equations}
\begin{equation}
   (\sigma(u),\varepsilon(\varphi)) = (g,\varphi)_{\Gamma_N}.
\end{equation}
Here $\tilde{\Omega}$ is a quadratic domain with side length $200$ mm, where a circular hole with radius $10$ mm around the center is cut out. Using symmetries of the domain, we restrict our actual computational domain $\Omega$ to the upper left quarter of $\tilde{\Omega}$.\\
In the above equation, $\varepsilon(v) := \frac{1}{2}(\nabla v + \nabla v^T)$ is the symmetric strain tensor, and \index{strain tensor}
\begin{equation*}
   \sigma(v) := 2\mu \varepsilon(v)^D + \kappa \cdot tr (\varepsilon(v)) I
\end{equation*}
denotes the symmetric stress tensor. Here $\tau^D$ is the deviatoric part of a tensor $\tau$, in two dimensions defined as \index{deviator}
\begin{equation*}
   \tau^D := \tau - \frac{1}{2} tr(\tau) I,
\end{equation*}
and the parameters $\mu$ and $\kappa$ are chosen as $\mu = 80193.800283$ resp. $\kappa = 190937.589172$.\\
The corner points of our computational domain are in anticlockwise order: $(0,0), (90,0)$, $(100,10), (100,100)$ and $(0,100)$. We prescribe homogeneous Dirichlet boundary conditions in $y$-direction between $(0,0)$ and $(90,0)$ (lower boundary part), homogeneous Dirichlet boundary conditions in $x$-direction between $(100,10)$ and $(100,100)$ (right boundary part), and we interpret the right hand side of equation $(1)$ with $g = 450$ as a boundary condition between $(0,100)$ and $(100,100)$ (upper boundary part).\\
The goal of our computations is to match the following functional reference values taken from \textit{E. Stein (editor), Error-controlled Adaptive Finite Elements in Solid Mechanics, Wiley (2003), pp. 386 - 387}:
\begin{table}[h]
\centering
\begin{tabular}{lccc}    
\hline
 Functional & $u_1$ at $(90,0)$ & $\sigma_{22}$ at $(90,0)$ & $u_2$ at $(100,100)$\\
\hline 
 Reference value & 0.021290 & 1388.732343 & 0.20951\\
\hline
\end{tabular}
\end{table}
\begin{table}[h]
\centering
\begin{tabular}{lcc}    
\hline
 Functional & $u_1$ at $(0,100)$ & $\int_{(100,100)}^{(0,100)} u_2$\\
\hline 
 Reference value & 0.076758 & 20.40344\\
\hline
\end{tabular}
\end{table}

\subsubsection{Program description}

From the previous examples we know how to read a grid from an \textit{.inp} file. The grid of our current example comes from the above mentioned benchmark problem. The parameter file (\textit{dope.prm}) contains the same variables as in the first two examples.\\
Apart from different point values of derivatives of the solution, we want to evaluate an integral over part of the boundary. \index{functional} \index{functional!boundary integral} This is newly implemented in \textit{functionals.h}.\\
The (linear) elasticity equation is much simpler than the FSI problems we treated before, which can be seen in \textit{localpde.h} where the \texttt{ElementEquation} and \texttt{ElementMatrix} are implemented. In principle, everything is clear from the preceding examples.\\
The \textit{main.cc} file does not contain any innovation, either. Here, we refine the grid globally instead of using an error estimator for local refinement.\\
The output of the program reflects again the linearity of the problem (only one Newton step is needed for solution).\\

\clearpage


%%%%%%%%%%%%%%%%%%%%%%%%%%%%%%%%%%%%%%%%%%%%%%%%%%
\chapter{Testing examples}
The DOpE testsuite consists of regression tests. They are run to compare 
the output to previous outputs. This is useful (necessary) after 
changing programming code anywhere in the library. If a test
succeeds, everything is fine in the library. If not, you should not
check in your code into DOpE. Please make sure what is going wrong and WHY!

Every command is computed via a Makefile. In the basic example 
directory

%%%%%%%%%%%%%%%%%%%%%%%%%%%%%%%%%%%%%%%%%%%%%%%%%%
\section{Where can I find the tests}
In each example directory you find a subdirectory `Test'. Herein, you find 
the parameter files for meshes (*.inp) and a param file (test.prm).
Moreover, the executable is denoted by `test.sh'. Please make sure, that 
the 
\begin{verbatim}
set never_write_list
\end{verbatim}
contains every possible output
\begin{verbatim}
Gradient;Hessian;Tangent;Residual;Update;Control;State
\end{verbatim}
That means, no solution files are written to the output.
Recall, that we are just interested in terminal output that 
is of course sufficient to verify the things.

Hence, the results directory should be empty
\begin{verbatim}
set results_dir   = ./
\end{verbatim}
The rest in the param file must be identically the same as 
in the dope.prm file in the parent directory. 

%%%%%%%%%%%%%%%%%%%%%%%%%%%%%%%%%%%%%%%%%%%%%%%%%%
\section{How to start testing?}
You start testing by typing 
\begin{verbatim}
> ./test.sh Store
\end{verbatim}
in the terminal.

After the run, you have to call 
\begin{verbatim}
> ./test.sh Test
\end{verbatim}
to compare your stored output. Of course, there should be no differences.

The useful point is now the following. After implemention of 
new pieces of code in the DOpE library or in the examples, you can
run 
\begin{verbatim}
> ./test.sh Test
\end{verbatim}
Hereby, you compare your `new' output with the previous stored output.

Attention: After changes you should NOT run again
\begin{verbatim}
> ./test.sh Store
\end{verbatim}
In that case, you overwrite your previous output.
\cleardoublepage





%%%%%%%%%%%%%%%%%%%%%%%%%%%%%%%%%%%%%%%%%%%%%%%%%%
\addcontentsline{toc}{chapter}{Index}
\printindex

%%%%%%%%%%%%%%%%%%%%%%%%%%%%%%%%%%%%%%%%%%%%%%%%%%
\end{document}
