\subsubsection{General problem description}
The setup of this example is the same as Example 6, i.e. we solve a vector-valued Laplace equation in 3d. The major difference is the use of MPI to parallelize and speedup our program.

\subsubsection{Program description}
MPI based parallelization in DOpE happens mainly behind the scenes. For simple examples like this one, there are only few changes required in the user provided source code.

The first one is to use data structures that allow for MPI based parallelization. This is not supported by the previously used dealii::Vector and dealii::SparseMatrix structures, but we have to switch to the corresponding vectors in the dealii::TrilinosWrappers namespace.

\begin{verbatim}
	using MATRIXBLOCK          = TrilinosWrappers::BlockSparseMatrix;
	using SPARSITYPATTERNBLOCK = TrilinosWrappers::BlockSparsityPattern;
	using VECTORBLOCK          = TrilinosWrappers::MPI::BlockVector;
\end{verbatim}

\begin{remark}
	\begin{verbatim} using \end{verbatim} is just the more modern version of \begin{verbatim} typedef \end{verbatim}
\end{remark}

MPI needs to be initialized, this is done via the line 
\begin{verbatim}
	dealii::Utilities::MPI::MPI_InitFinalize mpi(argc, argv);
\end{verbatim}
which is a deal.II class handling the initialization for us. This object has to be created before any call to a MPI function, hence it is advised to put it right at the beginning of the main function. MPI also needs to be "finalized", i.e. closed explicitly. This is included in the previous class and happens when the MPI_InitFinalize object gets out of scope.

% TODO pcout

This are already all the changes required to run our application in parallel. In order to run the application with MPI, we have to call it via 
\begin{verbatim}
	mpirun -np 4 application
\end{verbatim}
This example call spawns 4 instances of our application, which then proceed to jointly solve our problem. The number of cores used can be controlled via the "-np x" option. "-np 1" runs the application without MPI parallelization.

\begin{remark}
	Depending on your system, the exact command might be different, i.e. "srun -n 4 application" for slurm based systems.
\end{remark}

\begin{remark}
	Note that on standard personal computers you may not actually see any speedup. This is due to limitations of the memory bandwidth. 
\end{remark}