\subsubsection{General problem description}
\index{stationary PDE}\index{stationary PDE!FSI problem} \index{fluid} \index{fluid!incompressible} \index{fluid!Newtonian}
In this example we consider a simple stationary FSI problem. The fluid
is given as an incompressible Newtonian fluid modeled by the Stokes equation. Here, we use the symmetric stress tensor which has a little consequence when using the do-nothing outflow condition, see also section \ref{PDE_Stat_Stokes}.
The computation domain is $\Omega = [-6,6]\times [0,2]$ and we choose for simplicity $f=0$. 

The fluid reads: \index{fluid!Eulerian framework}
\begin{Problem}[Variational fluid problem, Eulerian framework]
 Find $\{v_f,p_f\} \in \{ v_f^D + V\} \times L_f$, such that,
  \begin{equation*}
    \begin{aligned} 
  %    (\rho_f v\cdot\nabla v_f,\phi^v)_{\Omega_f} +
      (\sigma_f,\nabla\phi^v)_{\Omega_f} 
      &= \langle n_f\cdot g_s^\sigma ,\phi^v \rangle_{\Gamma_i}
      &&\forall \phi^v\in V_f, \\
      %%%
      (\operatorname{div}\, v_f,\phi^p)_{\Omega_f} &= 0
      &&\forall\phi^p\in L_f.
    \end{aligned}
  \end{equation*}
  The Cauchy stress tensor $\sigma_f$ is given by \index{Cauchy stress tensor}
  \begin{eqnarray}
    \sigma_f:=-p_f I + \rho_f\nu_f (\nabla v_f+\nabla v_f^T),
  \end{eqnarray}
  with the fluid's density $\rho_f$  and the kinematic viscosity
  $\nu_f$. By $n_f$ we denote the outer normal vector on $\Gamma_i$
  and by $g^\sigma_f$ is a function which describes forces acting on the interface. These
  will be specified in the context of fluid-structure interaction
  models. \index{fluid-structure interaction (FSI)} \index{fluid-structure interaction (FSI)!interface}
\end{Problem}
We define:
\begin{equation*}
\hat T:=\text{id}+\hat u,\quad
\hat F:=I+\hat\nabla \hat u,\quad
\hat J:=\text{det}(I+\hat\nabla \hat u).
\end{equation*}


The structure equations are given by incompressible neo-Hookean material
\index{structure} \index{structure!incompressible neo-Hookean (INH) material} \index{structure!Lagrangian framework}
\begin{Problem}[Incompressible neo-Hookean Model (Lagrangian)]
  \begin{equation*}
    \begin{aligned}
      (\hat J_s \hat\sigma_s \hat F_s^{-T},\hat\nabla\hat\phi^v)_{\hat\Omega_s}
      &= \langle\hat J_s\hat n_s\cdot \hat g_s^\sigma\hat
      F_s^{-T},\hat\phi^v  \rangle_{\hat\Gamma_i}
      &&\quad\forall\hat \phi^v\in\hat V_s\\
      %%%%%%%%%%%%%%%% 
      (\hat v_s,\hat\phi^u)_{\hat\Omega_s}&=0
      &&\quad\forall\hat \phi^u\in\hat V_s    ,\\
      (\hat J-1,\hat\phi^p)_{\hat\Omega_s} &=0
      &&\quad\forall\hat \phi^p\in\hat L_s,\\
    \end{aligned}
  \end{equation*}
  where $\rho_s$ is the solid's density, $\mu_s$ the Lam\'e
  coefficient, $\hat n_s$ the outer normal vector at $\hat\Gamma_i$, 
  $\hat g_s^\sigma$ the force on the interface and with  
  \[
  \hat\sigma_s:=-\hat p_sI+\mu_s(\hat F_s\hat F_s^T-I).
  \]
\end{Problem}


The resulting FSI problem is then given by: \index{fluid-structure interaction (FSI)!ALE model}
\begin{Problem}[Stationary Fluid-Structure Interaction (ALE)]
  \begin{equation*}
    \begin{aligned}
%      (\hat J \rho_f \hat F^{-1}\hat  v\cdot\hat\nabla \hat v,
%      \hat\phi^v)_{\hat\Omega_f}
      (\hat J\hat\sigma_f\hat F^{-T},\hat \nabla\hat\phi^v)_{\hat\Omega_f}
      + (\hat J\hat\sigma_s\hat F^{-T},\hat \nabla\hat\phi^v)_{\hat\Omega_s}
      &= 0&&\forall\hat\phi^v\in \hat V,\\
      %%%%%%%%%%%% 
      (\hat v,\hat\phi^u)_{\hat\Omega_s}
      + (\alpha_u \hat \nabla \hat u,\hat \nabla\hat\phi^u)_{\hat\Omega_f}
      &=0&&\forall\hat\phi^u\in \hat V,\\
    %%%%%%%%%%%%%%%%%%%%%%%%%%%%%%
      (\widehat{\text{div}}\,(\hat J\hat F^{-1}\hat
      v_f),\hat\phi^p)_{\hat\Omega_f} 
      + (\hat J - 1,\hat \phi^p)_{\hat\Omega_s}
      &=0&&\forall\hat \phi^p\in \hat L,
    \end{aligned}
  \end{equation*}  
\end{Problem}

\begin{remark}
In the problems above and the code, we implement the term
\[
(\hat v_s, \hat\phi^u),
\]
although this is not physically necessary. It is first for computational
convenience in order to extend the fluid velocity variable to the whole
domain.
This could be resolved by using the FE Nothing element. Second, 
using $\hat v_s$ here makes it easier to understand the 
nonstationary FSI problem.
\end{remark}

\subsubsection{Program description}

%%% Unsinn! Das example hat nicht viel mit dem davor zu tun!
%There is not much difference to the prior example, so we will keep our remarks rather short. The \textit{functionals.h} and \textit{myfunctions.cc} files did not change at all, so we simply refer to the corresponding sections in the last example.\\
In the \textit{localpde.h} file, all functions of the \texttt{LocalPDE} class have to be adjusted to the current FSI problem. This only makes the equations and matrices a little more complicated, and our solution vector now consists of five components (two velocity components of the fluid, the pressure component, and two additional displacement components for the structure variables). Otherwise, everything is analogous to the former example.\\
In the \textit{main.cc} we only have to add two components to the \texttt{compmask} vector and prescribe boundary conditions for the structure variables. Apart from that, we define objects for the same classes as before that are even named equally and use the same solvers.\\ 
Again, the solution is reached within one Newton step, and all we see from the program output is the values of the functionals.
