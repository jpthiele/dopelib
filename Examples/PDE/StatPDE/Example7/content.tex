\subsubsection{General problem description}

In this example we consider the following benchmark problem from elasticity theory: \index{stationary PDE} \index{stationary PDE!elasticity equations}
\begin{equation}
   (\sigma(u),\varepsilon(\varphi)) = (g,\varphi)_{\Gamma_N}.
\end{equation}
Here $\tilde{\Omega}$ is a quadratic domain with side length $200$ mm, where a circular hole with radius $10$ mm around the center is cut out. Using symmetries of the domain, we restrict our actual computational domain $\Omega$ to the upper left quarter of $\tilde{\Omega}$.\\
In the above equation, $\varepsilon(v) := \frac{1}{2}(\nabla v + \nabla v^T)$ is the symmetric strain tensor, and \index{strain tensor}
\begin{equation*}
   \sigma(v) := 2\mu \varepsilon(v)^D + \kappa \cdot tr (\varepsilon(v)) I
\end{equation*}
denotes the symmetric stress tensor. Here $\tau^D$ is the deviatoric part of a tensor $\tau$, in two dimensions defined as \index{deviator}
\begin{equation*}
   \tau^D := \tau - \frac{1}{2} tr(\tau) I,
\end{equation*}
and the parameters $\mu$ and $\kappa$ are chosen as $\mu = 80193.800283$ resp. $\kappa = 190937.589172$.\\
The corner points of our computational domain are in anticlockwise order: $(0,0), (90,0)$, $(100,10), (100,100)$ and $(0,100)$. We prescribe homogeneous Dirichlet boundary conditions in $y$-direction between $(0,0)$ and $(90,0)$ (lower boundary part), homogeneous Dirichlet boundary conditions in $x$-direction between $(100,10)$ and $(100,100)$ (right boundary part), and we interpret the right hand side of equation $(1)$ with $g = 450$ as a boundary condition between $(0,100)$ and $(100,100)$ (upper boundary part).\\
The goal of our computations is to match the following functional reference values taken from \textit{E. Stein (editor), Error-controlled Adaptive Finite Elements in Solid Mechanics, Wiley (2003), pp. 386 - 387}:
\begin{table}[h]
\centering
\begin{tabular}{lccc}    
\hline
 Functional & $u_1$ at $(90,0)$ & $\sigma_{22}$ at $(90,0)$ & $u_2$ at $(100,100)$\\
\hline 
 Reference value & 0.021290 & 1388.732343 & 0.20951\\
\hline
\end{tabular}
\end{table}
\begin{table}[h]
\centering
\begin{tabular}{lcc}    
\hline
 Functional & $u_1$ at $(0,100)$ & $\int_{(100,100)}^{(0,100)} u_2$\\
\hline 
 Reference value & 0.076758 & 20.40344\\
\hline
\end{tabular}
\end{table}

\subsubsection{Program description}

From the previous examples we know how to read a grid from an \textit{.inp} file. The grid of our current example comes from the above mentioned benchmark problem. The parameter file (\textit{dope.prm}) contains the same variables as in the first two examples.\\
Apart from different point values of derivatives of the solution, we want to evaluate an integral over part of the boundary. \index{functional} \index{functional!boundary integral} This is newly implemented in \textit{functionals.h}.\\
The (linear) elasticity equation is much simpler than the FSI problems we treated before, which can be seen in \textit{localpde.h} where the \texttt{ElementEquation} and \texttt{ElementMatrix} are implemented. In principle, everything is clear from the preceding examples.\\
The \textit{main.cc} file does not contain any innovation, either. Here, we refine the grid globally instead of using an error estimator for local refinement.\\
The output of the program reflects again the linearity of the problem (only one Newton step is needed for solution).\\
