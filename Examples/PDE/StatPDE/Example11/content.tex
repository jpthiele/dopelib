\subsubsection{General problem description}
This example shows the use higher order mappings in \texttt{DOpElib}. To this end, we solve a simple Laplace equation
\begin{align*}
-\Delta u &= -4 \quad\text{ in } \Omega
\end{align*}
with the analytical solution
\begin{align*}
u = {x^2 + y^2},
\end{align*}
and the Dirichlet Conditions on $u=1$, where the domain is given by a circle with radius 1 and the center located in the origin.

We compute the $L^2$-norm of the error and, additionally, we evaluate a functional which does not depend on the solution at all.
We integrate the constant $\frac 12$ once over the boundary of the domain. The result is an approximation of $\pi$.

All this is standard and would not justify an additional example, however we solve the equation and the functional not one but two
times. First with the standard Q1-mapping, the second time we use a higher order mapping. The exact order can be determined by the 
parameter file, the preset is Q2-mapping. At the end, we gather the errors and convergence rates over some refinement cycles
in a nice table and notice the higher order of convergence for the higher order mapping solutions. This is due to the fact that
we can approximate the circular boundary much better by a quadratic mapping.
\subsubsection{Program description}
In this section we want to focus on what you have to do if you want to enhance your existing code to use higher order mappings,
which is actually pretty simple.

You have to include the file
\begin{verbatim}
mapping_wrapper.h
\end{verbatim}

and create a mapping of the desired order by
\begin{verbatim}
  DOpEWrapper::Mapping<dim, DOFHANDLER> mapping(order_of_mapping);
\end{verbatim}

The last step is to give the mapping to the DoFHandler:
\begin{verbatim}
MethodOfLines_StateSpaceTimeHandler<FE, DOFHANDLER, SPARSITYPATTERN,VECTOR, 2> 
	  DOFH(triangulation, mapping, state_fe);
\end{verbatim}
The rest of the program is as usual.
