\subsubsection{General problem description}
 In this example we show how we use the FEInterpolatedValues class in \texttt{DOpElib}.
 We consider a small variation to the nearly incompressible linear elasticity problem given 
 in ~\cite{Linke:2014}. The displacement and pressure are given as follows
 \begin{align*}
   \textbf{u}(x, y) & = \begin{bmatrix}
 		    200 x^2 (1-x)^2y(1-y)(1-2y) \\
 		    -200 y^2(10y)^2x(1-x)(1-2x)
 	  	   \end{bmatrix} \\
   p(x,y) & = 10\left(x - \frac{1}{2}\right)^3y^2 + (1-x)^3\left(y-\frac{1}{2}\right)^3 + \frac{1}{8}.
 \end{align*}
 respectively for the incompressible linear elasticity equation
 \begin{equation}
   \begin{aligned}[b] \label{eq:strong_form}
     -2\mu \nabla \! \cdot \! \varepsilon(\textbf{u}) + \nabla p = f, \\
     \nabla \cdot \textbf{u} + \frac{1}{\lambda} = 0,
   \end{aligned}
 \end{equation}
 where $\lambda \neq \infty$.
 
 The modified weak form of \eqref{eq:strong_form} is given as
  \begin{align}
   2\mu \int\limits_\Omega \varepsilon(\textbf{u}_h) : \varepsilon(\textbf{v}_h) \; \mathrm{d} x - 
   \int\limits_\Omega p_h \nabla \cdot \textbf{v}_h \; \mathrm{d} x & = \int\limits_\Omega \textbf{f} \cdot  \bm{\pi}^{\rm{div}} \textbf{v}_h \; \mathrm{d} x  \label{eq:modified_weak_form} , \\
   \int\limits_\Omega q_h \nabla \cdot \textbf{u}_h \; \mathrm{d} x  + \dfrac{1}{\lambda}\int\limits_\Omega p_hq_h \; \mathrm{d} x  & = 0  \label{eq:nearly_incompress_modified}
 \end{align}
 where, $\bm{\pi}^{\rm{div}}$ is the interpolation operator that maps the displacement finite element space 
 to $H^{\rm{div}}$ conforming space.

 Most things are identical to all the other programs. However, we discuss the minor changes needed here. 
 Firstly, we need to include the \texttt{interpolatedintegratordatacontainer.h} instead of \texttt{integratordatacontainer.h}.
 Secondly, we define \texttt{IDC idc(velocity\_component, map, fe\_interpolate, quadrature\_formula, face\_quadrature\_formula);}
 according to the syntax of interpolatedintegratordatacontainer. \par
 Now, we can move on to \texttt{localpde.h}. Since we need the interpolated values on the right hand
 side of the equation, we only need to change in \texttt{ElementRightHandSide} class. We use \texttt{InterpEDC} instead of 
 \texttt{EDC}. In addition to initializing the \texttt{state\_fe\_values}, we also initialize \texttt{fe\_values\_interpolated} as \\
 \texttt{InterpolatedFEValues<dealdim> fe\_values\_interpolated = edc.GetInterpolatedFEValues();} 
 When adding the \texttt{local\_vector}, we use \\ \texttt{fe\_values\_interpolated.value(i, q\_point)} instead of 
 \texttt{state\_fe\_values[disp].value(i, q\_point)}.

   


