\subsubsection{General problem description}
 In this example we show how to implement pressure robust finite elements
 as proposed by~\cite{Linke:2014} using the
 FEInterpolatedValues class in \texttt{DOpElib}.
 For this, we consider a small variation to the incompressible
 Stokes problem in~\cite{Linke:2014} towards quasi incompressible
 linear elasticity. The displacement and pressure are given as follows
 \begin{align*}
   \textbf{u}(x, y) & = \begin{bmatrix}
 		    200 x^2 (1-x)^2y(1-y)(1-2y) \\
 		    -200 y^2(10y)^2x(1-x)(1-2x)
 	  	   \end{bmatrix} \\
   p(x,y) & = -10\left(x - \frac{1}{2}\right)^3y^2 + (1-x)^3\left(y-\frac{1}{2}\right)^3 - \frac{1}{8}.
 \end{align*}
 in the incompressible limit $\lambda \rightarrow \infty$ of the 
 the quasi incompressible linear elasticity equation
 \begin{equation}
   \begin{aligned}[b] \label{eq:strong_form}
     -2\mu \nabla \! \cdot \! \varepsilon(\textbf{u}) - \nabla p = f, \\
     \nabla \cdot \textbf{u} - \frac{1}{\lambda} p = 0,
   \end{aligned}
 \end{equation}
 for given parameters $\lambda, \mu > 0$ and $\lambda$ ``large''.

 For the discretization we search $(\textbf{u}_h,p_h) \in V_h \times Q_h$,
 here given by continuous $Q_2$ finite elements for $V_h$ and discontinuous
 $P_1$ finite elements for $Q_h$.
 The gradient robust discretization of \eqref{eq:strong_form} is given as
  \begin{align}
   2\mu \int\limits_\Omega \varepsilon(\textbf{u}_h) : \varepsilon(\textbf{v}_h) \; \mathrm{d} x + 
   \int\limits_\Omega p_h \nabla \cdot \textbf{v}_h \; \mathrm{d} x & = \int\limits_\Omega \textbf{f} \cdot  \bm{\pi}^{\rm{div}} \textbf{v}_h \; \mathrm{d} x  \label{eq:modified_weak_form} , \\
   \int\limits_\Omega q_h \nabla \cdot \textbf{u}_h \; \mathrm{d} x  - \dfrac{1}{\lambda}\int\limits_\Omega p_hq_h \; \mathrm{d} x  & = 0  \label{eq:nearly_incompress_modified}
 \end{align}
 for all $\textbf{v}_h \in V_h$, $q_h \in Q_h$
 where, $\bm{\pi}^{\rm{div}} \colon V_h\rightarrow X_h \subset H^{\rm{div}}$
 is a suitable interpolation operator satisfying
 \[
 (q_h, \nabla \cdot \textbf{v}_h) = (q_h \nabla \cdot \bm{\pi}^{\rm{div}}\textbf{v}_h) 
 \]
 for all $q_h \in Q_h$ and $\textbf{v}_h \in V_h$. For the choice of $V_h,Q_h$ made here a
 suitable choice for this space $X_h$ is the space of Brezzi-Douglas-Marini $\mathcal{BDM}_2$ elements
 with the canonical interpolation.

 In contrast to standard mixed discretizations as discussed, e.g. in Section~\ref{PDE_Stat_Stokes} only some minor changes are needed which we discuss below. 

In \textit{main.cc}, we need to utilize modified versions of the ElementDataContainer,
FaceDataContainer and IntegratorContainer that give access to the values of
$\bm{\pi}^{\rm{div}}\textbf{v}_h$ in the local integrals for the righthandside.  
These classes are provided by the files
\texttt{interpolatedintegratordatacontainer.h}, 
\texttt{interpolatedelementdatacontainer.h} and \texttt{interpolatedfacedatacontainer.h}
For the instantiation, we only need to take care of the InterpolatedIntegratorDataContainer
via
\texttt{IDC idc(velocity\_component, map, fe\_interpolate, quadrature\_formula, face\_quadrature\_formula);}
here the first three arguments are new and needed for the interpolation. The \texttt{velocity\_component} is a \texttt{FEValuesExtractpor::Vector} that indicates which components of the finite
element $V_h\times Q_h$ should be interpolated onto $X_h$ consisting of the element
\texttt{fe\_interpolate} here initialized as \texttt{RaviartThomasNodal<2>} of degree $1$.
The \texttt{map} indicates the Mapping to be used for the transformation of the reference
RaviartThomas element to the real element.

In \texttt{localpde.h}, the only substantial change occurs in the
\texttt{ElementRightHandSide} where we need to multiply $f$ with
$\bm{\pi}^{\rm{div}} \textbf{v}_h$.
Here we utilize that the \texttt{InterpolatedElementDataContainer} gives access to $\bm{\pi}^{\rm{div}} \textbf{v}_h$ by the method \texttt{GetInterpolatedFEValues()}.
More precisely, initializing 
\begin{verbatim}
        InterpolatedFEValues<dealdim> fe_values_interpolated 
                                      = edc.GetInterpolatedFEValues();   
\end{verbatim}                
allows to access $\bm{\pi}^{\rm{div}} \textbf{v}_h$ transparently via
\begin{verbatim}
        fe_values_interpolated.value(i, q_point)
\end{verbatim}
whereas the choice 
\begin{verbatim}
        FEValues<dealdim> fe_values = edc.GetFEValuesState(); 
        const FEValuesExtractors::Vector disp(0);
\end{verbatim}                
allows to access $\textbf{v}_h$ via
\begin{verbatim}
        fe_values[disp].value(i, q_point)
\end{verbatim}

   


