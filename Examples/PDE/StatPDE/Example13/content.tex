\subsubsection{General problem description}
Within this example, we demonstrate how to use the
dG (discontinuous Galerkin) method for the solution of 
a transport equation. The example corresponds to 
the dealii step-12. Here we want to solve the transport equation 
\begin{align*}
 \nabla \cdot (\beta u) &= 0 & \text{in }&\Omega,\\
 u &= g &\text{on }&\Gamma_- := \{ x \in \partial \Omega\,|\, \beta(x) \cdot n(x) < 0\}.
\end{align*}
Where $n$ is the outward unit normal, $\Omega = (0,1)^2$ and 
\[
\beta(x) = \frac{1}{|x|} \begin{pmatrix} -x_2\\x_1
  \end{pmatrix}.
\]

For the numerical solution, as in dealii step~12, we use the upwind discontinuous 
Galerkin \index{discontinuous Galerkin}. Hence we solve the problem of finding $u_h$ 
such that 
\[
 -\sum_{T \in \mathcal T_h} (u_h \beta \cdot  \nabla v_h)_T 
 + \sum_{F \in \mathcal F_h} (u_h^-,[\beta \cdot n v_h])_F + (u_h,\beta \cdot n v_h)_{\Gamma_+} = -(g,\beta \cdot n v_h)_{\Gamma_-}
\]
where $\Gamma_+ =  \{ x \in \partial \Omega\,|\, \beta(x) \cdot n(x) > 0\}$, $\mathcal T_h$ and $\mathcal F_h$ denote the 
elements and interior faces of the mesh, respectively. The jump is defined as 
\[
[\beta \cdot n v_h] = (v^+ - v^-) \beta \cdot n^+
\]
where the superscript $+$ or $-$ denotes the dependence on the upstream $+$ or downstream $-$ element.

\subsubsection{Implementational Details}
Within this program, we need to make use of the additional \texttt{Face*} and \texttt{Interface*} methods as given in the  
\texttt{PDEInterface} class. The Face* methods define all integrals on $\mathcal F$ in which the element interacts 
with it self. The Interface* methods are used for the coupling between the two neighboring elements over the 
given face.

The program requires the following changes in contrast to the prior examples:

\paragraph{main.cc} We utilize Block Preconditioners for the solution of the resulting system. 
To this end we included the line
\begin{verbatim}
typedef DOpEWrapper::PreconditionBlockSSOR_Wrapper<MATRIX,4> 
        PRECONDITIONERSSOR; 
\end{verbatim}
In contrast to all other preconditioners, we need to specify the block size. This number needs 
to correspond to the number of unknowns per elements; here 4 since we use Q1-elements. Note that this works for dG
elements only.

The next important change is that we now not only use a discontinuous element, but we will need to assemble 
terms on faces between elements that couple the unknowns in the different elements together. For this,
the matrix needs to have the corresponding non-zero entries specified in the sparsity pattern.  
To do this, the space time handler has an argument \texttt{bool flux\_pattern} in the constructor that needs to be set to true, i.e., we instantiate as follows: 
\begin{verbatim}
STH DOFH(triangulation, state_fe, true);
\end{verbatim}

Finally, the integration will utilize a special function to be declared in the \texttt{LocalPDE}, 
hence all objects must use the \texttt{LocalPDE}
and not the \texttt{PDEInterface}.
To make sure that this is the case, the \texttt{PDEProblemContainer} needs to be initialized with the following arguments
\begin{verbatim}
typedef PDEProblemContainer<LocalPDE<EDC, FDC, DOFHANDLER, VECTOR, DIM>,
    SimpleDirichletData<VECTOR, DIM>, SPARSITYPATTERN, VECTOR, DIM> OP;
\end{verbatim}

\paragraph{localpde.h}
In order to integrate the PDE above, we have to deal with one term that has not been considered before
\[
 \sum_{F \in \mathcal F_h} (u_h^-,[\beta \cdot n v_h])_F 
\]
Since internally all terms by sums over elements we split this term into contributions on the 
element edges $\partial K$. On an element $K$ a face $F$, with outward normal $n$ 
connects to another element $K'$, depending on the sign of $\beta \cdot n$ we have two cases. 
$\beta \cdot n > 0$ in which case $u^-_h = u_h$ or $\beta \cdot n < 0$ in which case $u^-_h = u_h^* = u_h\lvert_{K'}$, i.e.
the value from the neighbor. The jump always contains the values $v_h$ and $v_h^*$. 

Let now $\beta \cdot n > 0$. Then we assemble the contributions coming only from this element in the \texttt{FaceEquation} 
(and \texttt{FaceMatrix}),
i.e.,
\[
 (u_h,\beta \cdot n v_h)_F
\]
The other part of the jump, namely 
\[
 -(u_h,\beta \cdot n v_h^*)_F
\]
is not assembled here, since the test functions do not live on the selected element $K$. These contributions will
be assembled once the element $K'$ is selected (and hence on the same face $\beta \cdot n < 0$.
Once this is the case, i.e. $\beta \cdot n < 0$, we assemble the other part of the jump,
which is now 
\[
(u_h^*,\beta \cdot n v_h)_F.
\]
This is done in the \texttt{InterfaceEquation} (and \texttt{InterfaceMatrix}) since we couple unknowns for the neighboring element $K^*$ (the values of $u_h^*$) with those on $K$ (the values of $v_h$). Note that in contrast to the view on the element with 
$\beta \cdot n > 0$ the term has apparently switched signs. This is no typo, but due to the fact, that the outward normal 
has changed direction.

The precise assembly is analogous to the usual integrals, hence we don't provide more details. The only 
thing different in the \texttt{InterfaceEquation} and \texttt{InterfaceMatrix} we need to access the 
values on the element on the other side of the face. To this end, all \texttt{Get*} functions used, 
such as \texttt{GetFEFaceValuesState},
have a counterpart \texttt{GetNbr*}, i.e., \texttt{GetNbrFEFaceValuesState}, to access the corresponding values 
on the neighboring element.

Naturally the two functions
\begin{verbatim}
bool HasFaces() const;
bool HasInterfaces() const;
\end{verbatim}
need to return \texttt{true}.

A last and important change is that we now need to implement the 
method 
\begin{verbatim}
template<typename ELEMENTITERATOR>
 bool AtInterface(ELEMENTITERATOR& element, unsigned int face) const
 {
   if (element[0]->neighbor_index(face) != -1) 
     return true;
   return false;
 }
\end{verbatim}
that returns true whenever we are on an interior face and false otherwise.

