\subsubsection{General problem description}

The problem under consideration is the so called multivariate Black-Scholes equation arising from pricing European style options in finance. \index{instationary PDE} \index{instationary PDE!Black-Scholes equation}\\
To state the general form of the equation we need some nomenclature: We consider an option on $d$ risky assets with {\it maturity} $T>0$ and {\it strikeprice} $K>0$. For the sake of simplicity we assume the {\it interest rate} $r>0$ and the {\it volatility} of the i-th asset $\sigma_i>0$, $1\leq i\leq d$, to be constant. Besides, we assume the matrix $\rho=(\rho_{ij})$ of the {\it correlation factors} $\rho_{ij}$ with  $-1\leq \rho_{ij}\leq 1$ for  $1\leq i,j \leq d$, to be positive definite. Of course $\rho$ is symmetric with $\rho_{ii}=1$.\\ 
With $(t,x) \in I=(0,T] \times \mathbb{R}^d_+$ denoting the prices of the underlying assets at time $t$, the problem of determining the fair price $u$ of such an option is (after a time reversal) given by the following equation:
\begin{subequations}
\begin{align} 
   \partial_t{u}-\frac12\sum_{i,j=1}^d \sigma_i\sigma_j\rho_{ij}x_ix_j \partial_{{x_i}}\partial_{x_j}u-r\sum_{i=1}^dx_i\partial_{x_i}{u}+ru&=0 &&\mbox{in } (0,T]\times \mathbb{R}_+^d,\\
   u(0)&=u_0 &&\mbox{in } \mathbb{R}_+^d.
\end{align}
\end{subequations}
The initial condition $u_0\in C^{0}\left({\mathbb{R}_+^d}\right) $ (i.e. the {\it payoff}) is given depending of the type of the option. For example
\begin{equation}
u_0 :=\begin{cases} 
\max(\sum_{i=1}^d \lambda_i x_i-K,0),&  u\text{ is a \textit{Call},}\\
\max(K-\sum_{i=1}^d\lambda_i x_i,0),& u\text{ is a \textit{Put},}
\end{cases}
\end{equation}  
for a plain vanilla European option on a basket of assets containing a share of $0<\lambda_i\leq 1$ of the $i$-th asset. For the computation, we truncate the domain, i.e. we choose $\overline x \in \mathbb{R}^d_+$ and consider the computational domain $\Omega := (x_1, \overline x_1)\times \dots \times (x_d, \overline x_d)$. On the new part of the boundary $\Gamma$ with $\Gamma :=\{x\in \partial\Omega |\exists_{1\leq i\leq d}  x_i=\overline x_i\}$) we impose asymptotic values as dirichlet conditions. For a put, we take $u|_\Gamma=0$. We emphasize that no boundary conditions will be imposed on $\partial \Omega \setminus \Gamma$.\\
In this particular example we examine the case of two uncorrelated stocks (with $\lambda_1=\lambda_2=\frac12$) and the following parameters:
 \begin{table}[hb]
    \centering
    \begin{tabular}{lr}
      \toprule 
 & 2d-Put)\\
       \cmidrule(l){2-2}
	 actual asset value $x_0$ 	&  (25,25)\\
	 strikeprice $K$			&  25 \\
	 maturity date $T$ 			& 1 \\
	 volatility $\sigma$		&  ($\frac12$, $\frac 3{10}$) \\
	 cutoff $\overline x$		& $(100,100)$\\
	 interest rate $r$ 				&  $0{,}05$\\
	 option value $u(T,x_0$)			& ca.  2{,}269172389 \\
      \bottomrule
    \end{tabular}
  \end{table}

%\subsubsection{Program description}
