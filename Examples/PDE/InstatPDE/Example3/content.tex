\subsubsection{General problem description}

The problem under consideration is the so called multivariate Black-Scholes equation arising from pricing European style options in finance. \index{instationary PDE} \index{instationary PDE!Black-Scholes equation}\\
To state the general form of the equation we need some nomenclature: We consider an option on $d$ risky assets with {\it maturity} $T>0$ and {\it strikeprice} $K>0$. For the sake of simplicity we assume the {\it interest rate} $r>0$ and the {\it volatility} of the i-th asset $\sigma_i>0$, $1\leq i\leq d$, to be constant. Besides, we assume the matrix $\rho=(\rho_{ij})$ of the {\it correlation factors} $\rho_{ij}$ with  $-1\leq \rho_{ij}\leq 1$ for  $1\leq i,j \leq d$, to be positive definite. Of course $\rho$ is symmetric with $\rho_{ii}=1$.\\ 
With $(t,x) \in I=(0,T] \times \mathbb{R}^d_+$ denoting the prices of the underlying assets at time $t$, the problem of determining the fair price $u$ of such an option is (after a time reversal) given by the following equation:
\begin{subequations}
\begin{align} 
   \partial_t{u}-\frac12\sum_{i,j=1}^d \sigma_i\sigma_j\rho_{ij}x_ix_j \partial_{{x_i}}\partial_{x_j}u-r\sum_{i=1}^dx_i\partial_{x_i}{u}+ru&=0 &&\mbox{in } (0,T]\times \mathbb{R}_+^d,\\
   u(0)&=u_0 &&\mbox{in } \mathbb{R}_+^d.
\end{align}
\end{subequations}
The initial condition $u_0\in C^{0}\left({\mathbb{R}_+^d}\right) $ (i.e. the {\it payoff}) is given depending of the type of the option. For example
\begin{equation}
u_0 :=\begin{cases} 
\max(\sum_{i=1}^d \lambda_i x_i-K,0),&  u\text{ is a \textit{Call},}\\
\max(K-\sum_{i=1}^d\lambda_i x_i,0),& u\text{ is a \textit{Put},}
\end{cases}
\end{equation}  
for a plain vanilla European option on a basket of assets containing a share of $0<\lambda_i\leq 1$ of the $i$-th asset. For the computation, we truncate the domain, i.e. we choose $\overline x \in \mathbb{R}^d_+$ and consider the computational domain $\Omega := (x_1, \overline x_1)\times \dots \times (x_d, \overline x_d)$. On the new part of the boundary $\Gamma$ with $\Gamma :=\{x\in \partial\Omega |\exists_{1\leq i\leq d}  x_i=\overline x_i\}$) we impose asymptotic values as Dirichlet conditions. For a put, we take $u|_\Gamma=0$. We emphasize that no boundary conditions will be imposed on $\partial \Omega \setminus \Gamma$.\\
In this particular example we examine the case of two uncorrelated stocks (with $\lambda_1=\lambda_2=\frac12$) and the following parameters:
 \begin{table}[hb]
    \centering
    \begin{tabular}{lr}
      \toprule 
 & 2d-Put)\\
       \cmidrule(l){2-2}
	 actual asset value $x_0$ 	&  (25,25)\\
	 strikeprice $K$			&  25 \\
	 maturity date $T$ 			& 1 \\
	 volatility $\sigma$		&  ($\frac12$, $\frac 3{10}$) \\
	 cutoff $\overline x$		& $(100,100)$\\
	 interest rate $r$ 				&  $0{,}05$\\
	 option value $u(T,x_0$)			& ca.  2{,}269172389 \\
      \bottomrule
    \end{tabular}
  \end{table}

\subsubsection{Program description}
Note that the initial conditions are only $H^1$-regular. Because of this, 
the Crank-Nicolson scheme, which is not strongly A-stable, is not suited to solve 
this problem. As we want to use a second order accurate time stepping scheme, we use 
the shifted Crank-Nicolson method. The rest of the program is as before.

In addition, this program shows how to change the vector behavior, from our
default option \texttt{fullmem}, where the whole vector
is stored in the computers main memory. Here the storage behavior is set to 
\texttt{store\underline{ }on\underline{ }disc}, where only those unknowns are loaded into 
main memory that are needed in the current time step, while all other 
unknowns are stored on the hard drive. 

This behavior is particularly useful, if many time steps are taken; so that the 
whole set of unknowns can no longer be stored in main memory. Note that all vectors 
will allocate the required memory when the vector is reinitialized to a new size.
Hence, for large vectors this may need some time.

To avoid multiple programs accessing the same files on the hard drive, a 
\texttt{lock} file is initialized. Under normal conditions, this will be deleted 
once the program terminates. However, should the program exit exceptionally, the 
lock file will still exist. Calling the program in this case will produce an 
exception, with the following text:
\begin{verbatim}
Warning: During execution of `StateVector<VECTOR>::StateVector` 
         the following Problem occurred!
The directory Results/tmp_state/ is propaply already in use.
\end{verbatim}
To resolve the issue, you have to delete the named directory with the 
temporary storage files manually.