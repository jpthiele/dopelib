\textbf{General problem description}

In the present example, we solve a 
nonstationary fluid-structure interaction problem. 
The underlying equations are stated in the following:
\begin{Problem}[Variational fluid-structure interaction framework]
  \label{eq:fsi:ale:harmonic}
  Find $\{\hat v,\hat u,\hat p\} \in \{ \hat v^D + \hat {\cal V}^0\} 
\times \{ \hat u^D + \hat {\cal V}^0 \}\times \hat {\cal L}$, 
  such that $\hat v (0) = \hat v^0$ and $\hat u(0) = \hat u^0$, for almost all
  time steps $t$, and
  \begin{eqnarray*}
    \begin{aligned}
      (\hat J\hat\rho_f  \partial_t \hat v,\hat\psi^v)_{\hat\Omega_f}  
      +(\hat\rho_f \hat J  (\hat F^{-1}(\hat
      v-\partial_t \hat u)\cdot\hat\nabla) \hat v),
      \hat\psi^v)_{\hat\Omega_f} &\\
      + (\hat J\hat\sigma_f\hat
      F^{-T},\hat\nabla\hat\psi^v)_{\hat\Omega_f}
      %
      + (\hat\rho_s \partial_t \hat v,\hat\psi^v)_{\Omega_s}  
      + (\hat J\hat\sigma_s\hat F^{-T},\hat\nabla\hat\psi^v)_{\hat\Omega_s}&\\
      - \langle \hat g, \hat\psi^v \rangle_{\hat\Gamma_N} -
      (\hat\rho_f \hat J\hat f_f, \hat\psi^v)_{\hat\Omega_f}
      - (\hat\rho_s\hat f_s, \hat\psi^v)_{\hat\Omega_s}
      &=0&&\forall\hat\psi^v\in \hat V^0,
      \\
      %%%%%%%%%%%% 
      (\partial_t\hat u-\hat v,\hat\psi^u)_{\hat\Omega_s} 
      + 
      (\hat\sigma_g ,\hat \nabla\hat\psi^u)_{\hat\Omega_f}
      -\langle \hat\sigma_g \hat n_f ,\hat\psi^u\rangle_{\hat \Gamma_i}
      &=0&&\forall\hat\psi^u\in \hat V^0,\\
      %%%%%%%%%%%%%%%%%%%%%%%%%%%%%% 
      (\widehat{\text{div}}\,(\hat J\hat F^{-1}
      \hat v_f),\hat\psi^p)_{\hat\Omega_f} 
      + (\hat p_s ,\hat \psi^p)_{\hat\Omega_s}
      &=0&&\forall\hat\psi^p\in \hat L,
    \end{aligned}
  \end{eqnarray*}  
  with $\hat\rho_f$, $\hat\rho_s$, $\nu_f$, $\mu_s$, $\lambda_s$, $\hat F$, and $\hat J$. 
  The stress tensors for the fluid and structure are
implemented in  
  $\hat\sigma_f$, $\hat\sigma_s$, and $\hat\sigma_g$ 
\end{Problem}







{\bf Code validation for fluid problems with ALE}

With the ALE code implemented in Example 
\ref{PDE_Stat_FSI_STVK} it is possible to treat fluid 
problems as well as FSI computations. 
In the case of fluid problems the deformation
gradient and its determinant become:
\begin{equation*}
F:= I , \quad \det F = J = 1.
\end{equation*}

The code is validated by the well-known 
fluid- and FSI benchmark problems. Here, 
the results have been summarized in the table below.

 \begin{table}[h]
   \small
   \centering
     \begin{tabular}{llllcccc}    
       \hline
       Configuration & Date         & Refinement & Time dis. & $k$ & $\Delta p$& $F_D$ & $F_L$   \\ \hline\hline
       BFAC 2D-1     & Mar 10, 2010 & 2 global   & BE        & 1.0    & 0.117412  &       &         \\
       BFAC 2D-1     & Mar 10, 2010 & 2 global   & FS        & 1.0    & 0.117412  &       &         \\
       BFAC 2D-1     & Mar 10, 2010 & 2 global   & CN(k)     & 1.0    & 0.117412  &       &         \\ 
       BFAC 2D-1     & Mar 10, 2010 & 2 global   & CN        & 1.0    & 0.117383  &       &         \\ 
       BFAC 2D-2     & Mar 10, 2010 & 2 global   & CN(k)     & 1.0    & 2.31541   &       &         \\ 
       BFAC 2D-2     & Mar 10, 2010 & 3 global   & CN(k)     & 1.0    & 2.32011   &       &         \\   
       BFAC 2D-2     & Mar 12, 2010 & 3 global   & CN(k)     & 1.0e-2 & 2.50288   &       &         \\   
     \end{tabular}
  \end{table}
 
\subsubsection{Code validation for FSI with ALE}

The results are summarized below:

 \begin{table}[h]
   \small
   \centering
     \begin{tabular}{llllccccc}    
       \hline
       Config. & Date & Refinement & Time dis. & $k$ &  $u_x(A) [\times 10^{-5}]$ &$u_y(A) [\times 10^{-4}]$ & $F_D$ & $F_L$  \\ \hline
       FSI 1   & Mar 12, 2010 & 3 global & BE   & 1.0 & 8.2003  & 2.2732  & &        \\
       FSI 1   & Apr 08, 2010 & 2 global & BE   & 1.0 & 8.2258  & 2.2813  & &        \\
       FSI 1   & Apr 08, 2010 & 2 global & CN(k)& 0.5 & 8.2268  & 2.2813  & &        \\
       FSI 1   & Apr 08, 2010 & 2 global & CN   & 0.5 & 8.2268  & 2.2813  & &        \\
     \end{tabular}
  \end{table}

\vspace{0.2cm}

\textbf{Program description}

\vspace{0.2cm}
