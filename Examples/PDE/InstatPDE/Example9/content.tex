\subsubsection{General problem description}

In this example we consider one of the prototypical nonstationary equations, the parabolic heat equation, i.e. for $x \in \Omega \subset \mathbb{R}^d, t\in I =[0,T], T\in \mathbb{R}^+,$ we search for the unknown solution $u:I \times \Omega \rightarrow \mathbb{R}$
\begin{align*}
\partial_t u(t,x) - \Delta u(t,x) &= f(t,x),\\
u(t,x)|_{\partial \Omega} &= g(t,x),\\
u(0,x) &= u_0(x).
\end{align*} In our example, we consider the case $d=2$,
where the Laplacian $\Delta$ reduces to $\partial_{x_1}^2 + \partial_{x_2}^2 $. The
computational domain is $ I\times \Omega = [0,1] \times [0,1]^2$. For
further simplification, we choose the right hand side as $f=0$ and the
initial condition is given by $u_0(x) = \min(x_1,1-x_1)$.
The Dirichlet-data are
\[
  g(t,x) = \begin{cases} tx_2 & x_1 = 1,\\
    0 & \text{otherwise.}
  \end{cases}
\]

\subsubsection{Program description}
The novelty of this program is the use of varying spatial meshes in
time, i.e., we use a \index{RotheDiscretization} Rothe-discretization,
where different spatial meshes at different time-points are allowed.

To use the Rothe-Discretization, we use a different DoF-Handler, i.e.,
we need to include
\begin{verbatim}
#include <basic/rothe_statespacetimehandler.h>
\end{verbatim}
To initialize the different meshes, we prepare a vector
\texttt{Rothe\_time\_to\_dof} which stores for each time-point a number
indicating the number of the DoF-Handler to be used. In this example,
we want a different DoF-Handler at each time-point, hence we
initialize
\begin{verbatim}
std::vector<unsigned int> Rothe_time_to_dof(n_time_steps+1,0);

for(unsigned int i = 0; i < Rothe_time_to_dof.size(); i++)
    Rothe_time_to_dof[i]=i;

Rothe_StateSpaceTimeHandler<FE, DOFHANDLER, SPARSITYPATTERN, VECTOR,
			      DIM> DOFH(triangulation, state_fe, times, Rothe_time_to_dof);
\end{verbatim}
Note, that the vector \texttt{Rothe\_time\_to\_dof} requires its indices
to satisfy the conditions
\begin{itemize}
\item \verb|Rothe_time_to_dof[0] == 0|
\item If \verb|Rothe_time_to_dof[i] == n| for some $n > 0$, then there
  must be an index $j < i$ with \verb|Rothe_time_to_dof[j] == n-1|. 
\end{itemize}
This allows, to use, e.g. only two DoF-Handlers, with number 0 and 1,
that are used on even and odd time-points by setting 
\begin{verbatim}
for(unsigned int i = 0; i < Rothe_time_to_dof.size(); i++)
    Rothe_time_to_dof[i]=i%2;
\end{verbatim}
Similarly, one could use one DoF-Handler for the first 10 time-points
and then a different DoF-Handler by appropriately assigning values to \texttt{Rothe\_time\_to\_dof}.

The only other change to other programs is that now the refinement
needs to be given error indicators for each time-point, i.e., a
\verb|std::vector<dealii::Vector<float> >| where the outer
\texttt{std::vector} needs to have the size \texttt{n\_time\_steps+1}.
This is immediately given by the
\texttt{ResidualErrorContainer}. Hence, for refinement of the
different spatial meshes, we just have to call
\begin{verbatim}
const std::vector<dealii::Vector<float> > 
               error_ind(h1resc.GetErrorIndicators());
DOFH.RefineSpace(SpaceTimeRefineOptimized(error_ind));
\end{verbatim}

To evaluate the error indicator, we write the methods
\begin{verbatim}
void StrongElementResidual;
void StrongFaceResidual;
\end{verbatim}
in \texttt{localpde.h} analogous to
Example~\ref{PDE_adap_Stat_Laplace},
except that the ElementResidual now needs to contain the
discrete time-derivative, see, e.g.,~\cite{Verfuehrt:2003}.
Note, that in this example no indicators for the temporal error and
the mesh-change error are included.
