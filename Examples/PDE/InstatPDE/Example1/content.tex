\subsubsection{General problem description}
In this example we consider the nonstationary incompressible Navier-Stokes equation. As in the stationary PDE Example 3 \ref{PDE_Stat_FSI_STVK},
we use the symmetric fluid stress tensor, i.e. 
in strong formulation we deal with 
\begin{align*}
\partial_t v -\nabla\cdot (\nabla v + \nabla v^{T}) + (v\cdot\nabla)v  +\nabla p &= f \\
\nabla\cdot v &= 0 
\end{align*}
on time interval $I=[0,T]$ and the domain 
$\Omega = [-6,6]\times [0,2]$. For simplicity,  we set $f=0$.


As introduced earlier, we formulate the time stepping scheme as \textit{One-step-$\theta$ scheme},
which are based on finite difference schemes. In order to keep the presentation simple, we describe the scheme using the stokes equation and thus neglecting the nonlinearity. Note that in the program we use the full Navier-Stokes operator.

The time interval is given by $I=[0,T]$. Let $v^n,p^n$ and the time
step $k=t^{n+1}-t^n$ be given. Find $v^{n+1}, p^{n+1}$ such that:

\begin{align*}
v^{n+1} - k\theta \left(\nabla\cdot (\nabla v^{n+1} + \nabla {v^{n+1}}^T) + \nabla p^{n+1}\right) &=
k\theta f^{n+1} + k(1-\theta)f^{n}\\
&+  v^n + k(1-\theta) \left(\nabla\cdot (\nabla v^n + \nabla (v^n)^{T}) 
+ \nabla p^n\right) \\
\nabla \cdot v^{n+1} &= 0 
\end{align*}
In the case of the BE-scheme, $\theta = 1$, and the equation is reduced to
\begin{align*}
v^{n+1} - k (\nabla\cdot (\nabla v^{n+1} + \nabla {v^{n+1}}^{T}) + \nabla p^n) &=
k\theta f^{n+1} +  v^n  \\
\nabla \cdot v^{n+1} &= 0
\end{align*}
Note, that one should prefer a complete implicit treatment of the
pressure $p$. Instead of using $\theta p^{n+1} + (1-\theta)p^n$, the pressure
appears only with $\theta p^{n+1}$.

After discretization in time, the space is treated, as ususally, with 
a Galerkin finite element scheme, here based on the Taylor-Hood element 
$Q_2^c / Q_1^c$.

The variational formulation reads:

\begin{Problem}[Backward Euler (BE) timestepping problem]
Let $\theta =1$.Find $v:= v^{n+1}\in V$ and $p:= p^{n+1}\in L$:
\begin{align}
(v,\phi^v) + k\theta (\nabla v + \nabla v^{T}, \nabla\phi^v) 
- k  (p, \nabla\cdot\phi^v) &=
k\theta (f^{n+1},\phi^v) +  (v^n,\phi^v)\\
(\nabla \cdot v,\phi^p) = 0  \label{eq.divfreiheit}
\end{align}
\end{Problem}
for all suitable test functions ${\phi^v , \phi^p} \in V\times L$. 


Derivation of the other timestepping problems is analogous.
\begin{remark}
Note that because of the zero right hand side we are allowed to multiply \eqref{eq.divfreiheit} by $k\theta$. So that we solve $$k\theta(\nabla \cdot v,\phi^p) = 0$$ instead of $(\nabla \cdot v,\phi^p) = 0$. 
\end{remark}


\subsubsection{Specific features for 
solving nonstationary problems}

In the following, we explain in more detail the 
different member functions that are required to
implement nonstationary equations.

\begin{verbatim}
void CellEquation (..., double scale, double scale_ico)
\end{verbatim}
The two arguments 
are used to distigunish between explicit 
components and fully implicit components. 
For standard equations (such as the heat 
equation and the wave equation), there is 
no special treatment required needed. 

However, solving the Navier-Stokes equations or 
multi-physics problems (like fluid-structure
interaction), parts of the equations are treated
with a fully implicit time-stepping scheme.

Thus, the argument
\begin{verbatim}
double scale
\end{verbatim}
is used to indicate that the present term 
can be used for implicit/explicit or mixed 
discretization (such as time discretization with
the Crank-Nicolson.

The other argument
\begin{verbatim}
double scale_ico
\end{verbatim} 
is used to indicate that the present term 
only is treated in a fully implicit manner. 
For example, the pressure term (which is of course
a Lagrange multiplier of the incompressibility 
term of the fluid). It is recommended to treat 
this term in a time discretization in a fully 
implicit manner. 

\begin{verbatim}
void CellMatrix (..., double scale, double scale_ico)
\end{verbatim}
The directional derivatives of the state equation
are implemented in the present function. As before,
the last two parameters
\begin{verbatim}
double scale, double scale_ico
\end{verbatim} 
are used to distigungish between fully implicit 
are other behavior. 

\begin{verbatim}
void CellTimeEquation (...) 
\end{verbatim}
This function is used to implement the 
time derivative in weak formulation 
\begin{equation*}
(\partial_t v, \phi)_{\Omega}.
\end{equation*}
This term is time discretized via
\begin{equation*}
k^{-1} (v^n - v^{n-1} , \phi)_{\Omega}. 
\end{equation*}
Here, it suffices to implement the term
\begin{equation*}
(v^n, \phi)_{\Omega},
\end{equation*}
because the already known term $v^{n-1}$ is 
automatically treated by the specific time
stepping scheme. 

In contrast to this behavior, the user 
has the possibility to write all terms of 
$\partial_t v$ explicitly. In this case, we 
use the 
\begin{verbatim}
void CellTimeEquationExplicit (...) 
\end{verbatim}
and we write 
\begin{equation*}
(v^n - v^{n-1}, \phi)_{\Omega}.
\end{equation*}
This behavior is useful for multi-phyics problems
where other solutions variables have to be considered
around $\partial_t v$. The user should have a look 
in the second Example 
\ref{PDE_Instat_FSI} for nonstationary problems 
for an illustration
of this function. 

Consequently, the directional derivatives of the 
cell terms are implemented in the corresponding
matrix functions, i.e., 
\begin{verbatim}
void CellTimeMatrix (...), void CellTimeMatrixExplicit 
\end{verbatim}


