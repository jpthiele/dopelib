\subsubsection{General problem description}
We consider phase-field fracture propagation on a slit domain, with time dependent 
Dirichlet data and homogeneous Neumann data.
The main novelties are:
\begin{itemize}
\item Imposing a variational inequality constraint ($\frac{\partial \varphi}{\partial t} \leq 0$)
via a Lagrange multiplier and complementarity formulation ;
\item Stress splitting into tension and compression \`a
la \cite{MieWelHof10a}, see as well \cite{AmorMarigoMaurini2009}, located 
in 
\begin{verbatim}
stress_splitting.h
\end{verbatim}
\item Implementation of a Newton solver that 
allows for an increase of the residual for which, however,
no convergence results are available:
\begin{verbatim}
DOpEsrc/templates/instat_step_modified_newtonsolver.h
\end{verbatim}
\item The localpde file:
\begin{verbatim}
localpde_fully_implicit.h
\end{verbatim}
that provides a fully implicit implementation.
The fully implicit implementation requires a modified Newton scheme (see the
previous bullet point) that 
allows for a temporary increase of the Newton residual (similar to \cite{Wi17_pff_error_oriented_Newton}).
\end{itemize}
We define the function spaces $\mathcal{V}:= H_0^1(\Omega)^2$ and
$\mathcal{W}:=  H^1(\Omega)$. 
Further, we define $\mathcal{X}:=\{\tau \in \mathcal{W}^*\, \vert\, \tau \geq 0\}$, 
where $\mathcal{W}^*$ is a dual space of $\mathcal{W}$ and $\mathcal{K}$ as the convex set
\[
\mathcal{K}:= \mathcal{K}^n = \{w\in \mathcal{W} |\, w\leq \varphi^{n-1} \leq 1 \text{ a.e. on }
\Omega\}.
\] 
Let $u \in \mathcal{V}$ be the displacement and $\varphi \in \mathcal{W}$ be the phase field variable which should highlight the crack.
To realize the inequality constraint, we introduce a Lagrange multiplier $\tau \in \mathcal{X}$ as a third unknown variable. \\
Furthermore we denote the $L^2(\Omega)$ inner product of $v_1$ and $v_2$ with $(v_1,v_2)$.
A material is supposed to be undamaged at position $x$ if $\varphi(x)$ is close to $1$ and completely cracked
if it is close to $0$.
We want to find a stationary point of the energy functional
\begin{align*}
E(u, \varphi):=& \frac{1}{2}(\sigma(u,\varphi),\varepsilon(u)) \\
+&\frac{G_c}{2}(\frac{1}{\varepsilon} \Vert 1 - \varphi \Vert^2)+\varepsilon \Vert \nabla \varphi  \Vert\\
%+&\underbrace{\frac{\gamma_{pen}}{2} ((\varphi(t_n)-\varphi(t_{n-1}))_+)^2}_{\text{penalization term}},
\end{align*}
under the constraint 
\begin{align*}
\partial_t \varphi \leq 0.
 \end{align*}
The constraint realizes the crack irreversibility. Physically-speaking: the
crack cannot heal. To derive an incremental version, the constraint is discretized in time via:
\begin{align*}
 \frac{\varphi(t^{n+1})-\varphi(t^n)}{t^{n+1}-t^n} \leq 0.
\end{align*}
Further it is defined
\begin{align*}
\sigma(u,\varphi)&:= (1-\kappa)\varphi^2 \sigma(u)^+ + \sigma(u)^-,\\
\sigma(u)^+ &:= 2\mu E(u)^+ +\lambda (\operatorname{trace}(E(u)))_+ \textbf{I},\\
\sigma(u)^- &:= 2\mu E(u)^- +\lambda (\operatorname{trace}(E(u)))_- \textbf{I},\\
(c)_+ &:= \max(0, c), \\
(c)_- &:= c-(c)_+, \\
E(u)^+ &:= \sum_{i=1}^{d} (\lambda_i)_+ n_i^T n_i,\\
E(u)^- &:= \sum_{i=1}^{d} (\lambda_i)_- n_i^T n_i,\\
E(u)   &:= \frac{1}{2}(\nabla u +(\nabla u)^T),
\end{align*}
where $c \in \mathbb{R}$, $t^n$ denotes the time at timestep $n$, $\lambda_i$ denotes the $i-\rm{th}$ eigenvalue of $E(u)$ and $n_i$ the corresponding eigenvector.
The idea for the choice of the energy functional is presented
in \cite{MieHofWel2010} (based on the original work \cite{BourFraMar00,BourFraMar08,FraMar98}).
To realize the inequality constraint $\varphi 
\leq \varphi^{n-1}$, we introduce
a Lagrange multiplier $\tau_h \in \mathcal{X}_h$ as proposed in \cite[Section~4.1]{MangWickWollner:2020}.
This leads to the following discrete problem: \\
Choose discrete function spaces $\mathcal{V}_h \subset \mathcal{V}$, $\mathcal{U}_h \subset \mathcal{U}$, $\mathcal{W}_h \subset \mathcal{W}$ and a proper subset $\mathcal{X}_h \subset \mathcal{X}$.
Given the initial data $\varphi_h^{n-1} \in \mathcal{W}_h$.
For the loading steps $n=1,2,\ldots, N$ solve the following system of equations:
Find $u_h \in \mathcal{V}_h$, $p_h \in \mathcal{U}_h$, $\varphi_h:= \varphi_h^n \in \mathcal{W}_{n,h}$ and $\tau_h \in \mathcal{X}_h$ such that 
 \begin{align*}
 \begin{aligned}
   \left(g(\varphi_h^{n})\left[2\mu E^+(u_h) + \lambda (\operatorname{trace}(E(u)))_+ \textbf{I}\right],\nabla w_h \right)\\
  + \left(2\mu \left(E(u_h)-E^+(u_h)\right),E(w_h)\right)\\
 + \left(\lambda (\operatorname{trace}(E(u)))_- \textbf{I},E(w_h)\right) =&\ 0,\\
 %\forall w_h\in&\ \mathcal{V}_h,\\
 %\left(g(\varphi_h) \nabla \cdot u_h,q_h\right) - \frac{1}{\lambda} \left(g(\varphi_h) p_h,q_h\right)=&\ 0,\\
 %\forall q_h\in&\ \mathcal{U}_h,\\
  (1- \kappa)\left(\varphi_h 2\mu E^+(u_h) + \lambda (\operatorname{trace}(E(u)))_+ \textbf{I} : E(u_h),\psi_h\right)\\
  + G_c \left(-\frac{1}{\epsilon} (1-\varphi_h),\psi_h\right)\\ 
  + G_c \epsilon \left(\nabla \varphi_h,\nabla\psi_h \right) - \left(\tau_h,\psi_h\right) =&\ 0,\\
  %\forall \psi_h \in&\ \mathcal{W}_h,\\
%\tau_h\geq&\ 0,\\
%\varphi_h^n - \varphi_h^{n-1} \leq&\ 0,\\
%\left(\tau_h,\varphi_h^n -\varphi_h^{n-1}\right)=&\ 0,
\left(\tau_h - \chi_h,\varphi_h -\varphi_h^{n-1}\right)\geq&\ 0,
\end{aligned}
\end{align*}
for all $\left\{w_h,\psi_h,\chi_h\right\} \in \mathcal{V}_h \times
\mathcal{K}_h \times \mathcal{X}_h$.
For the implementation, the last inequality is replaced by a
non-smooth complementarity function, see, e.g.,~\cite[Section~4.1]{MangWallothWickWollner:2019}.

The test case is the so-called single edged notched 
shear test. For further details about this problem it is referred
to \cite{MieHofWel2010,BoVeScoHuLa12}, and \cite{WeWiWo2014}.
For the choice of the boundary conditions see also~\cite[Section~5]{MangWallothWickWollner:2019}.





