\subsubsection{General problem description}
We consider phase-field fracture propagation on a slit domain, with time dependent 
Dirichlet data and homogenous Neumann data.
The main novelties are:
\begin{itemize}
\item Imposing a variational inequality constraint ($\frac{\partial \varphi}{\partial t} \leq 0$)
via penalization;
%\item Implementation of an augmented Lagrangian loop (not yet)
\item Stress splitting into tension and compression \`a
la \cite{MieWelHof10a}, see as well \cite{AmorMarigoMaurini2009}, located 
in 
\begin{verbatim}
stress_splitting.h
\end{verbatim}
\item Implementation of a Newton solver that 
allows for an increase of the residual for which, however,
no convergence results are available:
\begin{verbatim}
DOpEsrc/templates/instat_step_modified_newtonsolver.h
\end{verbatim}
\item Two localpde* files:
\begin{verbatim}
localpde_quasi_monolithic.h
localpde_fully_implicit.h
\end{verbatim}
that provide implementations of a quasi-monolithic scheme (similar, but not
the same as \cite{HeWheWi15}) and a fully implicit implementation.
The fully implicit implementation requires a modified Newton scheme (see the
previous bullet point) that 
allows for a temporary increase of the Newton residual (similar to \cite{Wi17_pff_error_oriented_Newton}).
\end{itemize}
Let $u$ be the displacement and $\varphi$ be the phase field variable which should highlight the crack.
Furthermore we denote the $L^2(\Omega)$ inner product of $v_1$ and $v_2$ with $(v_1,v_2)$.
A material is supposed to be undamaged at position $x$ if $\varphi(x)$ is close to 1 and comletely cracked
if it is close to $0$.
We want to find a stationary point of the energy functional
\begin{align*}
E(u, \varphi):=& \frac{1}{2}(\sigma(u,\varphi),\varepsilon(u)) \\
+&\frac{G_c}{2}(\frac{1}{\varepsilon} \Vert 1 - \varphi \Vert^2)+\varepsilon \Vert \nabla \varphi  \Vert\\
+&\underbrace{\frac{\gamma_{pen}}{2} ((\varphi(t_n)-\varphi(t_{n-1}))_+)^2}_{\text{penalization term}},
\end{align*}
with
\begin{align*}
\sigma(u,\varphi)&:= (1-\kappa)\varphi^2 \sigma(u)^+ + \sigma(u)^-,\\
\sigma(u)^+ &:= 2*\mu \varepsilon(u)^+ +\lambda (trace(\varepsilon(u)))_+ I,\\
\sigma(u)^- &:= 2*\mu \varepsilon(u)^- +\lambda (trace(\varepsilon(u)))_- I,\\
(c)_+ &:= max(0, c), \\
(c)_- &:= c-(c)_+, \\
\varepsilon(u)^+ &:= \sum_{i=1}^{d} (\lambda_i)_+ n_i^T n_i,\\
\varepsilon(u)^- &:= \sum_{i=1}^{d} (\lambda_i)_- n_i^T n_i,\\
\varepsilon(u)   &:= \frac{1}{2}(\nabla u +(\nabla u)^T),
\end{align*}
where $c \in \mathbb{R}$, $t^n$ denotes the time at timestep $n$, $\lambda_i$ denotes the $i-th$ eigenvalue of $\varepsilon(u)$ and $n_i$ the corresponding eigenvector.
The idea for the choice of the energy functional is presented
in \cite{MieHofWel2010} (based on the original work \cite{BourFraMar00,BourFraMar08,FraMar98}).
This leads to the following problem: Find $(u, \varphi) \in V_u \times V_\varphi$ such that

\begin{align*}
A(u,\varphi)(\phi_u,\phi_\varphi):=& (\sigma(u,\varphi),\varepsilon(\phi_u)) \\
+& (((1-\kappa)\varphi \sigma(u)^+,\varepsilon(u)),\phi_\varphi)\\
 +& G_c(\frac{1}{\varepsilon} (1 - \varphi,\phi_\varphi)+\varepsilon (\nabla \varphi, \nabla \phi_\varphi)\\
+&\gamma_{pen} ((\varphi(t_n)-\varphi(t_{n-1}))_+,\phi_\varphi)=0 \quad \forall (\phi_u, \phi_\varphi) \in V_u \times V_\varphi.
\end{align*}

The test case is the so-called single edged notched 
shear test. For further details about this problem it is referred
to \cite{MieHofWel2010,BoVeScoHuLa12}, and \cite{WeWiWo2014}.



\subsubsection{Future improvements}
Possible future improvements are:
\begin{itemize}
\item Implement the phase-field extrapolation in the displacement equation proposed in \cite{HeWheWi15};
\item Implement the augmented Lagrangian loop proposed by
Wheeler/Wick/Wollner, CMAME 2014 \cite{WeWiWo2014}. 
\end{itemize}

% Need to implement extrapolation with two 
% previous time steps
% To be changed: 
% instatpdeproblem.h
% instat_step_newton_solver.h
%
% augmented Lagrangian solver loop
% -> currently simple penalization
% has to be implemented in instat_step_newton_solver.h
%
% Dokumentation in content.tex



