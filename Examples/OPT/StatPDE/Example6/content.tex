\subsubsection{General problem description}
This example solves the distributed minimization problem
\begin{gather*}
\min J(q,u) = \frac{1}{2} \|u-u^d\|^2 + \frac{\alpha}{2}\|q\|^2\\
\text{s.t.} (\nabla u,\nabla \phi) = (q+f,\phi)\;\;\forall\,\phi \in H^1_0(\Omega)\\
\text{s.t.} -500 \le q \le 500\;\text{a.e. in }\Omega
\end{gather*}
on the domain $\Omega = [0,1]^2$, and the data is chosen as follows:
\begin{gather*}
 f = \left(20\pi^2  \sin(4 \pi x) - \frac{1}{\alpha}  \sin(\pi x)\right) \sin(2 \pi y)\\
 u^d = \left( 5 \pi^2 \sin(\pi x) + \sin(4 \pi x)\right)  \sin(2\pi y)
\end{gather*}
and $\alpha = 10^{-3}$.

In addition the following functionals are evaluated:
\begin{gather*}
  \text{MidPoint: } u(0.5 ; 0.5)\\[2mm]
  \text{MeanValue: }\int_\Omega u
\end{gather*}


\subsubsection{Program description}

{\textcolor{red}{
The Problem is similar to that of {\tt OPT/StatPDE/Example1} except for the box control constraints.
The implementation of these constraints is taken care of in the main file
(where we add a constrained description \texttt{lcc}) and in the
\texttt{localconstraints.h} file. This files serves to implement the actual constraints. 
In this example, we introduce the handling of \textit{local} constraints
whereas the mixture of local and \textit{global} constraints will be discussed
in the next Example \ref{OPT_Stat_MBB-Beam}. 
%
First, we implement the upper and lower control bounds in 
\texttt{localconstraints.h}, i.e,
\[
q_{min} \leq q \leq q_{max}.
\]
with $q_{min} = -500$ and $q_{max} = 500$.
%
Second, the \texttt{lcc} vector describes the meaning and the applicability of
these constraints. Further information can be found
in \texttt{basic/constraints.h}. We give here the necessary information for
the 
present example. We initialize \texttt{lcc(1)}, i.e., we have 1 control block.
Then, this block is given the constant size $2$ (in the case of more than one 
control block, each block would be given a size of $2$). The first
entry \texttt{lcc[0][\textcolor{red}{0}] = 1} tells 
you how many local entries in the present block are locally constrained, here 
it is $1$ local entry. The second entry \texttt{lcc[0][\textcolor{red}{1}] =
2}, determines the number of constraints on this local entry. Here, we impose 
a lower and an upper bound, i.e., we give 2 constraints.
}




\subsubsection{External optimization solver}
The problem is solved using the optimization library IPOPT that you can obtain for free. To use it 
a correct link to the ipopt library needs to be created in  {\tt DOpE/ThirdPartyLibs} by the name 
{\tt ipopt}, i.e., you should have the file
{\tt DOpE/ThirdPartyLibs/ipopt} pointing to the ipopt directory. If you have not done this you can compile the
example but when running the example you will only get an error message like\\
{\tt Warning: During execution of `Reduced\underline{ }IpoptAlgorithm::Solve`\\ the following Problem occurred!\\
To use this algorithm you need to have IPOPT installed! \\To use this set the WITH\underline{ }IPOPT CompilerFlag.}

Alternatively the commercial optimization library SNOPT can be used in this example.
In order to use this library you need to install SNOPT on your computer and then generate a symlink to 
the snopt directory (where you have the libs and the header files) 
in the {\tt DOpE/ThirdPartyLibs} directory named {\tt snopt}, i.e., you should have the file
{\tt DOpE/ThirdPartyLibs/snopt} pointing to the snopt directory. If you have not done this you can compile the
example but when running the example you will only get an error message like\\
{\tt Warning: During execution of `Reduced\underline{ }SnoptAlgorithm::Solve`\\ the following Problem occurred!\\
To use this algorithm you need to have SNOPT installed! \\To use this set the WITH\underline{ }SNOPT CompilerFlag.} 
