\subsubsection{General problem description}
This example solves the distributed minimization problem
\begin{gather*}
\min J(q,u) = \frac{1}{2} \|u-u^d\|^2 + \frac{\alpha}{2}\|q\|^2\\
\text{s.t.} (\nabla u,\nabla \phi) = (q+f,\phi)\;\;\forall\,\phi \in H^1_0(\Omega)\\
\text{s.t.} -500 \le q \le 500\;\text{a.e. in }\Omega
\end{gather*}
on the domain $\Omega = [0,1]^2$, and the data is chosen as follows:
\begin{gather*}
 f = \left(20\pi^2  \sin(4 \pi x) - \frac{1}{\alpha}  \sin(\pi x)\right) \sin(2 \pi y)\\
 u^d = \left( 5 \pi^2 \sin(\pi x) + \sin(4 \pi x)\right)  \sin(2\pi y)
\end{gather*}
and $\alpha = 10^{-3}$.

In addition the following functionals are evaluated:
\begin{gather*}
  \text{MidPoint: } u(0.125 ; 0.75)\\[2mm]
  \text{MeanValue: }\int_\Omega |u|
\end{gather*}


\subsubsection{Program description}

The Problem is similar to that of \texttt{OPT/StatPDE/Example1} except for the box 
control constraints.
The implementation of these constraints is taken care of in the main file
(where we add a constrained description \texttt{lcc}) and in the
\texttt{localconstraints.h} file. This files serves to implement the actual constraints.
 
In this example, we introduce the handling of \textit{local} constraints
whereas the mixture of local and \textit{global} constraints will be discussed
in the Example \ref{OPT_Stat_MBB-Beam}. 
%
First, we implement the upper and lower control bounds in 
\texttt{localconstraints.h}, i.e,
\[
q_{\min} \leq q \leq q_{\max}.
\]
with $q_{\min} = -500$ and $q_{\max} = 500$.
The constraints are `local', by which we mean the constraints are imposed 
on the nodal values of the control vector. Thus, in the constraint 
description \texttt{localconstraints.h},
these vectors are manipulated directly without additional integration. We 
note that the constraints need to be written such, that a feasible control
generates non positive entries, i.e., we 
calculate the vector
\[
 \mathcal C(q) = \begin{pmatrix}
   q_{\min} - q\\q-q_{\max}
 \end{pmatrix}.
 \]
%
Second, the \texttt{lcc} vector is used to describe the amount of unknowns that 
need to be reserved to store the constraints, and, eventually, corresponding 
Lagrange multipliers. 
Further information can be found in \texttt{basic/constraints.h}. 
In our example, we have only one block in the control (The control is stored in 
a \texttt{deal.II::BlockVector}). Hence, the \texttt{lcc} vector has a length of 
one. 
The only entry \texttt{lcc[0]} is a vector of size two (in the case of more than one 
control block, each block would be given a size of $2$). 
Each of the two entries has a specific meaning. 
The first entry \texttt{lcc[0][\textcolor{blue}{0}] = 1} tells 
you how many local entries in the present block are locally constrained, here 
it is one local entry. ({\em Note that this entry will typically be one. 
it is not one  would be if we have constraints of the type $q_{i} + q_{i+1} \le 1$ for each 
even $i$, or similar combinations of multiple entries in the control vector.})

The second entry \texttt{lcc[0][\textcolor{blue}{1}] = 2}, determines the number of 
constraints on this local entry. Here, we impose 
a lower and an upper bound, i.e., we give 2 constraints.
This information tells the SpaceTimeHandler, that the vector 
$\mathcal C(q)$ needs exactly twice the amount of unknowns as the vector $q$.
In general, the space needed for $\mathcal C(q)$ is given as 
\[
\frac{\text{\texttt{lcc[0][1]}}}{\text{\texttt{lcc[0][0]}}}
\]
times the unknowns for the control.

\subsubsection{External optimization solver}
The problem is solved using the optimization library IPOPT that you can obtain for free. To use it 
a correct link to the ipopt library needs to be created in  \texttt{DOpE/ThirdPartyLibs} by the name 
\texttt{ipopt}, i.e., you should have the file
\texttt{DOpE/ThirdPartyLibs/ipopt} pointing to the ipopt directory. If you have not done this you can compile the
example but when running the example you will only get an error message like\\
\texttt{Warning: During execution of `Reduced\underline{ }IpoptAlgorithm::Solve`\\ the following Problem occurred!\\
To use this algorithm you need to have IPOPT installed! \\To use this set the WITH\underline{ }IPOPT CompilerFlag.}
If you receive this message and have the ipopt installation complete, you
might have overseen to add ipopt to your \texttt{LD\underline{ }LIBRARY\underline{ }PATH}:
\begin{lstlisting}
**************************************************************
                 Installation complete!
Add /home/..../dopelib-2.0/ThirdPartyLibs/ipopt/lib64
    to your $LD_LIBRARY_PATH variable
**************************************************************
\end{lstlisting}


Alternatively the commercial optimization library SNOPT can be used in this example.
In order to use this library you need to install SNOPT on your computer and then generate a symlink to 
the snopt directory (where you have the libs and the header files) 
in the \texttt{DOpE/ThirdPartyLibs} directory named \texttt{snopt}, i.e., you should have the file
\texttt{DOpE/ThirdPartyLibs/snopt} pointing to the snopt directory. If you have not done this you can compile the
example but when running the example you will only get an error message like\\
\texttt{Warning: During execution of `Reduced\underline{ }SnoptAlgorithm::Solve`\\ the following Problem occurred!\\
To use this algorithm you need to have SNOPT installed! \\To use this set the WITH\underline{ }SNOPT CompilerFlag.} 
