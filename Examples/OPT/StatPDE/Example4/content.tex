\subsubsection{General problem description}
This example solves the minimization problem
\begin{gather*}
\min J(q,u) = \frac{1}{2} \|u-u^d\|^2 + \frac{\alpha}{2}\|q\|^2\\
\text{s.t. } (\nabla u,\nabla \phi) = (f,\phi)\;\;\forall\,\phi \in H^1_0(\Omega; \mathbb R^2)\\
\end{gather*}
on the domain $\Omega = [0,1]^2$. In addition, we set the Dirichlet data of the state on the boundary as follows
\begin{gather*}
 u_0(0,y) = q_0,\quad u_0(1,y) = q_1,\quad u_0(x,0) = q_2,\quad u_0(x,1) = q_3,\\
 u_1 = q_4^3.
\end{gather*}
The data is chosen as follows:
\begin{align*}
 f &= \left(\begin{matrix}20\pi^2  \sin( \pi x) \sin(\pi y)\\1 \end{matrix}\right)\\
 u^d&= \left(\begin{matrix}\sin( \pi x) \sin(\pi y)*x\\x \end{matrix}\right)
\end{align*}
with $\alpha = 10$.
\subsubsection{Program description}
The control in the Dirichlet boundary values is incorporated via the
class \texttt{LocalDirichletData} which is defined in the
file \textit{localdirichletdata.h}. The class is then given to
the \texttt{OptProblemContainer} as a template argument. This is all that is
needed to use the control in the Dirichlet boundary values. 
In the main file and the localpde program, we work still with the 
\texttt{FE\underline{}Nothing} element to assign the five controls.

Finally, we refine only the boundary of the domain, to demonstrate how
refinement driven by geometric features can be realized. To this end,
the class \texttt{BoundaryRefinement} in \texttt{boundaryrefinement.h}
implements the marking base upon local information on the individual elements.