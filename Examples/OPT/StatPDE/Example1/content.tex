\subsubsection{General problem description}
This example solves a distributed minimization problem
and shows how to estimate the error in the cost functional for stationary optimization
problems. The problem reads:
\begin{gather*}
\min J(q,u) = \frac{1}{2} \|u-u^d\|^2 + \frac{\alpha}{2}\|q\|^2\\
\text{s.t.} (\nabla u,\nabla \phi) = (q+f,\phi)\;\;\forall\,\phi \in H^1_0(\Omega)
\end{gather*}
on the domain $\Omega = [0,1]^2$, and the data is chosen as follows:
\begin{gather*}
 f = \left(20\pi^2  \sin(4 \pi x) - \frac{1}{\alpha}  \sin(\pi x)\right) \sin(2 \pi y)\\
 u^d = \left( 5 \pi^2 \sin(\pi x) + \sin(4 \pi x)\right)  \sin(2\pi y)
\end{gather*}
and $\alpha = 10^{-3}$.
Hence its solution is given by:
\begin{gather*}
 \overline{q} = \frac{1}{\alpha} \sin(\pi x) \sin(2 \pi y)\\
 \overline{u} = \sin(4 \pi x) \sin(2 \pi y).
\end{gather*}
Thus the exact optimal value of the cost functional can be calculated as 
\begin{gather*}
 J^* = J(\overline{q},\overline{u}) = \frac{1}{8}\Bigl(25\pi^4 + \frac{1}{\alpha}\Bigr).
\end{gather*}

In addition the following functionals are evaluated:
\begin{gather*}
  \text{MidPoint: } u(0.5 ; 0.5)\\[2mm]
  \text{MeanValue: }\int_\Omega u
\end{gather*}

{
\subsubsection{Background information and program description}
In the following, we describe all extensions to the previous problems 
relevant to solving PDE-based optimization with \dope{}. So far, 
we had only to implement the \texttt{ElementEquation} and the corresponding
matrix \texttt{ElementMatrix}. Now, based on the idea of the reduced cost
functional, we have to compute certain additional equations representing the 
adjoint, tangent, and adjoint hessian equations
\texttt{ElementEquation\underline{ }U, ElementEquation\underline{ }UT,
ElementEquation\underline{ }UTT}
for the state equation and in the same terms arising from the functional
itself. Let us shed some light into all equations by giving some background
information and overview first.

In abstract form, we are given the following optimization problem:
\begin{equation*}
J(q,u) \, \rightarrow \, \min , \quad a(q,u)(\psi) = 0 \quad \forall\psi\in V
\end{equation*}
Lagrangian:
\begin{equation*}
{\cal L}(q,u,z):=J(q,u) - a(q,u)(z)
\end{equation*}
Optimality system (KKT system):
\begin{align*}
a_u'(q,u)(\phi , z) &= J'_u(q,u)(\phi) \quad \forall \phi\in V \\
a_q'(q,u)(\chi , z) &= J'_q(q,u)(\chi) \quad \forall \chi\in Q \\
a(q,u)(\psi) &= 0 \qquad \forall\psi\in V
\end{align*}
or equivalently, in terms of the Lagrangian
\begin{align*}
{\cal L}'_u(q,u,z)(\phi) &= 0 \quad\forall \phi\in V \quad\text{(Adjoint Equation)} \\
{\cal L}'_q(q,u,z)(\chi) &= 0 \quad\forall \chi\in V \quad\text{(Gradient Equation)} \\
{\cal L}'_z(q,u,z)(\psi) &= 0 \quad\forall \psi\in V \quad\text{(State Equation)}
\end{align*}
The continuous problem is discretized by a standard Galerkin method using 
finite dimensional subspaces $Q_h \times V_h \subset Q\times V$:
\begin{equation*}
J(q,u) \, \rightarrow \, \min , \quad a(q,u)(\psi) = 0 \quad \forall\psi\in V
\end{equation*}
Discrete saddle-point problems
\begin{align*}
a_u'(q_h,u_h)(\phi_h , z_h) &= J'_u(q_h,u_h)(\phi_h) \quad \forall \phi_h\in V_h \\
a_q'(q_h,u_h)(\chi_h , z_h) &= J'_q(q_h,u_h)(\chi_h) \quad \forall \chi_h\in Q_h \\
a(q_h,u_h)(\psi_h) &= 0 \qquad \forall\psi_h\in V_h
\end{align*}
\subsubsection{Solution process}
In this section, we briefly discuss the solution process for the 
optimization problem. For further details, we refer to the standard literature.
The unconstrained optimal control problem is reformulated as follows. 
We introduce the solution operator $S:Q\rightarrow V$ of the state equation.
Then:
\begin{equation*}
j(q):= J(q,S(q))\, \rightarrow \, \min , \quad a(q,S(q))(\Psi) = 0 \quad 
\forall \Psi \in V.
\end{equation*}
The local existence and sufficient regularity of $S$ is assumed. The 
necessary optimality conditions of first and second order are
\begin{equation*}
j'(q)(\delta q) = 0 , \quad j''(q)(\delta q,\delta q) \geq 0 \quad 
\forall\delta q\in Q.
\end{equation*} 
The derivatives of the reduced functional can be computed using the 
Lagrangian
\begin{equation*}
{\cal L}(q,u,z) =J(q,u) - a(q,u)(z)
\end{equation*}
as already introduced. Let $q\in Q$, and the corresponding state $u=S(q)\in V$ 
be given. To calculate the derivative of the reduced cost functional $j$, 
we introduce the dual variable
$z\in V$ solving the\\[3mm]
%
\textbf{Dual equation}\\
\begin{equation*}
{\cal L}_u'(q,u,z)(\psi) = 0 \quad\forall\psi\in V.
\end{equation*}
Then,
\begin{equation*}
j'(q)(\delta q) = {\cal L}_q' (q,u,z)(\delta q) \quad \text{for }\delta q\in Q.
\end{equation*}
\\[2mm]
%
To calculate the second derivatives, we need to solve additional equations.\\
%
Let $\delta q\in Q$ be a given direction. Then we search $\delta u\in V$  
solving the\\[2mm]
%
\textbf{Tangent equation}\\
%
\begin{equation*}
{\cal L}_{qz}'' (q,u,z)(\delta q,\phi) + {\cal L}_{uz}'' (q,u,z)(\delta
u,\phi) = 0
\quad\forall \phi\in V.
\end{equation*}
Further, we have an auxiliary\\[2mm]  
%
\textbf{Dual for Hessian equation}
to find $\delta z\in V$ solving
\begin{equation*}
{\cal L}_{qu}'' (q,u,z)(\delta q,\phi) + {\cal L}_{uu}'' 
(q,u,z)(\delta u,\phi) + {\cal L}_{zu}'' (q,u,z)(\delta z,\phi) =0 
\quad\forall \phi\in V.
\end{equation*}

Then, for $\delta r\in Q$, we can express the second derivatives of $j$ by
\begin{align*}
j''(q)(\delta q, \delta r) 
&= {\cal L}_{qq}'' (q,u,z)(\delta q,\delta r) \\
&\;\;\;\;+ {\cal L}_{uq}'' (q,u,z)(\delta u,\delta r) \\ 
&\;\;\;\;+ {\cal L}_{zq}'' (q,u,z)(\delta z,\delta r).
\end{align*}
With these terms, we can calculate the Newton direction $\delta q$, at a given 
iterate $q^n$, as solution to the problem
\begin{align*}
j''(q^n )(\delta q,\chi ) =&\, -j'(q^n)(\chi) \quad\forall\chi \in Q.
\end{align*}

Moreover, we would like to work in the Hilbert space $Q$. However, the derivative 
$j'(q) \in H^*$ only. Hence, we need to calculate the Riesz representation 
for the gradient $\nabla j(q) \in H$ using the definition:
\[
 (\nabla j(q),\delta q)_Q = j'(q)(\delta q) \quad \forall\,\delta q\in Q.
\]
In the given example, $Q = L^2(\Omega)$ and hence the scalar product will be the
standard $L^2$-inner product. Similarly, we can define the Hessian operator 
$H(q) \in \mathcal L(Q,Q)$ by defining
\[
 (H(q)\tau q,\delta q)_Q = j''(q)(\tau q,\delta q) \quad \forall\,\delta q,\tau q\in Q.
\]
}

\subsubsection{Implementation in \dope{}}
From the previous details, and the definition of the Lagrangian and its 
derivatives it is clear, that the user has to provide the respective derivatives. 
Since the Lagrangian consists of the PDE and the cost functional it 
is sufficient to provide the respective derivatives, while \texttt{DOpElib} 
will assemble them as required.
Test functions for vector valued terms will be denoted by an index $i$ while 
matrix valued terms are indexed in $i$ and $j$. Test functions in the control 
space $Q$ are denoted as $\phi^q_i$ while those in the state space $V$ are denoted 
as $\phi_i$.

To solve the linear equations the following matrices are needed
\begin{align*}
&\texttt{ElementMatrix} 
&\Leftrightarrow \quad a_{i,j} &= a'_u(q,u)(\phi_j,\phi_i), \\
&\texttt{ControlElementMatrix}  
&\Leftrightarrow \quad a_{i,j} &= (\phi^q_j, \phi^q_i)_Q.
\end{align*}
The first one is required for all primal and dual PDE solves, while the second one is needed 
to calculate the Riesz representation of the derivatives of $j$. 
If desired, the matrix for the adjoint PDEs can be provided separately as 
\texttt{ElementMatrix\underline{ }T}, but otherwise this will be 
calculated automatically from the primal matrix.

Additional terms are needed to calculate the corresponding 
right hand sides. These are for the PDE the following: 
{
\begin{align*}
&\texttt{ElementEquation}\text{ (state)} &\Leftrightarrow &\quad a(q,u)(\phi_i), \\
&\texttt{ElementRightHandSide}\text{ (state)}  &\Leftrightarrow &\quad f(\phi_i), \\
&\texttt{ControlElementEquation}\text{ (gradient or hessian)}  &\Leftrightarrow &\quad (\nabla j(q),\phi^q_i)_Q,
\end{align*}
as well as 
\begin{align*}
&\texttt{ElementEquation\underline{ }U}\text{ (adjoint)} 
&\Leftrightarrow &\quad a_u'(q,u)(\phi_i, z),\\
&\texttt{ElementEquation\underline{ }Q}\text{ (gradient)} 
&\Leftrightarrow &\quad a_q'(q,u)(\phi^q_i,z), \\
\end{align*}
the terms
\begin{align*}
&\texttt{ElementEquation\underline{ }UU}\text{ (adjoint hessian)}  
&\Leftrightarrow &\quad a_{uu}''(q,u)(\delta u, \phi_i, z),\\
&\texttt{ElementEquation\underline{ }UQ}\text{ (hessian)}
&\Leftrightarrow &\quad a_{uq}''(q,u)(\delta u, \phi^q_i, z),\\
&\texttt{ElementEquation\underline{ }QU}\text{ (adjoint hessian)} 
&\Leftrightarrow &\quad a_{qu}''(q,u)(\delta q, \phi_i, z), \\
&\texttt{ElementEquation\underline{ }QQ}\text{ (hessian)} 
&\Leftrightarrow &\quad a_{qq}''(q,u)(\delta q, \phi^q_i, z),\\
&\texttt{ElementEquation\underline{ }UT}\text{ (tangent)}  
&\Leftrightarrow &\quad a_u'(q,u)(\delta u, \phi_i), \\
&\texttt{ElementEquation\underline{ }QT}\text{ (tangent)}  
&\Leftrightarrow &\quad a_q'(q,u)(\delta q, \phi_i),
\end{align*}
and finally
\begin{align*}
&\texttt{ElementEquation\underline{ }UTT}\text{ (adjoint hessian)} \quad 
&\Leftrightarrow &\quad  a_u'(q,u)(\phi_i, \delta z), \\
&\texttt{ElementEquation\underline{ }QTT}\text{ (hessian)} \quad 
&\Leftrightarrow &\quad a_{q}'(q,u)(\phi^q_i,\delta z).
\end{align*}}
As for PDE problems, it is up to the user to decide if the \texttt{ElementRightHandSide}
is used, or if the terms are included in the \texttt{ElementEquation}. 

For the cost functional, we have to provide{
\begin{align*}
&\texttt{ElementValue}\text{ (all)}   &\Leftrightarrow &\quad J(q,u), \\
&\texttt{ElementValue\underline{ }U}\text{ (all)}    &\Leftrightarrow &\quad J_u'(q,u)(\phi_i), \\
&\texttt{ElementValue\underline{ }Q}\text{ (all)}    &\Leftrightarrow &\quad J_q'(q,u)(\phi_i), \\
&\texttt{ElementValue\underline{ }UU}\text{ (all)}  &\Leftrightarrow &\quad J_{uu}''(q,u)(\delta u, \phi_i), \\
&\texttt{ElementValue\underline{ }UQ}\text{ (all)}  &\Leftrightarrow &\quad J_{uq}''(q,u)(\delta u, \phi^q_i), \\
&\texttt{ElementValue\underline{ }QU}\text{ (all)}   &\Leftrightarrow &\quad J_{qu}''(q,u)(\delta q, \phi_i), \\
&\texttt{ElementValue\underline{ }QQ}\text{ (all)}  &\Leftrightarrow &\quad J_{qq}''(q,u)(\delta q, \phi^q_i).
\end{align*}}

Clearly, if the PDE or cost functional contains other terms, such as boundary or face integrals 
corresponding derivatives must be provided as well.

\subsubsection{Back to the specific equations in this example}
We have
\begin{align*}
a(q,u)(\phi) &= (\nabla u, \nabla\phi) - (q + f, \phi),\\
a_u'(q,u)(\phi, z) &= (\nabla\phi, \nabla z),\\
a_u'(q,u)(\delta u, \phi) &= (\nabla\delta u, \nabla\phi),\\
a_u'(q,u)(\phi, \delta z) &= (\nabla\phi, \nabla\delta z),\\
a_q'(q,u)(\delta q, \phi) &= -(z,\psi^q),\\
a_q'(q,u)(\delta q, \phi) &= (\delta q,\phi),\\
a_{q}'(q,u)(\delta q,\delta z) &= - (\delta z,\psi^q).
\end{align*}
%
For the cost functional, we have the following terms:
\begin{align*}
J(q,u) &= \frac{1}{2} \|u-u^d\|^2 + \frac{\alpha}{2}\|q\|^2,\\
J_u'(q,u)(\phi) &= (u-u^d, \phi),\\
J_q'(q,u)(\phi) &= \alpha (q,\psi^q),\\
J_{uu}''(q,u)(\psi, \phi) &= (\delta u, \phi).
\end{align*}
All other terms, specifically mixed terms with $QU$ etc. are zero in this
example. 

\subsubsection{\texttt{main.cc}}
Finally, the main file of the optimization examples does not look very much 
different than for pure PDE computations - which is one of the crucial aims 
of our library. 
Here, instead of using a \texttt{pdeproblemcontainer}, we use 
now an \texttt{optproblemcontainer} which can assemble all additionally needed information,
such as adjoint and tangent PDEs. 
Furthermore, we define \texttt{ReducedNewtonAlgorithm} and 
\texttt{ReducedTrustregion\underline{ }NewtonAlgorithm}
to solve the optimization problem with a linesearch and a trust-region Newton algorithm.
Of course one would be sufficient, but we wanted to show how to change optimization 
solvers easily using \texttt{DOpElib}.\\
Next, in the body of the main file, we introduce a second FE function for the 
control variable. Then, we define 
a \texttt{COSTFUNCTIONAL}. Finally, the problem is either solved 
by calling \texttt{Alg.Solve(q)} and/or the user might check if the 
derivatives are implemented correctly by calling 
\texttt{Alg.CheckGrads} or \texttt{Alg.CheckHessian}. The latter 
two functionalities are highly recommended to check your
implementation before wondering about your results.\\
Finally, this example uses a DWR-error estimator to estimate the 
error made in the cost functional. In contrast to the 
error estimation for PDEs here, we have to include the error in the
control by using the \texttt{HigherOrderDWRContainerControl}.
