\subsubsection{General problem description}
This example is similar to Example~\ref{OPT_Stat_Param_Nonlin_Fluid}.
The notable difference in the setting is that now, the boundary control areas, 
where the fluid can be sucked out of the domain, are in front of the 
circular inclusion. Hence the drag
\[
k(v,p) = \int_{\Gamma_O} n\cdot \sigma(v,p)\cdot d \, \mathrm{d}s,
\]
where $\Gamma_O$ denotes the cylinder boundary, and $d$ is a vector in the
direction of the mean in-flow can become negative, i.e., minimizing the 
value of $k(v,p)$ is no longer viable. 

Instead, we consider the objective 
\[
K(q,v,p) = \frac{1}{2}|k(v,p)|^2 + \frac{\alpha}{2}||q - q_0||^2
\] 
is to be minimized.

In contrast to all previous examples, this means that the functional can
no longer be calculated by one integration, but instead the values of the 
integration (for the drag) need to be post-processed.

To do so, the calculation of the functional and its derivatives
is reordered in to two steps. First the value of $k(v,p)$ is
calculated (and stored) then in a second sweep. The value of
$K$ is calculated.

To this end the following modifications are needed in the
code:


\textbf{\texttt{localfunctional.h}}
There is a new method \texttt{unsigned int NeedPrecomputations() const}
returning the value $1$ as we need one calculation of $k$ before we can
assemble the value of the cost-functional.

This pre-iteration has the Type \texttt{cost\_functional\_pre} and a corresponding
number (here $0$ as only one pre-iteration is performed.

In the cost functional, for the pre-iteration we set the type to
\texttt{boundary} since $k$ is a boundary functional.
For the evaluation of $K$ itself the type is \texttt{boundary algebraic}
as we calculate the boundary integral $\|q\|^2$ and the algebraic calculation
$|{k}|^2$.

For higher derivatives, we notice that for a differentiable functions $g,f\colon \mathbb R\rightarrow \mathbb R$
it holds for the directional derivative in direction $\delta u$
\begin{align*}
 \left((g\left(\int f(u(x))\,\mathrm{d}x\right)\right)'\delta u
  &= g'(\int f(u(x))\,\mathrm{d}x) \int f'(u(x)) \delta u(x) \mathrm{d}x \\
  &= \int g'(\int f(u(x))\,\mathrm{d}x) f'(u(x)) \delta u(x) \mathrm{d}x 
\end{align*}
Consequently, the first derivative can be calculated with only a single integration
-- as a boundary integral in the present example where the 
factor $g'$ can be calculated using the drag value in the 
last iterate  
\[
g'(k(v,p)) \quad\text{and}\quad k'(v,p)\delta u(x) = k(\delta v, \delta p)
\]
since the drag is linear in $(v,p)$.

For the second derivative, we can calculate in the directions $\delta u$ and $\tau u$
by the following observation 
\begin{align*}
 \left(g\left(\int f(u(x))\,\mathrm{d}x\right)\right)''&(\delta u,\tau u)\\
  &= g''(\int f(u(x))\,\mathrm{d}x) \int f'(u(x)) \delta u(x) \mathrm{d}x\int f'(u(x)) \tau u(x) \mathrm{d}x \\
  &\;\;\;\;+ g'(\int f(u(x))\,\mathrm{d}x) \int f''(u(x)) \delta u(x) \tau u(x) \mathrm{d}x \\
  &= \int \left(g''(\int f(u(x))\,\mathrm{d}x) \int f'(u(x)) \delta u(x) \mathrm{d}x\right)
 f'(u(x)) \tau u(x) \mathrm{d}x \\
 &\;\;\;\;+ \int g'(\int f(u(x))\,\mathrm{d}x)f''(u(x)) \delta u(x) \tau u(x) \mathrm{d}x.
\end{align*}
Consequently, the second derivative can be calculated as one boundary integral, if 
the values of 
\[
\int f(u(x))\,\mathrm{d}x \quad \text{and} \quad \int f'(u(x)) \delta u(x) \mathrm{d}x
\]
in the tangent direction $\delta u$ are available by pre-computations.

To do so in the present example, the following modifications are needed in the cost functional:
\begin{itemize}
\item There is a method \texttt{AlgebraicValue} which calculates $0.5 x^2$ for a given value $x$
- here the pre-computed value of the drag.
\item In the method \texttt{BoundaryValue}, we have to distinguish several cases.
  based upon the problem type evaluated. The current value can be accessed by
  \texttt{GetProblemType()} and if this matches
  \begin{description}
    \item[cost\_functional\_pre] then we must calculate the drag-value, i.e., $k(u)$.
    \item[cost\_functional\_pre\_tangent] then we must calculate the derivative of 
      the drag in direction $\delta u$ (the provided tangent-direction), i.e., $k(\delta u)$ 
      since $k$ is linear in its argument.
    \item[cost\_functional] then we must calculate the rest of the functional. Here only the integral
      over the control costs is needed.
  \end{description}
\item In the methods \texttt{BoundaryValue\_U} and \texttt{BoundaryValue\_UU} we calculate
  the entire derivative w.r.t $u$ (or second derivative respectively) as one integral 
  using the formulas for the first and second derivative, respectively. The needed
  values, i.e.,
  \[
  \int f(u)\mathrm{d}x = \int_{\Gamma_O} n \cdot \sigma(v,p)\,d \mathrm{d}s 
  \]
  and 
  \[
  \int f'(u(x)) \delta u(x) \mathrm{d}x = \int_{\Gamma_O} n \cdot \sigma(\delta v,\delta p)\,d \mathrm{d}s 
  \]
  are available as \texttt{ParamValues} with the respective labels 
  \texttt{cost\_functional\_pre} and \texttt{cost\_functional\_pre\_tangent}.

  Notice, that both \texttt{ParamValues} are a vector of the size given 
  by \texttt{unsigned int NeedPrecomputations() const}. I.e. if multiple functional parts 
  need a pre-integration these can be calculated in an arbitrary number of pre-integration runs.
  In the case of multiple pre-integrations the method \texttt{*Value} needs not only to
  consider the value of \texttt{GetProblemType()} but also the respective \texttt{GetProblemNum()}
  running between $0$ and \texttt{NeedPrecomputations()}$-1$. 
  The order in the ParamValues Vector corresponds to the chosen order of integration.
\end{itemize}
