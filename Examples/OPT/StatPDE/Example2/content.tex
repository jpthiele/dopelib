\subsubsection{General problem description}
This example solves the following pointwise minimization problem
\begin{align*}
\min_{(q,u)\in \mathbb R^3 \times H_0^1(\Omega; \mathbb R^2)} J(q,u) &= \frac{1}{2} \sum_{i=0}^2\abs{(u-\overline u)(x_i)}^2 + \frac{\alpha}{2}\|q\|^2&&\\
\text{s.t. } (\nabla u,\nabla \phi) &= (f(q),\phi) \quad\forall\,\phi \in H^1_0(\Omega; \mathbb R^2)
\end{align*}
on the domain $\Omega = [0,1]^2$, with zero Dirichlet boundary conditions and
\begin{itemize}
\item the observation points
\begin{align*}
x_0 = (0.5, 0.5), \quad x_1 = (0.5, 0.25),\quad x_2 = (0.25, 0.25),
\end{align*}
\item the regularization parameter $\alpha = 0$, 
\item the right hand side
\begin{align*}
 f(q) &= q_0 \left(\begin{matrix}2\pi^2  \sin( \pi x) \sin(\pi y)\\0 \end{matrix}\right)\\
      &+ q_1 \left(\begin{matrix}5\pi^2  \sin( \pi x) \sin(2\pi y)\\0 \end{matrix}\right)\\
      &+ q_2 \left(\begin{matrix}0 \\8\pi^2  \sin(2\pi x) \sin(2\pi y)\end{matrix}\right)\\
\end{align*}
\item and the exact solution given by 
\begin{align*}
 \overline{q} &= \bigl(1;0.5;1\bigr)\\
 \overline{u}& = \left(\begin{matrix} \sin( \pi x)( \sin(\pi y)+0.5\sin(2\pi y))\\\sin(2\pi x) \sin(2\pi y) \end{matrix}\right).
\end{align*}
\end{itemize}
\subsubsection{Program description}
{
In contrast to the first example, the control is now a discrete quantity (for
the three observation points) where we use the \texttt{FE\underline{}Nothing}
element to assign the three controls. Here, the number of components equals
the number of controls. 
In addition, }notice that the cost functional is of mixed type (from our computational point of view), i.e. the first part is a pointfunctional whereas the regularization part requires the evaluation of a domain integral. To handle this, we need a special integrator as well as a special newtonsolver. Additionally, our \texttt{LocalFunctional} returns as his type:
\begin{verbatim}
      string
      GetType() const
      {
        return "point domain";
      }
\end{verbatim}
Indicating, we have both some point values in the functional, as well as a domain contribution, 
i.e., we calculate $\|q\| = \int_\Omega |q|\,dx$,
even though this in not necessary, since $\alpha =0$ and the euclidean 
norm of $q\in \mathbb R^3$ could be evaluated more easily using an \text{AlgebraicValue} in the functional.
  
This brings along that we have not only to implement {the methods
\begin{align*} 
 &\texttt{ElementValue},&&
 \texttt{ElementValue\_U},\\
 &\texttt{ElementValue\_Q}, &&
 \texttt{ElementValue\_UU},\\
 &\texttt{ElementValue\_UQ},&&
 \texttt{ElementValue\_QU},\\
 &\texttt{ElementValue\_QQ},
\end{align*}}
 from \texttt{FunctionalInterface}, but also all the aforementioned methods with a preceding \text{Point} (\texttt{PointValue} etc.).
%FE_Nothing, handle of the control erlaeutern.

