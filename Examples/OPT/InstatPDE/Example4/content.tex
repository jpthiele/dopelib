\subsubsection{General problem description}

This example is a modified version of \texttt{OPT/InstatPDE/Example1}. 
Again, we consider the heat equation, this time with an additional nonlinear
term and most important a time derivative leading to a 
nonstationary optimization problem. The governing equation is
\begin{equation*}
\partial_t u(t,x,y) - \Delta u(t,x,y) + u(t,x,y)^2 = f(t,x,y),
\end{equation*}
with homogeneous Dirichlet-data.
The computational domain is $\Omega \times I = [0,\pi]^2 \times [0,1]$ with the data
\begin{align*}
f(t,x,y) &= (3-2t)e^{t-t^2} \sin(x) \sin(y) + e^{(t-t^2)^2} \sin^2(x) \sin^2(y),\\
u_0(x,y) &= q(x,y).
\end{align*}

In contrast to \texttt{OPT/InstatPDE/Example1}, we modify the cost-functional
such that a (spatial) mean-value of zero is desired for the state.
The corresponding functional is
\[
 \min_{q,u} J(q,u) = \int_0^1 \frac{1}{2}\left|\int_{\Omega} u(t,x,y) \,\mathrm{d}(x,y)\right|^2 \,\mathrm{d}t + \frac{1}{2} \int_{\Omega} (q(x,y) - \sin(x) \sin(y))^2\,\mathrm{d}(x,y).
\]
\subsubsection{Program description}
The following things differ from the previous examples:\\[2mm]
We are now considering a cost-functional, that can not be evaluated as one single integral. Instead,
we need to evaluate the integral $\int_{\Omega} u(t,x,y) \,d(x,y)$ in space first in each time step and 
then integrate over its absolute value in time. We have seen a similar structure in 
\texttt{OPT/StatPDE/Example10}~\ref{OPT_Stat_Param_Nonlin_Fluid_extension}. Here, the integral 
in question is of the form
\[
\int_I g(\int f(u(t,x)) \mathrm{d}x) \mathrm{d}t
\]
Consequently, we need to set \texttt{unsigned int NeedPrecomputations() const} to the value $1$ 
as we need to calculate -- in each time point -- the values $\tilde{f}(t) = \int f(u(t,x)) \mathrm{d}x$ 
prior to the evaluation of the cost functional.
Then the first derivative of the functional in the direction $\delta u$ is given as 
\begin{align*}
   \left(\int_I(g\left(\tilde{f}(t)\right)\mathrm{d}t\right)'\delta u
  &= \int_I g'(\tilde{f}(t)) \int f'(u(t,x)) \delta u(t,x) \mathrm{d}x\;\mathrm{d}t \\
  &= \int_I \int g'(\tilde{f}(t)) f'(u(t,x)) \delta u(t,x) \mathrm{d}x\;\mathrm{d}t
\end{align*}
And for the second derivative, we also calculate the tangential derivatives 
\[
\tilde{f}_{u}(t) = \int f'(u(t,x)) \delta u(t,x) \mathrm{d}x = \int_{\Omega} \delta u(t,x,y) \,\mathrm{d}(x,y).
\]
With this the second derivative in direction $\delta u $ and $\tau u$ is given as 
\begin{align*}
   \Biggl(\int_Ig&\left(\tilde{f}(t)\right)\mathrm{d}t\Biggr)''(\delta u,\tau u)\\
  &= \int_I g''(\tilde{f}(t)) \tilde{f}_{u}(t) \int f'(u(t,x)) \tau u(t,x) \mathrm{d}x\;\mathrm{d}t \\
  &\;\;\;\;+ \int_I \int g'(\tilde{f}(t)) f''(u(t,x)) \delta u(t,x) \tau u(t,x) \mathrm{d}x\;\mathrm{d}t\\
&= \int_I \int \left( g''(\tilde{f}(t)) \tilde{f}_{u}(t)f'(u(t,x))+ g'(\tilde{f}(t)) f''(u(t,x)) \delta u(t,x) \right) \tau u(t,x) \mathrm{d}x\;\mathrm{d}t.
\end{align*}
As in \texttt{OPT/StatPDE/Example10} the higher derivatives can be calculated by one space-time 
integral if the needed values $\tilde{f}$ and $\tilde{f}_u$ are available at the 
different time-points. These values can be accessed as ParamValues with the respective 
labels \texttt{cost\_functional\_pre} and \texttt{cost\_functional\_pre\_tangent}.

One further thing differs from the previous examples. This is that the cost functional 
now has one part being integrated over space-time while the other 
is acting in space (at initial-time) only. The two types of functionals can at present 
not be mixed. Hence both terms are evaluated as a distributed integral 
over space-time by adding an additional weight to the control-cost integral.
This weight needs to be specially modified for the derivatives w.r.t. the control
which are evaluated at the initial time only!
This modification is not exact and consequently the derivatives are not exactly calculated. 
The accuracy is increased as the temporal meshes are refined. Hence, if more accuracy 
in the residual is desired the temporal mesh needs to be further refined. 